% Plantilla realizada por Alberto Brunete (UPM). Basada en la de Santiago Morante Cendrero (UC3M)

%parametros de tipo libro
\documentclass[10pt,a4paper]{book}

%idioma español y acentos
\usepackage[spanish, es-tabla]{babel}
%\usepackage[latin1]{inputenc}
\usepackage[utf8]{inputenc}

%algunos símbolos matemáticos y paquete para usar subimágenes
\usepackage{amsmath}
\usepackage{amsfonts}
\usepackage{amssymb}
\usepackage{graphicx}
\usepackage{subfigure}
\usepackage{listings}
\usepackage[title,titletoc,toc]{appendix}
%Márgenes
\usepackage[left=3cm,top=3cm,right=3cm,bottom=3cm]{geometry}
\usepackage{url}

\usepackage{pdflscape}
\usepackage{changepage}
%
\usepackage{multicol}

\usepackage{longtable}

\usepackage{float}
\usepackage{eurosym}
\usepackage[usenames]{color}

 \definecolor{pRojo}{RGB}{255,30,30}
 \definecolor{pAmarillo}{RGB}{255,203,15}
 \definecolor{pAzul}{RGB}{60,145,188}
 \definecolor{pVerde}{RGB}{92,196,123}
 \definecolor{pGris}{RGB}{69,69,69}

 %para generar índice con hipervínculos
 \usepackage{hyperref}
 \usepackage{tabularx}
 \usepackage{multirow}
 \usepackage{rotating}
 \usepackage{caption}
 \usepackage{pdfpages}

\usepackage[xindy,toc,acronyms,nonumberlist]{glossaries}
 %\usepackage[xindy,acronym,toc,nonumberlist,sort=use]{glossaries} % nomain, if you define glossaries in a file, and you use \include{INP-00-glossary}
 \makeglossaries
 \usepackage[xindy]{imakeidx}
 \makeindex

 \loadglsentries[main]{capitulos/glosario}
% or using \input:
%\input{INP-00-glossary}

\graphicspath{figuras/} %path de los gráficos

%parametros del documento (sus propiedades)
\hypersetup{
    pdftitle={Nombre del alumno - TFG - año},
    pdfsubject={TFG - año},
    pdfauthor={Nombre del alumno},
    pdfkeywords={palabraclave1} {palabraclave2} {palabraclave3},
    colorlinks,
    citecolor=black,
    filecolor=black,
    linkcolor=black,
    urlcolor=black,
}


% %%%%%%%%% SUPER COMANDOS %%%%%%%%%%
\newcommand{\completar}{\textcolor{pRojo}{\small{\textbf{-- COMPLETAR --}}}}
\newcommand{\completarCon}[1]{ \textcolor{pRojo}{\small{\MakeUppercase{\textbf{#1}}}}}
\newcommand{\ingles}[1]{\textit{#1}}
\newcommand{\glosario}[1]{\ingles{\gls{#1}}}
\newcommand{\glosarioPlural}[1]{\ingles{\glspl{#1}}}
\newcommand{\minititulo}[1]{\subsection{#1}}
%\newcommand{\minititulo}[1]{{\vspace{\baselineskip}
%                            {\Large\textbf{#1}}
%                            \vspace{0.5\baselineskip plus 0pt}
%                            \normalsize}
%                            }
\newcommand{\codigo}[1]{\ttfamily{#1}\normalfont}
\newcommand{\iconoImagen}[1]{\includegraphics[width=0.25\textwidth]{figuras/Imagenes_Piezas/#1.PNG}}

\usepackage{graphicx,eso-pic}

\newcommand{\chapterimage}[1]{
	\AddToShipoutPictureBG*{% Add picture to BackGround on this page only
		\AtPageUpperLeft{% Position at upper left of text block
			\vspace{2cm}
			\hspace*{1.2\textwidth}% Move over to upper right of text block
			\llap{% Ignore horizontal width and overlap to the left
				\smash{% Ignore vertical height
					\raisebox{-\height}{% Lower so top touches baseline
						\includegraphics[width=12cm]{#1}}}}}}}% Include image with options
%width height
% %%%%%%%%%%%%%%%%%%%%%%%%%%%%%%%%%%%
\newcommand{\immagesource}[1]{
	\vspace{-10pt}
	\captionsetup{font={footnotesize,bf,it}}
	\caption*{\hfill Fuente: #1}
	}

%empieza el documento
\begin{document}
% %elementos antes del trabajo en sí se meten dentro de frontmatter
 \frontmatter

% %cada incluye referencia a un archivo de tipo .tex
 \begin{titlepage}
\begin{center}

%forma de introducir imágenes. el \\[0.5 cm] de final de línea introduce un salto de ese tamaño.
%width=1\textwidth indica el tamaño de la imágen (valores entre 0-1). 
 \includegraphics[width=1\textwidth]{figuras/cabecera.png}  \\[0.3 cm]

\Large UNIVERSIDAD POLITÉCNICA DE MADRID \\ [0.8 cm]

\Large ESCUELA TÉCNICA SUPERIOR DE INGENIERÍA Y DISEÑO INDUSTRIAL \\ [0.8 cm]

\LARGE Grado en Ingeniería Electrónica y Automática Industrial\\ [0.8 cm]

\LARGE \textbf{TRABAJO FIN DE GRADO}\\[0.8 cm]

\Huge \textsc{Diseño y construcción de un brazo robótico para el apoyo a la interactividad de personas enfermas}\\[0.8 cm]

\LARGE Enrique Heredia Aguado \\[1.7 cm]

%flushleft alinea a la izquierda el texto

\begin{multicols}{2} 
\begin{flushleft} \Large
\emph{Cotutor:} Alberto Brunete González \\
\emph{Departamento:} Ingeniería eléctrica, electrónica automática y física aplicada
\end{flushleft}

\begin{flushleft} \Large
\emph{Tutor:} Miguel Hernando Gutiérrez\\
\emph{Departamento:} Ingeniería eléctrica, electrónica automática y física aplicada
\end{flushleft}

\end{multicols} 

%rellena de blanco el resto de la página para escribir abajo del todo
\vfill

% Bottom of the page
{\large Madrid, Febrero 2018}

\end{center}
\end{titlepage}

 \begin{titlepage}
\begin{center}

%forma de introducir imágenes. el \\[0.5 cm] de final de línea introduce un salto de ese tamaño.
%width=1\textwidth indica el tamaño de la imágen (valores entre 0-1). 
 \includegraphics[width=1\textwidth]{figuras/cabecera.png}  \\[0.5 cm]

\LARGE UNIVERSIDAD POLITÉCNICA DE MADRID \\ [1 cm]

\LARGE ESCUELA TÉCNICA SUPERIOR DE INGENIERÍA Y DISEñO INDUSTRIAL \\ [1 cm]

\LARGE Grado en Ingeniería Electrónica y Automática Industrial\\ [1 cm]

\LARGE \textbf{TRABAJO FIN DE GRADO}\\[1 cm]

\Huge \textsc{Diseño y construcción de un brazo robótico para el apoyo a la interactividad de personas enfermas}\\[3 cm]

\Large Firma Autor \\[2 cm]

%flushleft alinea a la izquierda el texto

\begin{multicols}{2} 
\begin{flushleft} 
\Large \emph{Firma Cotutor}
\end{flushleft}

\begin{flushright} 
\Large \emph{Firma Tutor}
\end{flushright}

\end{multicols} 

%rellena de blanco el resto de la página para escribir abajo del todo
\vfill

\end{center}
\end{titlepage}

 %Licencia opcional
 %\begin{flushleft}

Copyright \copyright  2018. Enrique Heredia Aguado

%ejemplo de licencia, se puede elegir cualquier otra

Esta obra está licenciada bajo la licencia Creative Commons Atribución-CompartirIgual 4.0 International (CC BY-SA 4.0). Para ver una copia del resumen de esta licencia, visite https://creativecommons.org/licenses/by-nd/4.0/. Puede acceder al código legal de la misma en https://creativecommons.org/licenses/by-nd/4.0/legalcode


\includegraphics[width=0.3\textwidth]{figuras/ccbysa.png}


Todas las opiniones aquí expresadas son del autor, y no reflejan necesariamente las opiniones
de la Universidad Politécnica de Madrid.

\end{flushleft}


 \cleardoublepage

\begin{flushleft} \large
\textbf{Título:} Diseño y construcción de un brazo robótico para el apoyo a la interactividad de personas enfermas \\
\textbf{Autor:} Henrique Heredia Aguado\\
\textbf{Tutor:} Miguel Hernando Gutiérrez \\ 
\textbf{Cotutor:} Alberto Brunete González\\ [1 cm]

\end{flushleft} 

\begin{center} \LARGE
EL TRIBUNAL \\ [1 cm]
\end{center}

\begin{flushleft} \LARGE
Presidente: Miguel Hernando Gutiérrez \\ [1 cm]
Vocal: Alberto Brunete González \\ [1 cm]
Secretario: Cecilia García Cena \\ [1.5 cm]
\end{flushleft}

\large
Realizado el acto de defensa y lectura del Trabajo Fin de Grado el día 28 de Febrero de 2018 en Madrid, en la Escuela Técnica Superior de Ingeniería y Diseño Industrial de la Universidad Politécnica de Madrid, acuerda otorgarle la CALIFICACIÓN de: \\ [2 cm]

\begin{center}
 \large VOCAL \\ [2.2 cm]
\end{center}

\begin{minipage}{0.5\textwidth}
 \begin{flushleft}
 \large SECRETARIO
\end{flushleft}
\end{minipage}
\begin{minipage}{0.5\textwidth}
\begin{flushright}
 \large PRESIDENTE
\end{flushright} 
\end{minipage}

 \chapter{Agradecimientos} \label{chap:Agradecimientos}


Agradecimiento especial a mi familia por el apoyo incondicional desde las pequeñas a las grandes cosas.
\\

En concreto agradecer a Miguel Hernando, tutor a cargo de este proyecto, toda la ayuda, ilusión y conocimientos transmitidos antes, durante y seguramente después de finalizar el mismo.
\\

También agradecer a amigos y compañeros de HackLab con los que, en mayor o menor medida, he podido compartir esta experiencia.

 %chapter introduce un nuevo capítulo
\chapter{Resumen}

Este proyecto se resume en el diseño y construcción de un brazo robótico asistencial para su inclusión en entornos hospitalarios. Junto con una tablet funcionarán como interfaz entre pacientes con movilidad reducida de forma que puedan interactuar con el entorno.

\paragraph{Palabras clave:} brazo robótico, robótica asistencial.

\chapter{Abstract}

This project covers from the design to the implementation and construction of a roboti arm aimed to be used in hospitals. Along with a tablet device it will be used as an interface for pacients in bed to interact with the environment.

\paragraph{Keywords:} robotic arm, robotic assistance.

 %genera índice
 \tableofcontents

% %índice de figuras.
 \listoffigures

 %índice de tablas.
 \listoftables

 %\glsaddall
 %\printglossary[type={main,acronym}]
 %\printglossary[type=\acronymtype]
 \printglossaries
% TODO: \glsaddall

 %empieza la parte descriptiva del trabajo
 \mainmatter

 \chapter{Introducción} \label{chap:Introduccion}
\chapterimage{figuras/ImagenesPortada/PortadaIntro.jpg}
\hrule
\vspace{3mm}

En este capítulo se perfila la estructura, organización y contenidos principales del documento así como la motivación y objetivos del proyecto.


\section{Motivación del proyecto}



\section{Objetivos}

El objetivo que este Trabajo de Fin de Grado persigue es el del diseño, construcción y control de brazo robótico previo estudio de las opciones comerciales disponibles y su posible adaptación. Este proyecto está enmarcado bajo el proyecto Robohealth, proyecto financiado por el Ministerio de Economía y Competitividad con el objetivo del diseño de sistemas robóticos y domóticos para entornos hospitalarios que mejoren el sistema de salud actual. (\completar http://robohealth.es/)
\\

El  prototipo debe estar diseñado para una completa adaptación a un entorno hospitalario en el que deberá estar en contacto constante con usuarios a los que deberá respetar.
\\

El objetivo del brazo robótico es el de ubicar ante un paciente una \ingles{tablet}. Ésta llevará montado un dispositivo de seguimiento de vista de forma que, a través del movimiento de las pupilas, el paciente podrá interactuar con el resto de dispositivos de la habitación así como el personal médico. Es necesario motorizar el dispositivo para mantener el dispositivo de seguimiento siempre a una distancia y ángulo, respecto a la cara del paciente, mínima para facilitar el funcionamiento del mismo. Está pensado principalmente para pacientes con movilidad reducida o sin movilidad (temporal o permanente), aunque también podría agilizar la interfaz humano-habitación para el resto de pacientes.
\\

Así pues, algunos aspectos claves del prototipo deben ser:
\begin{itemize}
    \item El Objetivo del prototipo será el de permitir una interacción más cómoda y automatizada entre los pacientes cuya capacidad de interacción se ha visto reducida por la causa que sea.
    \item Se debe tener en cuenta es que el prototipo estará en constante contacto con gran variedad de usuarios: pacientes, médicos, familiares, etc. El diseño debe proteger en todo momento la seguridad de dichos usuarios
    \item Concretamente el diseño está pensado para interactuar con pacientes que se encuentran recostados en una camilla en el hospital.
\end{itemize}

\subsection{Objetivos derivados}

La realización de dicho prototipo implica el cumplimiento de otros objetivos secundarios o derivados del principal. Se pueden listar algunos como:
\begin{itemize}
    \item Prueba de diferentes tipos de estructuras y materiales como base física del brazo robótico.
    \item Adquisición de conocimientos sobre modelado 3D digital así como diferentes métodos de fabricación como son la impresión 3D, el corte láser, fresado CNC así como el uso de otras herramientas de mecanizado más tradicionales.
    \item Diseño e implementación de un sistema de control en cascada, que permita un control en posición y en velocidad del brazo robótico.
\end{itemize}


\section{Estructura del documento}

El documento está organizado de tal forma que irá introduciendo al lector progresivamente en los diferentes aspectos del diseño y montaje del prototipo mencionado, desde aspectos más generales hasta los más técnicos.
\\

Los capítulos están a su vez organizados en el orden que se seguiría de cara a montar el prototipo empezando por una base física, añadiendo a posteriori los componentes electromecánicos para finalizar con los aspectos de control. Concretamente los capítulos contienen la siguiente información:

\begin{itemize}
    \item Como continuación de los requerimientos generales se encuentra descritos en la introducción, en el capítulo \ref{chap:estadoarte}, un estudio de diferentes modelos y diseños comerciales que podrían adaptarse para cumplir los objetivos presentados.
    \item El estudio del estado del arte ayuda a definir las ideas más importantes que regirán el diseño del prototipo. Éstas consideraciones se presentan en el capítulo \ref{chap:Punto_partida}.
    \item El capítulo \ref{chap:Mecanica} entra de lleno en los aspectos mecánicos del brazo robótico, pasando por una valoración de distintas posibilidades para llegar al diseño definitivo.
    \item Continuando con la descripción de soporte físico, el capítulo \ref{chap:Cinematica} supone un cambio de perspectiva que guiará al lector desde la parte mecánica y física expuesta anteriormente a la descripción matemática y cinemático del modelo.
    \item Una vez descrito el soporte físico del brazo robótico se detalla el hardware escogido para su puesta en marcha. En el capítulo \ref{chap:Electronica} presentan los componentes electromecánicos que se han integrado en prototipo; sus características principales así como su ubicación en el montaje.
    \item El análisis de la estructura software diseñado queda cubierto en el capítulo \ref{chap:SW}. Este apartado anticipa además algunas ideas sobre el sistema de control diseñado.
    \item Continuando con las pinceladas aportadas en el apartado anterior, el capítulo \ref{chap:Control} expone de forma detallada los distintos aspectos de diseño y desarrollo del control del brazo.
    \item Una vez alcanzado este punto, habiendo cubierto los aspectos del diseño del brazo, el capítulo \ref{chap:Resultados} recoge los resultados de funcionamiento de prototipo para ser analizados.
    \item No se dejan de lado aspectos de gestión, costes y viabilidad del prototipo que se detallan en el capítulo \ref{chap:Gestion}.
    \item Finalmente, el capítulo \ref{chap:Conclusiones} expone las conclusiones finales del trabajo así como posibles desarrollos futuros.
\end{itemize}

Como complemento a la información que exponen los apartados de esta memoria se añaden al final diferentes anexos:

\begin{itemize}
    \item Se adjunta un listado de todas las piezas diseñadas con información relevante sobre su uso y fabricación en el anexo \ref{app:listadoPiezas}.
    \item Vistas las piezas que conforman el prototipo es importante localizarlas para entender su uso y diseño. El anexo \ref{app:montajePiezas} detalla unas pautas y orden para el ensamblado del prototipo. Además permite localizar en su contexto las piezas listadas en el anexo \ref{app:listadoPiezas}.
    \item Volviendo sobre los aspectos del software, en el Anexo \ref{app:codificacionSW} se concretan las reglas de codificación, mencionadas en el capítulo \ref{chap:SW}, más relevantes que se han aplicado en el desarrollo del código.
    \item De igual forma se amplia la información sobre el software desarrollado en el anexo \ref{app:documentacion_software}. En él se encuentra la documentación del software generada a través de la herramienta \glosario{doxygen}.
    \item El desarrollo del proyecto conlleva la utilización de diferentes herramientas software cuya instalación y puesta apunto se detalla en el anexo \ref{app:instalacion_software}.
\end{itemize}

\section{Software utilizado}

Aunque en el anexo \ref{app:instalacion_software} se detalla la instalación del software en más detalle no está de más conocer las herramientas a utilizar de antemano, ya que éstas marcan unas pautas en la ideología de diseño y una estructura a la hora de ordenar y desarrollar el proyecto.

\begin{itemize}
    \item Autodesk Inventor 2016: Es un software de modelado paramétrico 3D de la compañía Autodesk Inc.
    \item Atom: Se trata de un editor de texto \ingles{open source}. Permite la instalación de diferentes extensiones para ampliar sus utilidades, entre otras será necesario instalar \glosario{PlatformIO}, que convierte el editor en un \glosario{IDE} completo para el desarrollo de software para diferentes placas como Arduino, que será la base de este proyecto.
    \item Matlab: herramienta de cálculo utilizada para analizar la información recogida del ámbito de control así como para las comprobaciones pertinentes sobre la cinemática.
    \item Lizard: Software que permite el análisis de la complejidad de código. Se compone de una serie de scripts en python que, al ser ejecutados devuelven un fichero con métricas de complejidad referentes a los ficheros de código sobre los que se invoca: complejidad ciclomática, número de funciones en cada fichero, líneas de código en cada función y fichero, entre otras..
    \item Cloc: Es una herramienta sencilla que cuenta, de forma más precisa, el número de líneas de código, comentarios y líneas en blanco de los ficheros de código.
    \item cpplint: Análisis del cumplimiento de las reglas de codificación en el software. Es una herramienta desarrollada en python por Google LLC para asegurar que los proyectos cumplen con sus reglas de codificación, que se han seguido de forma parcial en este proyecto. Se pueden ver los aspectos más relevantes de las reglas de codificación en el Anexo \ref{app:codificacionSW}.
    \item doxygen: Permite la generación de documentación para código de diferentes lenguajes, c++ en este caso, de forma automática. La herramienta obtiene comentarios del código, escritos con una sintaxis determinada, para documentar los diferentes métodos, objetos y estructura del software.
\end{itemize}

Además es interesante repasar los términos que se incluyen en el Glosario y que aparecerán referenciados a lo largo del documento. \completar

\section{Iconografía del proyecto}

Se han diseñado y utilizado diferentes iconos que pueden servir para orientar el propósito o temática de las diferentes figuras incluidas en el documento así como el capítulo al que pertenecen, manteniendo un color uniforme a lo largo de cada capítulo. Además de tener en la figura \ref{fig:introduccion:iconos} una recopilación de los mismos, a continuación se expone una lista con su significado y capítulo al que se asocia.


\begin{multicols}{2} 
	\begin{flushleft} 
		\subsection*{Capítulo 1: Introducción}
		\inconExplanation{intro_yellow}{Introducción, apertura del proyecto}
	\end{flushleft}	
		
	\begin{flushright} 
		\subsection*{Capítulo 2: Estado del arte}
		\inconExplanation{arte_blue}{Aspectos relacionados con el estado del arte}
	\end{flushright}
\end{multicols} 

\subsection*{Capítulo 3: Punto de inicio del diseño}
\begin{multicols}{2} 
	\begin{flushleft} 
		\inconExplanation{lupa_yellow}{Detalles sobre el robot}
	\end{flushleft}	
	
	\begin{flushright} 
		\inconExplanation{idea_yellow}{Ideas de interés}
	\end{flushright}
\end{multicols} 

\subsection*{Capítulo 4: Mecánica}
\begin{multicols}{2} 
	\begin{flushleft} 
		\inconExplanation{brazo_red}{Aspectos generales de la estructura del brazo robótico}
	\end{flushleft}
	
	\begin{flushright} 
		\inconExplanation{mecanismo_red}{Transmisión mecánica de movimiento}
	\end{flushright}	
\end{multicols} 

\subsection*{Capítulo 5: Electromecánica}
\begin{multicols}{2} 
	\begin{flushleft} 
		\inconExplanation{motor_green}{Actuadores para el brazo robótico}
	\end{flushleft}
	
	\begin{flushright} 
		\inconExplanation{bateria_green}{Alimentación y etapas de potencia}
	\end{flushright}
\end{multicols} 
\begin{multicols}{2}
	\begin{flushleft} 
		\inconExplanation{velocidad_green}{Sensores para el brazo robótico}	
	\end{flushleft}
\end{multicols} 

\subsection*{Capítulo 6: Estudio matemático}
\begin{multicols}{2} 
	\begin{flushleft} 
		\inconExplanation{math_red}{Ecuaciones y relaciones matemáticas}
	\end{flushleft}
	
	\begin{flushright} 
		\inconExplanation{abaco_red}{Resultados calculados}
	\end{flushright}
\end{multicols} 

\begin{multicols}{2} 
	\begin{flushleft} 
		\subsection*{Capítulo 7: Aspectos de Control}
		\inconExplanation{control_yellow}{Gráficas y aspectos del control}
	\end{flushleft}
\end{multicols} 

\subsection*{Capítulo 8: Software}
\begin{multicols}{2} 
	\begin{flushleft} 
		\inconExplanation{llaves_blue}{Diseño y desarrollo del software}
	\end{flushleft}
	
	\begin{flushright} 
		\inconExplanation{debug_blue}{Test, verificación y debug del software desarrollado}
	\end{flushright}	
\end{multicols} 

\begin{multicols}{2} 
	\begin{flushleft} 
		\subsection*{Capítulo 9: Resultados y discusión}
		\inconExplanationOne{megafono_red}
	\end{flushleft}
	\begin{flushright} 
		\subsection*{Capítulo 10: Gestión del proyecto}
		\inconExplanation{agenda_green}{Aspectos relacionados con el estado del arte}
	\end{flushright}
\end{multicols} 

\begin{multicols}{2} 
	\begin{flushleft} 
		\subsection*{Capítulo 11: Conclusiones}
		\inconExplanation{megafono_red}{\completar}
	\end{flushleft}
\end{multicols} 

\begin{figure}[H]
   	\centering
   	\includegraphics[width=\textwidth]{figuras/Icons/iconos.jpg}
   	\caption{Recopilación de los iconos diseñados y utilizados \completarCon{Añadir la diana}}
   	\label{fig:introduccion:iconos}
   	\immagesource{Autor}
\end{figure}

 %partes finales del trabajo: conclusiones, bibliografia y anexos

 \chapter{Estado del arte} \label{chap:estadoarte}
\chapterimage{figuras/ImagenesPortada/PortadaEArte.jpg}
\hrule
\vspace{3mm}

Una vez se han repasado los aspectos generales que se buscan para este proyecto conviene hacer un estudio de la situación actual de modelos comerciales o de investigación con los que se puedan encontrar sinergias 
\\

De esta manera este capítulo hace un repaso de diferentes soluciones destinadas al soporte y posicionamiento de monitores, ordenadores o tablets. Además se hace referencia también a modelos de brazos robóticos específicamente destinados a la asistencia en entornos de usuarios. Dentro de todas las soluciones comerciales se centrará el estudio en las que están específicamente pensadas para su instalación en entornos médicos siempre que sea posible.
\\

Según el apoyo así como los tipos de grados de libertad hay diferentes configuraciones posibles, en este capítulo se verán algunos modelos concretos de cada caso evaluando sus características, ventajas e inconvenientes de los mismos de forma que se pueda generalizar a modelos equivalentes. De igual manera se podrán encontrar modelos motorizados y modelos sin motorizar, clasificación por la que serán agrupados a continuación.
\\

\section{Soportes articulados sin motorizar}

Dentro del grupo de soportes articulados sin motorizar se pueden clasificar según el tipo de anclaje que tienen, ya se enganchen al techo, pared, suelo, etc.
\\

Dentro de cada tipo de anclaje los soportes comerciales disponibles son bastante parecidos por lo que se presentará un modelo concreto que encaje dentro de las dimensiones y capacidad de carga requeridas para la aplicación que se pretende explotar para sacar conclusiones generales sobre cada tipo de soporte.

\subsection{Anclaje a la pared: Cotytech MW-M13P}

Dentro de esta gama (Cotytech MW-M*) se pueden encontrar modelos para soportar diferentes cargas. Concretamente se ha elegido el modelo con menores prestaciones y que soporta menos carga por ser suficiente para la aplicación que se pretende dar. Otros modelos pueden incluir soporte para teclado, caja y cobertura para cables y enganche a pared, entre otros, suponiendo un incremento sobre el precio de este modelo. Obtenida de \cite{CotytechMWM13P:2018} tenemos información relevante que se resume a continuación:

\begin{minipage}{0.35\textwidth}
	\includegraphics[width=\linewidth]{figuras/Imagenes_EstadoArte/Cotytech_MW-M13P.jpg}
\end{minipage}
\begin{minipage}{0.65\textwidth}\raggedright
	\hspace{1cm}
	\begin{itemize}
		\item Tipo de anclaje: Anclaje a la pared.
		\item Tipo de articulación: Articulaciones rotacionales.
		\item Capacidad de carga: entre $1kg$ y $6kg$.
		\item Extensión máxima: $185.7cm$.
		\item Número de grados de libertad: 5.
		\item Ángulos articulaciones: $370^o$ (brazo posición), $270^o$ (muñeca posición), $180^o$ (pared) $^1$. %fake footnote
		\item Ángulos de orientación: Tilt: $20^o$/- $35^o$ (muñeca orientación) y $20^o$/- $60^o$ (brazo orientación).
		\item Peso del soporte: $4.76 kg$.
		\item Precio estándar: 686.98\euro. $^2$. %fake footnote
	\end{itemize}
\end{minipage}

\begin{minipage}{1\textwidth}
	\footnotesize{$^1$En la imagen se pueden ver tres puntos articulados diferenciados, el punto que se fija a la pared con un grado de libertad, el punto central del brazo, que tiene dos grados de libertad (se separarán entre orientación y posición, aunque su efecto no está desacoplado)y la muñeca, que incluye la articulación que se aprecia justo encima de la pantalla como la rotacional sobre la que queda enganchada la misma (con una clasificación análoga al caso intermedio).}
	
    \footnotesize{$^2$Precio a pasado a Euros según el cambio oficial en el día consultado.}
\end{minipage}

 Paralelamente a este modelo la marca Cotytech tiene una versión que, manteniendo el mismo esquema de mecánico, permite un anclaje al techo: el modelo CM-M13 visto en \cite{CotytechCMM13:2018} con un coste de 853.06\euro.

\subsection{Anclaje al techo: Titan Elite T2EQ-C8X5}

 Concretamente se ha tomado el modelo con la montura doblada hacia arriba para un anclaje en el techo. La siguiente información constituye un resumen con los puntos más importantes vistos en \cite{TitanElite:2018}:

 \begin{minipage}{0.35\textwidth}
 	\includegraphics[width=\linewidth]{figuras/Imagenes_EstadoArte/T2EQ.jpg}
 \end{minipage}
 \begin{minipage}{0.65\textwidth}\raggedright
 	\hspace{1cm}
 	\begin{itemize}
 		\item Tipo de anclaje: Anclaje al techo.
 		\item Tipo de articulación: Articulaciones rotacionales.
 		\item Capacidad de carga: hasta $12.7kg$ en diferentes configuraciones.
 		\item Extensión máxima: $106cm$ (la longitud vertical del anclaje al techo variará según se elija al comprar).
 		\item Número de grados de libertad: 5.
 		\item Ángulos articulaciones: $360^o$ para las tres primeras articulaciones que rotan sobre el eje horizontal.
 		\item Ángulos de orientación: Tilt: $50^o$.
 		\item Peso del soporte: $9kg$.
 		\item Precio estándar: 628.30\euro.
 	\end{itemize}
 \end{minipage}
 \\

 Este mismo modelo cuenta con diferentes enganches y longitudes de los tubos que permiten anclarlo al suelo, al techo o a la pared indistintamente.

\subsection{Anclaje a una mesa o superficie de trabajo: Ergotron LX Sit-Stand Desk Arm}

 De entre la gama incluida en los modelos de Ergotron LX se ha elegido aquel que tenía unas dimensiones más ajustadas a las necesidades reales. En general el resto de la gama y otros soportes similares tienen unas dimensiones más reducidas. Resumidas de \cite{LXSitStand:2018} y \cite{LXSitStandWeb:2018} se encuentran a continuación las principales características del modelo:

 \begin{minipage}{0.35\textwidth}
 	\includegraphics[width=\linewidth]{figuras/Imagenes_EstadoArte/LX_Sit-Stand_Desk_Arm.jpg}
 	\\

 	\includegraphics[width=\linewidth]{figuras/Imagenes_EstadoArte/LX_Sit-Stand_Desk_Arm_2.jpg}
 \end{minipage}
 \begin{minipage}{0.65\textwidth}\raggedright
 	\hspace{1cm}
 	\begin{itemize}
 		\item Tipo de anclaje: Anclaje a una mesa, camilla o similar.
 		\item Tipo de articulación: Primera articulación prismática, resto rotacionales.
 		\item Capacidad de carga: hasta $11.3kg$.
 		\item Extensión máxima: se puede variar hasta $36cm$ en altura (articulación prismática, esta es susceptible de ser modificada para ajustarla a otras alturas) y alcanza una extensión de $84cm$
 		\item Número de grados de libertad: 6 (una prismática y cinco rotacionales).
 		\item Ángulos articulaciones: $180^o$ la primera articulación, fija al anclaje; $360^o$ a mitad del brazo y $180^o$ en la muñeca.
 		\item Ángulos de orientación: Tilt: (giro sobre el eje medio horizontal de la pantalla) $75^o$; Pan (eje perpendicular a la pantalla): $360^o$.
 		\item Peso del soporte: $8.9kg$.
 		\item Precio estándar: 247.00\euro – 299.00\euro dependiendo de la tienda y configuración.
 	\end{itemize}
 \end{minipage}
 \\

 \vspace{0.1cm}
 Este mismo modelo cuenta con diferentes enganches y longitudes de los tubos que permiten anclarlo al suelo, al techo o a la pared indistintamente.

 \subsection{Consideraciones generales sobre los soportes no motorizados}
 Los modelos vistos hasta ahora presentan un rango de movimientos muy amplio, están certificados y preparados para su uso en entornos hospitalarios además de estar destinados precisamente al soporte de monitores. En todos los casos la capacidad de carga excede con creces la que se estima necesaria en este proyecto por lo que todas las opciones podrían ser válidas.
 \\

 Aunque cuentan con puntos bastante favorables se trata de productos cerrados sobre los cuales sería complicado integrar actuadores y sensores de manera adecuada, segura y en última instancia, elegante. Además el rango de precios en el que se encuentran es bastante elevado.

\section{Soportes articulados motorizados}

Se centrará este apartado en los modelos que se pueden ver en \cite{maiormover:2018}, como por ejemplo el MaiorFlip 900.

\begin{minipage}{0.35\textwidth}
   \includegraphics[width=\linewidth]{figuras/Imagenes_EstadoArte/not_valid_1.jpg}
\end{minipage}
\begin{minipage}{0.65\textwidth}\raggedright
   \hspace{1cm}
   \begin{itemize}
       \item Tipo de anclaje: Anclaje al techo.
       \item Tipo de articulación: Primera articulación rotacional, segunda prismática y tercera rotacional.
       \item Capacidad de carga: máximo de $28kg$.
       \item Extensión máxima: Descenso de hasta $68cm$
       \item Número de grados de libertad: 3 (una prismática y dos rotacionales).
       \item Ángulos articulaciones: Primera rotación giro de hasta $90^o$; articulación prismática con un alcance de $68cm$ con una capacidad de giro de la articulación final de $360^o$.
       \item Peso del soporte: $35kg$.
       \item Precio estándar: Bajo demanda.
   \end{itemize}
\end{minipage}
\\

 \vspace{0.1cm}
Otros modelos que se pueden encontrar presentan unas características similares:

\begin{minipage}{0.5\textwidth}
   \includegraphics[width=0.8\linewidth]{figuras/Imagenes_EstadoArte/not_valid_2.jpg}
\end{minipage}
\begin{minipage}{0.5\textwidth}
   \includegraphics[width=0.8\linewidth]{figuras/Imagenes_EstadoArte/not_valid_3.jpg}
\end{minipage}
 \subsection{Consideraciones generales sobre los soportes motorizados}

    En esta línea podemos encontrar también los diseños de \cite{Chung2009} o \cite{ChungL.Chang2008}. Se puede ver también el modelo \cite{Chung2009} dotado de articulaciones paralelas que permiten controlar la orientación del dispositivo.
    \\
    
    Este tipo de soportes están principalmente pensados para motorizar televisores de gran tamaño con unos rangos de movimiento bastante limitados. Comúnmente presentan grados de libertad para modificar la orientación de la pantalla y en pocos casos un grado de libertad extra para desplazar, sobre algún eje el dispositivo.  No son aptos para la aplicación que se pretende dar puesto que no permiten su posicionamiento a una distancia adecuada del usuario.

\section{Brazos robóticos para asistencia de pacientes}

	En general para el propósito que se plantea en este proyecto se podría adaptar una solución robótica comercial implementando una interfaz entre la tablet y el controlador del brazo. En esta sección se presentan algunos modelos de brazos robóticos especialmente pensados como robots asistenciales, preparados para una interacción directa con pacientes en entornos hospitalarios o en casa.

 \subsection{JACO 3 fingers, Kinova robotics}
	 Pertenece a la línea de productos de Kinova robotics especialmente diseñados como robots asistenciales. Están pensados para una interacción directa con el paciente o usuario de forma que pueda convertirse en una extensión del mismo proporcionándole una mayor independencia. Entre las características descritas en \cite{Jaco:2018} y \cite{JacoWeb:2018} podemos recoger las siguientes más relevantes:
     \\

	  \begin{minipage}{0.35\textwidth}
	  	\includegraphics[width=\linewidth]{figuras/Imagenes_EstadoArte/jaco-3-finger.jpg}
	  \end{minipage}
	  \begin{minipage}{0.65\textwidth}\raggedright
	  	\hspace{1cm}
	  	\begin{itemize}
	  		\item Tipo de anclaje: Adaptativo a una mesa, silla de ruedas, etc
	  		\item Tipo de articulación: 6 grados de libertad rotacionales
	  		\item Capacidad de carga: entre $1.8kg$ y $1.6kg$ dependiendo de la versión elegida (con tres dedos tiene una capacidad menor).
	  		\item Extensión máxima: alcanza $90cm$.
	  		\item Número de grados de libertad: 6.
	  		\item Ángulos de posición: Rotación continua en las articulaciones.
	  		\item Ángulos de orientación: $55^o$ en la muñeca.
	  		\item Peso del brazo: $5.7kg$.
	  		\item Precio estándar: desde 34900\euro.
	  	\end{itemize}
	  \end{minipage}
	  \\

	  \vspace{0.1cm}

      Este modelo concreto, Jaco, viene en dos formatos pudiendo tener dos o tres dedos en el manipulador de su extremo.
      \\

	  De esta marca se puede adquirir también el modelo MICO con algo menos de alcance ($70cm$) como se ve en \cite{Mico:2018}. Los grados de libertad ofertados varían entre 4 y 7, se ha optado por tomar una solución lo más parecida a la requerida.

 \subsection{Brazo Multi-manipulador de la Universidad de Pamplona}

    Dejando de lado el ámbito puramente comercial se encuentran proyectos que es interesante repasar. En \cite{Marquez:2013} se describe un modelo diseñado específicamente como robot asistencial con capacidad de cambiar, de forma autónoma, entre diferentes manipuladores disponibles como pueden ser una cuchara, un tenedor, un cuenco, entre otros.
    \\

     \begin{minipage}{0.35\textwidth}
       \includegraphics[width=\linewidth]{figuras/Imagenes_EstadoArte/UPamplona.png}
     \end{minipage}
     \begin{minipage}{0.65\textwidth}\raggedright
       \hspace{1cm}
       \begin{itemize}
           \item Tipo de anclaje: Lleva su propia plataforma sobre la que se monta
           \item Tipo de articulación: 4 grados de libertad rotacionales
           \item Capacidad de carga: manipuladores que adjunta.
           \item Extensión máxima: alcanza Superficie de trabajo de 40cm x 40cm.
           \item Número de grados de libertad: 4.
           \item Actuadores: Dynamixel AX-12
           \item Precio estándar: proyecto no comercial.
       \end{itemize}
     \end{minipage}
     \\

     \vspace{0.1cm}

     Es interesante destacar que aunque no se aportan demasiados datos sobre el modelo (peso del modelo, capacidad de giro de sus articulaciones, etc) si presenta un estudio completo de su aceptación así como facilidad de uso. Un sistema intuitivo es más fácilmente aceptado por los usuarios, a los cuales les será más fácil empezar a hacer uso de las facilidades que ofrece un robot de este tipo.
     \\

     Los actuadores que utiliza son de la gama de Dynamixel, una serie de servo motores muy versátiles aunque sin demasiada carga útil una vez montados sobre las articulaciones.

 \subsection{Brazo robótico para personas con movilidad reducida}

    Continuando con modelos de un ámbito más académico se encuentra el modelo descrito en \cite{Hideyuki:2010}. Se trata de un modelo plegable que puede ser almacenado o transportado dentro de una maleta de forma sencilla. Viene con un manipulador en forma de manopla que permite agarres de una amplia gama de objetos. También pensado para la asistencia de personas con movilidad reducida ha sido principalmente testeado para administrar alimentos y bebidas.
    \\

     \begin{minipage}{0.35\textwidth}
       \includegraphics[width=\linewidth]{figuras/Imagenes_EstadoArte/Tokyo.png}
     \end{minipage}
     \begin{minipage}{0.65\textwidth}\raggedright
       \hspace{1cm}
       \begin{itemize}
           \item Tipo de anclaje: Sin determinar.
           \item Tipo de articulación: 7 grados de libertad rotacionales
   	  		\item Extensión máxima: alcanza $71cm$.
   	  		\item Número de grados de libertad: 7.
   	  		\item Peso del brazo: $5kg$ con dos baterías.
   	  		\item Precio estándar: proyecto no comercial.
       \end{itemize}
     \end{minipage}
     \\

     \vspace{0.1cm}
 \subsection{Consideraciones generales sobre los brazos robóticos}
 
	Además de los mencionados, en esta misma línea podemos encontrar otros modelos como \cite{Arai2011}, para ayuda a la alimentación, en \cite{Chang2003} diseñado específicamente para personas con movilidad reducida o \cite{Ali2007} presentando un prototipo asistencial genérico.
	\\
	
	Dentro de este formato conviene repasar el estudio hecho por \cite{Groothuis2013} en el cual se comparan diferentes modelos de brazos robóticos asistenciales en parámetros como movilidad, capacidad de carga y también seguridad.
	\\
	
    Aunque entre los modelos descritos sería fácil encontrar uno válido para la aplicación que se pretende dar, la mayoría de modelos comerciales (en los casos académicos se desconoce el precio que rondaría en caso de convertirse en producto comercial) tienen unos precios elevados que dificultarían su aplicación a gran escala.
    \\



 \chapter{Diseño del brazo robótico: Punto de inicio} \label{chap:Punto_partida}

 \chapter{Mecánica y soporte físico del proyecto} \label{chap:Mecanica}
\hrule
\vspace{3mm}

En este capítulo se describen los aspectos mecánicos y se valoran diferentes cuestiones de diseño del brazo robótico. Se dará una visión general comparando diferentes propuestas y justificando la elección definitiva para continuar describiendo cada una de las articulaciones así como el posicionamiento de sensores y actuadores así como de la transmisión de movimiento desde/a los mismos.
 
\section{Visión general} \label{sec:Mecanica:vision_general}

\section{Articulación uno. Giro en el eje Z} \label{sec:Mecanica:articulacion_uno}
    Junto con las articulaciones dos y tres, descritas en la Sección \ref{sec:Mecanica:articulacion_dostres} están consideradas como los grados de libertad que gestionan la posición del extremo del robot en un espacio tridimensional. En adelante se las podrá denominar también "grados de libertad de posición".
    \\ 
    
    Esta articulación está actuada por un \ingles{Servo G15 Cube} (descrito en la Sección \ref{sec:Introduccion:materiales_software}. El movimiento de dicho servo se transmite a la articulación a través de un juego de ruedas que, solidarias a la parte superior (parte móvil) de la articulación y por rozamiento, transmiten el movimiento hasta la pista inferior (parte fija a la base del robot).
    \\ 
    
    De esta forma aseguramos que el usuario, en cualquier momento, podrá desplazar el robot superando el rozamiento de esta cadena de transmisión anulando, en caso de estar en proceso, el movimiento que pueda estar efectuando el \glosario{servo}.
	\\ 
	En la figura \ref{fig:Mecanica:giro_z} se puede ver en detalle el montaje de dicha estructura. Las piezas de la imagen se encuentran, con la misma referencia, en el Anexo \ref{app:listadoPiezas}.
	
	\begin{figure}[H]
		\centering
		\includegraphics[width=0.5\textwidth]{figuras/Imagenes_Mecanica/RuedasGiroZ.png}   
		\caption{Montaje de la transmisión del movimiento del servo a la articulación encargada de girar en Z}
		\label{fig:Mecanica:giro_z}
	\end{figure}
	
	El apoyo del peso en la última versión se hace sobre un rodamiento \completarCon{coaxial? o como se llamaba?} para conseguir un apoyo completo. Previamente se contempló la posibilidad de utilizar ruedas sobre un carril, de forma que se mantenía una estructura similar a la rueda motriz. En este caso tras probar ambas opciones se optó por colocar la rueda motriz en el lado sobre el que cae la carga del brazo al extenderse para maximizar el apoyo. Este aspecto se ha mantenido al integrar el rodamiento, que asegura un mayor apoyo que las ruedas aun reduciendo el diámetro de dicho apoyo (dificultando la distribución de cargas).
	
	\completarCon{Imagen de las ruedas con las distintas configuraciones de carga y rueda motriz}
	
	\completarCon{Imagen de como afecta el cambio de diámetro del apoyo}
	
\section{Articulaciones dos y tres. Posicionamiento en el plano sagital} \label{sec:Mecanica:articulacion_dostres}
    Estas dos articulaciones son las encargadas de posicionar el extremo en el plano sagital del robot.
    \\ 
    
    \begin{figure}[H]
       	\centering
       	\includegraphics[width=0.9\textwidth]{figuras/Imagenes_Mecanica/mecanismos_4_barras_triangulo.png}   
       	\caption{Esquema de la cadena cinemática correspondiente a los \completarCon{Glosario a GDL} dos y tres. Idea preliminar}
       	\label{fig:Mecanica:4_bar_mecanism_triangle}
    \end{figure}
    
    Están formadas por dos mecanismos de cuatro barras acoplados en serie. Tienen la particularidad de que las barras son iguales dos a dos, de forma que las barras se mantienen siempre en paralelo. 
    \\ 
    
    \begin{figure}[H]
    	\centering
    	\includegraphics[width=0.9\textwidth]{figuras/Imagenes_Mecanica/mecanismos_4_barras.png}   
    	\caption{Esquema de la cadena cinemática correspondiente a los \completarCon{Glosario a GDL} dos y tres}
    	\label{fig:Mecanica:4_bar_mecanism}
    \end{figure}
    
    Como se puede ver en la figura \ref{fig:Mecanica:4_bar_mecanism} y \ref{fig:Mecanica:4_bar_mecanism_triangle} la actuación se realiza sobre cada articulación en los puntos marcados. El movimiento se transmite desde los servos ubicados en la base de la articulación 1 hasta las mismas a través de un hilo de \ingles{kevlar}. Un sistema de poleas amplifica y redirige el par hasta los mismos. En la figura \ref{fig:Mecanica:transmision_poleas_cuerda} se puede ver como se redirecciona el hilo desde la base en dirección hacia las articulaciones superiores.
    
   	\begin{figure}[H]
   		\centering
   		\includegraphics[width=0.6\textwidth]{figuras/Imagenes_Mecanica/TransmisionMotorArticulacion.png}   
   		\caption{Montaje de la transmisión del movimiento de los servos a las articulaciones 2 y tres.}
   		\label{fig:Mecanica:transmision_poleas_cuerda}
   	\end{figure}
   	
   	En las figuras \ref{fig:Mecanica:4_bar_mecanism} y \ref{fig:Mecanica:4_bar_mecanism_triangle} se muestran dos posibles configuraciones mecánicas que se han probado. En la primera, ubicada al principio de la sección (figura \ref{fig:Mecanica:4_bar_mecanism_triangle}) se cuenta con la ventaja de actuar ambas articulaciones sobre el mismo punto de actuación, facilitando la realimentación así como el paso de cables (tanto eléctricos como de transmisión mecánica). Como se puede ver la orientación del extremo depende de los ángulos del triángulo ubicado en el equivalente a la tercera articulación, donde se redirecciona el giro. En este caso las barras a partir de dicho triángulo se encuentran soldadas para hacer la actuación sobre el punto ya mencionado. Otra configuración posible, mostrada en la figura \ref{fig:Mecanica:4_bar_mecanism}, aprovecha esta propiedad de mantener la orientación del extremo a lo largo de la cadena cinemática haciendo que, desde el inicio todas estas conexiones se hagan de forma perpendicular al plano del suelo. De esta forma se puede asegurar que el extremo de dicha cadena mecánica siempre se mantendrá perpendicular al suelo consiguiendo un desacople absoluto de los grados de posición del robot de los grados de libertado que controlan la orientación. En este caso, las barras del extremo no se encuentran soldadas como en el primer caso por lo que el control de la tercera articulación no se puede hacer sobre el mismo punto, como se hacía antes si no que se lleva a la unión de ambos mecanismos. Se ha dado finalmente más peso a la característica de mantener la orientación por facilitar mucho el desarrollo matemático y de control futuros.

\section{Posicionamiento de sensores y actuadores} \label{sec:Mecanica:sensores_actuadore}

	Como e ha anticipado en secciones anteriores los tres motores correspondientes al giro en el eje Z así como el posicionamiento en el plano sagital están ubicados sobre la primera articulación uno encima de otro formando una torre. Desde ahí se dirige a través de hilos el par motor del primer y tercer servo hasta las articulaciones que dos y tres y a través de fricción por ruedas hasta el giro en el eje Z. En la figura \completar se puede ver en detalle dicho montaje así como la fijación a la base de giro.
	\\ 
	
	\completarCon{Imagen en detalle de la torre de motores}
	
	Las articulaciones realimentadas son la articulación dos y la articulación tres, que llevan una realimentación de posición articular a través de un potenciómetro en cada una. El primer grado de libertad, el encargado del giro en el eje Z no está realimentado externamente. En este caso se cuenta con la información proporcionada por el servo y finalmente, la proporcionada por la tablet en el extremo.
	\\
	
	En el caso de la segunda articulación el potenciómetro se conecta con la propia articulación a través de un juego de engranajes con una relación de \completar que maximiza el uso del potenciómetro, ajustando el rango de movimiento del mismo (\completarCon{cuantos grados gira?}) con el ángulo de giro de la articulación (\completarCon{cuantos grados gira?}). En el caso de la tercera articulación el eje del potenciómetro se encuentra en la misma línea que el eje de giro que se pretende realimentar, el giro es solidario a dicha barra por lo que la relación de giro es unitaria. En la figura \completar se pueden ver ambos montajes: a la derecha la articulación dos con el juego de engranajes; a la izquierda la articulación tres con la transmisión solidaria.
	
	\completarCon{Añadir fotos de como se enganchan ambos potenciómetros}

\section{Estudio de la cadena cinemática completa} \label{sec:Mecanica:cadena_cinematica}

\completarCon{Hablar de como se reduce la carga hasta los servos: columpios, poleas dobles, palancas, etc}

 \chapter{Electrónica involucrada en el proyecto} \label{chap:Electronica}
\hrule
\vspace{3mm}
En este capítulo se hará una descripción detallada de todos los componentes involucrados en el proyecto.

\section{Actuadores} \label{sec:Electronica:Actuadores}
\label{sec:Electronica:Actuadores:G15}

    Para los grados de libertad de posición del prototipo se ha optado por utilizar los \glosarioPlural{smartservo} G15 Cube de la marca Cytron.
    
    Algunas características importantes de los mismos se muestran en la tabla \ref{tab:g15_catact}:
    
    \begin{table}[H]
    	\caption{Características mas importantes de los Servos G15 de Cytron. Tabla traducida y resumida a los puntos más importantes del Cytron G15 Cube servo User Manual \cite{CytronTechnologies2012} \completarCon{¿esto está bien, hay que poner paginas involucradas?}}
    	\label{tab:g15_catact}
    	\begin{center}
    		\begin{minipage}{\textwidth}
    			\begin{tabular}{ |c|c|c|c| }
    				\hline
    				\multicolumn{4}{|c|}{\textbf{Características eléctricas}} \\ 
    				\hline
    				\textbf{Parámetro} & \textbf{Valor Mínimo} & \textbf{Valor Típico} & \textbf{Valor Máximo} \\
    				\hline
					Voltaje & $6.5V$ & $12V$ & $17.8V$ \\
    				\hline
    				Consumo de corriente ($12V$) & & & $1.5A$ \\
    				\hline
    				Temperatura de funcionamiento & $0^oC$ & & $80^oC$ \\
    				\hline
    				Par capaz de soportar & & & $15km \cdot cm$ \\
    				\hline
    				\multicolumn{4}{c}{\textbf{}} \\ 
    				\hline
    				\multicolumn{4}{|c|}{\textbf{Especificaciones técnicas}} \\ 
    				\hline
    				\multicolumn{2}{|c|}{Peso} & \multicolumn{2}{|c|}{$63g$}\\ 
    				\hline
    				\multicolumn{2}{|c|}{Par capaz de realizar (a $12V$)} & \multicolumn{2}{|c|}{$12kg \cdot cm$}\\ 
    				\hline
    				\multicolumn{2}{|c|}{Margen angular de operación } & \multicolumn{2}{|c|}{$360^o$ en giro continuo}\\ 
    				\hline
    				\multicolumn{2}{|c|}{Máxima velocidad (en vacío a $12V$)} & \multicolumn{2}{|c|}{$63 RPM$ }\\ 
    				\hline    			
    				\multicolumn{2}{|c|}{Comunicaciçon} & \multicolumn{2}{|c|}{Half duplex asynchronous serial communication ($7812.5bps-500kbps $)}\\ 
    				\hline    				 	 
    			\end{tabular}
    		\end{minipage}
    	\end{center}
    \end{table}
    
	Estos servos utilizan un protocolo de comunicación basado en una comunicación \ingles{Half Duplex Serial}. En este caso toda la información fluye por un mismo cable. Los servos se conectan en un bus uno a continuación del otro, teniendo tres pines, uno de voltaje positivo, otro de GND y el tercero de datos. Se puede ver en la figura \ref{fig:Electronica:bus-servos} como quedan conectados a la placa, quedando uno de los extremos (el del último servo) libre.
    \begin{figure}[H]
    	\centering
    	\includegraphics[width=0.9\textwidth]{figuras/Imagenes_Electronica/G15_bus_conection.png}   
    	\caption{Esquema de la conexión el bus serie de servos y la placa Shield}
    	\label{fig:Electronica:bus-servos}
    \end{figure}

\section{Etapa de potencia} \label{sec:Electronica:Potencia}

	Para la correcta comunicación entre los servos y la placa se utilizan tres pines de la misma. Dos serán los de entrada (RX) y salida (TX) para comunicar y el tercer pin funcionará a modo de pin de control, gestionando en que momentos se publica información por el pin de salida y en que momento se escucha por el pin de entrada.
	\\
	
	Dependiendo de que placa se utilice este puerto serie (el TX y el RX) podrán estar conectados a un puerto serie de tipo Hardware o emular uno por Software. 
	\\
	
	En primeras versiones del desarrollo se ha estado utilizando un Arduino Uno en el cual se emulaba, por los pines 9 (TX) y 8 (RX) un puerto software quedando el puerto Hardware de los pines 1 y 0 para la comunicación serie con el ordenador.
	\\
	
	Debido a los ciclos de funcionamiento del software para aplicar el control al brazo robótico esta comunicación resulta ser demasiado lenta, es por ello que se sustituye el Arduino Uno por un Arduino Mega, con 4 puertos serie Hardware que se podrán aprovechar, para la comunicación con el ordenador (pines 0 y 1) y para la comunicación con los servos (pines 18 y 19).
	\\ 
	
	En la Shield utilizada para comunicar con los servos viene preparado para, mediante unos jumpers \completarCon{definir?¿}, poder seleccionar unos pines u otros. En este caso no está preparado para comunicar con un Arduino Mega directamente, es por ello que se han sacado unos cables para conectar, la parte de la shield que conecta con el puerto de los servos (están los 4 pines cortocircuitados) con los pines de los puertos hardware del Arduino Mega. Se puede ver dicha conexión en la figura \ref{fig:Electronica:shield-arduino}.
	
    \begin{figure}[H]
    	\centering
    	\includegraphics[width=0.65\textwidth]{figuras/Imagenes_Electronica/Shield-Arduino-Conection.png}   
    	\caption{Esquema de la conexión entre la placa Shield y Arduino para utilizar los puertos Hardware Serie de la Arduino Mega}
    	\label{fig:Electronica:shield-arduino}
    \end{figure}
\section{Procesamiento}

\section{Sensores} \label{sec:Electronica:Sensores}





 \chapter{Estudio Cinemático y Matemático} \label{chap:Cinematica}
\chapterimage{figuras/ImagenesPortada/PortadaCinematica.jpg}
\hrule
\vspace{3mm}

Este capítulo describe y desarrolla las relaciones matemáticas que definen los aspectos del brazo robótico tal y como se ha descrito, principalmente las relacionadas con la cinemática del robot.

\section{Convenio de ángulos y referencias}

	Antes de comenzar con las ecuaciones matemáticas se deben establecer unas referencias básicas. En la figura \ref{fig:Control:cinematica_3} se pueden ver representadas las que se utilizarán en apartados posteriores.
	
	\begin{itemize}
		\item \textbf{q1, q2 y q3}: ángulo de giro de las articulaciones uno, dos y tres respectivamente. La flecha marca el sentido creciente de los ángulos estando el cero situado en: eje X para el caso de q1 y en el eje vertical para el caso de q2 y q3.
		\item \textbf{Dimensiones del brazo}: siendo conocidas L1, L2, L3, LA, LB.
		\item \textbf{Punto $(x_r,y_r,z_r)$}: extremo del brazo respecto al cual se obtiene la cinemática.
		\item \textbf{Ángulos auxiliares}: q2' y q3' que tendrán un papel importante en cálculos posteriores.
	\end{itemize}
	
	Tal y como se ha explicado en el capítulo \ref{chap:Mecanica} se contempla una variación en el apoyo y/o base del brazo robótico en función del lugar donde se instale. La cinemática está referenciada respecto de la articulación $A_1$, utilizando el sistema de referencia representado en la figura \ref{fig:Control:cinematica_3}.
	
    \begin{figure}[H]
    	\centering
    	\includegraphics[width=1\textwidth]{figuras/Imagenes_cinematica/cinematica_3.jpg}
    	\caption{Relación con los ángulos y medidas de interés}
    	\label{fig:Control:cinematica_3}
    	\immagesource{Autor}
    \end{figure}
    
    Además de los parámetros descritos el desarrollo de la cinemática utiliza una serie de medidas auxiliares que se pueden ver en la figura \ref{app:codificacionSW} donde es importante destacar la definición del punto $(x_p,y_p,z_p)$.

    \begin{figure}[H]
    	\centering
    	\includegraphics[width=1\textwidth]{figuras/Imagenes_cinematica/cinematica_1.jpg}
    	\caption{Representación de los ángulos y distancias auxiliares más representativas. GDL 2 y 3}
    	\label{fig:Control:cinematica_1}
    	\immagesource{Autor}
    \end{figure}
    
    Las coordenadas \textit{X} y \textit{Z} del punto \textit{p} se obtienen de desplazar el punto \textit{r} simplificar los cálculos angulares. La coordenada en el eje Y coincide para ambos puntos (ver figura \ref{fig:Control:cinematica_2}).
    \begin{figure}[H]
    	\centering
    	\includegraphics[width=1\textwidth]{figuras/Imagenes_cinematica/cinematica_2.jpg}
    	\caption{Representación de los ángulos y distancias más representativas. Primer grado de libertad}
    	\label{fig:Control:cinematica_2}
    	\immagesource{Autor}
    \end{figure}
\section{Cinemática del robot}

    La solución del problema cinemático del brazo tal y como se ha diseñado es bastante sencilla de obtener mediante métodos geométricos. Aunque se presentan los parámetros de Denavit-Hartenberg y el cálculo definitivo y control del brazo utilizan las relaciones obtenidas mediante relaciones geométricas.
    \\
    
    Como se ha descrito en el capítulo \ref{chap:Mecanica} la orientación de la barra acopladora así como de la barra del extremo se mantienen siempre perpendiculares al plano del suelo; esto permite referenciar los ángulos de las articulaciones dos y tres (d2 y q3 en la figura \ref{fig:Control:cinematica_3}) respecto del eje vertical, de forma que son independientes de la posición angular del resto de articulaciones. Como se verá a continuación esta característica facilita los cálculos matemáticos.
    \\
    
    Las funciones trigonométricas seno, coseno y tangente no son unívocas para giros completos de $360^o$, la cinemática calculada tiene unos rangos de aplicación definidos para cada articulación en base al movimiento que físicamente son capaces de realizar. Se pueden ver estos intervalos en la tabla \ref{tab:rangos_art}.
    
    \begin{table}[H]
    	\caption{Rango de movimiento articular.}
    	\immagesource{Autor}
    	\label{tab:rangos_art}
    		\begin{center}
    			\begin{tabular}{ |c|c|c| }
    				\hline
    				\textbf{Articulación} & \textbf{Extremo inferior} & \textbf{Extremo superior}  \\
    				\hline
    				q1 & $0^o$ & $180^o$ \\
    				\hline
    				q2 & $90^o$ & $180^o$ \\
    				\hline
    				q3 & $0^o$ & $90^o$ \\
    				\hline
    			\end{tabular}
    		\end{center}
    \end{table}

	Las medidas del brazo robótico, presentadas inicialmente en la figura \ref{fig:Control:cinematica_3} pueden verse en la tabla \ref{tab:medidas} , los cálculos de la cinemática mantendrán estas variables de manera simbólica.
	
 \begin{table}[H]
    	\caption{Medidas estructurales}
    	\immagesource{Autor}
    	\label{tab:medidas}
    	\begin{center}
    		\begin{tabular}{ |c|c| }
    			\hline
    			L1 & 0.3270 m \\
    			\hline
    			L2 & 0.455 m \\
    			\hline
    			L3 & 0.455 m \\
    			\hline
    			LA & 0.030 m \\
    			\hline
    			LB & 0.042 m \\
    			\hline
    		\end{tabular}
    	\end{center}
 \end{table}
	    
\subsection{Parámetros de Denavit-Hartenberg}
	Se ha desarrollado siguiendo los pasos descritos por \cite{barrientos}. En la figura \ref{fig:Control:dh} puede verse la posición de los sistemas de referencia definidos. En la tabla \ref{tab:dh} se muestran los parámetros de Denavit-Hartenberg en base a dichos sistemas de referencia. Se calcula respecto al punto \textit{p}, cuya relación posterior con el extremo \textit{r} es directa.
		
	\begin{figure}[H]
		\centering
		\includegraphics[width=1\textwidth]{figuras/Imagenes_cinematica/denavit-hartemberg_virtuales.jpg}
		\caption{Sistemas de referencia}
		\label{fig:Control:dh}
		\immagesource{Autor}
	\end{figure}
	
	Nótese que los ángulos $q_2$ y $q_3$ a los que se hace referencia en este apartado, los ángulos de los parámetros de Denavit-Hartenberg, son en realidad $q_2'$ y $q_3'$ presentados anteriormente.
	
	 \begin{table}[H]
	 	\caption{Medidas estructurales}
	 	\immagesource{Autor}
	 	\label{tab:dh}
	 	\begin{center}
	 		\begin{tabular}{ |c|c|c|c|c| }
	 			\hline
	 			Articulación & $\Theta$ (grados) & d & a & $\alpha$ (grados)\\
	 			\hline
	 			1 & $q_1$ & L1 & 0 & 90 \\
	 			\hline
	 			2 & $q_2$ & 0 & L2 & 0 \\
	 			\hline
	 			3 (virtual) & $\left(180+q_2-\arctan{\frac{LB}{LA}}\right)$ & 0 & $\sqrt{{LA}^2+{LB}^2}$ & 0 \\
	 			\hline
	 			4 & $\left(\arctan{\frac{LB}{LA}}+q3\right)$ & 0 & L3 & 0 \\
	 			\hline
	 			5 (virtual) & $-q_3$ & 0 & 0 & 0 \\
	 			\hline
	 		\end{tabular}
	 	\end{center}
	 \end{table}
	 
	 Aunque no se muestra el desarrollo completo matricial para resolver la cinemática directa se ha desarrollado una serie de funciones en matlab para resolverla. Se presentan a continuación en el fragmento de código 6.1.
	 
	 \lstset{language=matlab, breaklines=true, basicstyle=\footnotesize}
	     \begin{lstlisting}[frame=single, caption=Cálculos con DH en matlab, label=code:dhmatlab]
function [dh] = denavith_Aij(theta,d_i,a_i,alfa_i)
	syms theta_i %d_i a_i alfa_i	
	theta_i=theta;	
	dh = [cos(theta_i),-cosd(alfa_i)*sin(theta_i),sind(alfa_i)*sin(theta_i),a_i*cos(theta_i);sin(theta_i),cosd(alfa_i)*cos(theta_i),-sind(alfa_i)*cos(theta_i),a_i*sin(theta_i);0,sind(alfa_i),cosd(alfa_i),d_i;0,0,0,1];	
end

%%%%%%%%%%%%%%%%%%%%%%%%%%%%%%%

function Dirkinematics	
	syms th1 th2 th3 L1 L2 L3	
	
	% Parametros Denavit-Hartenberg del robot	
	teta = [th1 90+th2 90+th3 0];	
	d = [0 L1 0 L3];	
	a = [0 0 L2 0];	
	alfa = [0 90 0 0];	
	
	% Matrices de transformacion homogenea entre sistemas de coordenadas consecutivos	
	A01 = denavith_Aij( teta(1),d(1),a(1),alfa(1) );	
	A12 = denavith_Aij( teta(2),d(2),a(2),alfa(2) );	
	A23 = denavith_Aij( teta(3),d(3),a(3),alfa(3) );	
	A34 = denavith_Aij( teta(4),d(4),a(4),alfa(4) );	
	
	% Matriz de transformacion del primer al ultimo sistema de coordenadas	
	T04 = A01*A12*A23*A34;
		
	%Coordenadas posicion ejes cartesianos	
	PX = T04(1,4);	
	PY = T04(2,4);	
	PZ = T04(3,4);	
end
	     \end{lstlisting}
	
\subsection{Cinemática directa}
	Siendo conocidos las posiciones articulares de los tres grados de libertad se puede obtener la posición en coordenadas cartesianas del punto \textit{r} definido respecto al punto A1. Se definen primero algunos puntos intermedios sobre los que se apoya el cálculo:
	\\
	
	Ángulos auxiliares q2' y q3':
	 \begin{equation}
	 \label{cd:eq1}
	 q2' = q2 - \frac{\varPi}{2}
	 \end{equation}
	 \begin{equation}
	 \label{cd:eq2}
	 q3' = \frac{\varPi}{2} - q3 
	 \end{equation}
	 Conviene también recordar de la tabla \ref{tab:medidas} la relación siguiente:
	 \begin{equation}
	 \label{cd:eq3}
	 L1 = L2
	 \end{equation}
	
	Apoyándose en las relaciones de \ref{cd:eq1}, \ref{cd:eq2} y \ref{cd:eq3} se obtiene la cinemática directa del brazo robótico:
	    
    \begin{equation}
    x_r = x_p+LA = \cos(q1)\left(LA + L2 \cdot \left(\cos(q2') + \cos(q_3')\right)\right)
    \end{equation}
    
    \begin{equation}
    y_r = y_p = \sin(q1)\left(LA + L2 \cdot \left(\cos(q2') + \cos(q_3')\right)\right)
    \end{equation}
    
    \begin{equation}
    z_r = z_p+LB+L1 = LB + L1 + L2\cdot\left( \sin(q2') - \sin(q3') \right)
    \end{equation}
\subsection{Cinemática inversa}

	Siendo conocida la posición del extremo (punto \textit{r}) en coordenadas cartesianas se calcula la posición angular de cada articulación. Se definen algunas relaciones sobre las que se apoyará el cálculo matemático.
	\\
	
	Se definen las coordenadas X y Z del punto \textit{p} de la siguiente manera: 
	\begin{equation}
	\label{ci:eq0}
	x_p = \sqrt{x_r^2+y_r^2} - LA
	\end{equation}
	\begin{equation}
	\label{ci:eq1}
	y_p = y_r
	\end{equation}
	\begin{equation}
	\label{ci:eq2}
	z_p = z_r - LB - L1
	\end{equation}
	
	La diagonal r representada en la figura \ref{fig:Control:cinematica_1}:
	\begin{equation}
	\label{ci:eq31}
	r = \sqrt{x_p^2+y_p^2}
	\end{equation}
	
	
	Además se utilizan los ángulos auxiliares descritos a continuación:
	\begin{equation}
	\label{ci:eq3}
	\alpha = \arctan\left(\frac{z_p}{x_p}\right)
	\end{equation}
	
	\begin{equation}
	\label{ci:eq4}
	\beta = \arccos\left(\frac{r^2+L2^2-L3^2}{2 \cdot L2 \cdot r}\right) = \arccos\left(\frac{r}{2 \cdot L2}\right)
	\end{equation}
	
	\begin{equation}
	\label{ci:eq5}
	\varphi = \arccos\left(\frac{L2^2+L3^2-r^2}{2 \cdot L2 \cdot L3}\right) = \arccos\left(\frac{r}{2L2}\right)
	\end{equation}
	
	\begin{equation}
	\label{ci:eq6}
		q2' = \alpha + \beta
	\end{equation}
	
	\begin{equation}
	\label{ci:eq7}
		q3' = \varPi - q2' - \varphi
	\end{equation}
	
	Finalmente, apoyándose sobre las relaciones descritas (y recordando las ecuaciones \ref{cd:eq1} y \ref{cd:eq2}) se obtiene las relaciones cinemáticas para obtener las posiciones articulares conociendo el punto del extremo: 
	 \begin{equation}
	 \label{ci:eq8}
	 q1 = \arctan\left(\frac{y_p}{x_p}\right)
	 \end{equation}
	 
	 \begin{equation}
	 \label{ci:eq9}
	 q2 = \frac{\varPi}{2} - q2'
	 \end{equation}
	 
	 \begin{equation}
	 \label{ci:eq10}
	 q3 = q3'- \frac{\varPi}{2}
	 \end{equation}
\subsection{Cinemática diferencial}
	 Por derivación de las relaciones de la cinemática directa se puede obtener la relación entre las velocidades articulares y la velocidad del extremo, el punto \textit{r}. De forma matricial se puede expresar esta relación de la siguiente manera:
	\begin{equation}
		\label{cdif:eq0}
		\left[ \begin{array}{c}
			\dot x_r \\
			\dot y_r \\
			\dot z_r \\
		\end{array}\right]
		=
		\left[ \begin{array}{ccc}
		N_{11} & N_{12} & N_{13} \\
		N_{21} & N_{22} & N_{23} \\
		N_{31} & N_{32} & N_{33} \\
		\end{array}\right]
		\cdot
		\left[ \begin{array}{c}
		\dot q_1 \\
		\dot q_2 \\
		\dot q_3 \\
		\end{array}\right]
	\end{equation}
	Teniendo los valores $N_{xx}$ la siguiente forma:
	 \begin{equation}
	 \label{cdif:eq1}
	 N_{11} = \frac{\partial x_r}{\partial q_1}
	 \end{equation}
 	 \begin{equation}
 	 \label{cdif:eq2}
 	 N_{33} = \frac{\partial z_r}{\partial q_3}
 	 \end{equation}
 	 
	Se ha desarrollado un \textit{script} en matlab para resolver esta serie de cálculos de manera simbólica, puede verse en el fragmento de código 6.2.
	
		 \lstset{language=matlab, breaklines=true, basicstyle=\footnotesize}
		 \begin{lstlisting}[frame=single, caption=Cálculo cinemática diferencial, label=code:diferencialmatlab]
	%Cinematica diferencial
	
	syms q1 q2 q3 l2 l1 a b
	
	q2_transf = q2 - pi/2;
	q3_transf = pi/2 - q3;
	
	xp = cos(q1)*(a + l2*cos(q2_transf) + l2*cos(q3_transf));
	yp = sin(q1)*(a + l2*cos(q2_transf) + l2*cos(q3_transf));
	zp = b + l2*sin(q2_transf) - l2*sin(q3_transf) + l1;
	
	N11 = diff(xp,q1)
	N12 = diff(xp,q2)
	N13 = diff(xp,q3)
	
	N21 = diff(yp,q1)
	N22 = diff(yp,q2)
	N23 = diff(yp,q3)
	
	N31 = diff(zp,q1)
	N32 = diff(zp,q2)
	N33 = diff(zp,q3)
		 \end{lstlisting}
		 
 	 Obteniéndose los siguientes resultados:
 	 
	\begin{equation}
	N_{11} = - \sin\!\left(\mathrm{q1}\right)\, \left(a + \mathrm{l2}\, \cos\!\left(\mathrm{q2} - \frac{\pi}{2}\right) + \mathrm{l2}\, \cos\!\left(\mathrm{q3} - \frac{\pi}{2}\right)\right)
	\end{equation}
	\begin{equation}
	N_{12} = - \mathrm{l2}\, \cos\!\left(\mathrm{q1}\right)\, \sin\!\left(\mathrm{q2} - \frac{\pi}{2}\right)
	\end{equation}
	\begin{equation}
	N_{13} = - \mathrm{l2}\, \cos\!\left(\mathrm{q1}\right)\, \sin\!\left(\mathrm{q3} - \frac{\pi}{2}\right)
	\end{equation}

 	 
 	 \begin{equation}
 	 N_{21} = \cos\!\left(\mathrm{q1}\right)\, \left(a + \mathrm{l2}\, \cos\!\left(\mathrm{q2} - \frac{\pi}{2}\right) + \mathrm{l2}\, \cos\!\left(\mathrm{q3} - \frac{\pi}{2}\right)\right)
 	 \end{equation}
 	 \begin{equation}
 	 N_{22} = - \mathrm{l2}\, \sin\!\left(\mathrm{q1}\right)\, \sin\!\left(\mathrm{q2} - \frac{\pi}{2}\right)
 	 \end{equation}
 	 \begin{equation}
 	 N_{23} = - \mathrm{l2}\, \sin\!\left(\mathrm{q1}\right)\, \sin\!\left(\mathrm{q3} - \frac{\pi}{2}\right)
 	 \end{equation}

 	 
 	 \begin{equation}
 	 N_{31} = 0
 	 \end{equation}
 	 \begin{equation}
 	 N_{32} = \mathrm{l2}\, \cos\!\left(\mathrm{q2} - \frac{\pi}{2}\right)
 	 \end{equation}
 	 \begin{equation}
 	 N_{33} = \mathrm{l2}\, \cos\!\left(\mathrm{q3} - \frac{\pi}{2}\right)
 	 \end{equation}

	\begin{landscape}
	 	 Una vez resuelto matemáticamente el cálculo de todos los valores de $N_{xx}$ y sustituyendo en la ecuación \ref{cdif:eq0} se obtiene la siguiente relación:
	 	 
		\begin{equation}
		\left[ \begin{array}{c}
		\dot x_r \\
		\dot y_r \\
		\dot z_r \\
		\end{array}\right]
		=
		\left[ \begin{array}{ccc} - \sin\!\left(\mathrm{q1}\right)\, \left(a + \mathrm{l2}\, \cos\!\left(\mathrm{q2} - \frac{\pi}{2}\right) + \mathrm{l2}\, \cos\!\left(\mathrm{q3} - \frac{\pi}{2}\right)\right) & - \mathrm{l2}\, \cos\!\left(\mathrm{q1}\right)\, \sin\!\left(\mathrm{q2} - \frac{\pi}{2}\right) & - \mathrm{l2}\, \cos\!\left(\mathrm{q1}\right)\, \sin\!\left(\mathrm{q3} - \frac{\pi}{2}\right)\\ \cos\!\left(\mathrm{q1}\right)\, \left(a + \mathrm{l2}\, \cos\!\left(\mathrm{q2} - \frac{\pi}{2}\right) + \mathrm{l2}\, \cos\!\left(\mathrm{q3} - \frac{\pi}{2}\right)\right) & - \mathrm{l2}\, \sin\!\left(\mathrm{q1}\right)\, \sin\!\left(\mathrm{q2} - \frac{\pi}{2}\right) & - \mathrm{l2}\, \sin\!\left(\mathrm{q1}\right)\, \sin\!\left(\mathrm{q3} - \frac{\pi}{2}\right)\\ 0 & \mathrm{l2}\, \cos\!\left(\mathrm{q2} - \frac{\pi}{2}\right) & \mathrm{l2}\, \cos\!\left(\mathrm{q3} - \frac{\pi}{2}\right) \end{array}\right]
		\cdot
		\left[ \begin{array}{c}
		\dot q_1 \\
		\dot q_2 \\
		\dot q_3 \\
		\end{array}\right]
		\end{equation}
	\end{landscape}
	
\subsection{Rango de movimientos}
	Con los limites angulares definidos se obtiene un espacio de trabajo de brazo robótico (para el plano de trabajo XZ, en la posición y=0). Tal y como puede apreciarse en la figura \ref{fig:cinematica:espacioTrabajo}, este espacio de trabajo es suficiente para el objetivo que se desea dar. Abarca una distancia óptima para alcanzar los diferentes puntos a lo ancho de una camilla pudiendo posicionar la tablet a diferentes alturas. Para obtener el espacio de trabajo tridimensional solo hay que girar el espacio obtenido en el plano XZ al rededor del eje Z, para abarcar los $180^o$ que gira la primera articulación.
	
	 \begin{figure}[H]
	 	\centering
	 	\includegraphics[width=\textwidth]{figuras/Imagenes_cinematica/workspace_robot.jpg}
	 	\caption{Espacio de trabajo en el plano XZ para Y = 0}
	 	\label{fig:cinematica:espacioTrabajo}
	 	\immagesource{Autor.}
	 \end{figure}
\section{Realimentación articular}

	La relación entre las medidas angulares de los potenciómetros y de las articulaciones varían de una a otra. En ambos casos una vez transformado a ángulo articular habrá que aplicar el desfase para posicionar el cero angular en las posiciones descritas en la figura \ref{fig:Control:cinematica_3}. Hay que recordar de la sección \ref{sec:Electronica:Integracion} de qué manera se han integrado los potenciómetros en la estructura.
	
	   \begin{equation}
		   \textnormal{Segunda articulación:} \to	angulo_{articulacion} = angulo_{potenciometro} 
	   \end{equation}
	   
	   \begin{equation}
		   \textnormal{Tercera articulación:} \to	   angulo_{articulacion} = angulo_{potenciometro} \cdot 0.679
	   \end{equation}


 \chapter{Diseño del Control} \label{chap:Control}
\chapterimage{figuras/ImagenesPortada/PortadaControl.jpg}
\hrule
\vspace{3mm}

El manejo del brazo de una forma adecuada requiere la aplicación de diferentes lazos de control; en este capítulo se detalla el trabajo realizado en esta línea.
\\

El control más básico a realizar sobre un brazo robótico es el control en posición articular: el objetivo es poder introducir una consigna en grados (posición que se quiere alcanzar) que deberá ser alcanzada por el sistema, por cada articulación. El enfoque a seguir para conseguir este tipo de control puede variar en funcion de los diferentes componentes utilizado; en este caso el diseño de este control se ve fuertemente marcado por los actuadores elegidos, los servos Cytron G15 Cube (vistos en \ref{sec:Electronica:Actuadores}).
\\

Poder controlar la posición de las articulaciones lleva implícito el control del movimiento de las mismas. Esto implica controlar la velocidad a la que se mueven, de forma que, ajustándola se podrá acelerar y desacelerar el movimiento articular para alcanzar, de forma suave las diferentes posiciones articulares deseadas. De esta manera se enlaza un control en posición con un control de velocidad en cascada. La figura \ref{fig:Control:control_cascada} representa el lazo de control completo de posición con un lazo de velocidad integrado. En los siguientes apartados se detalla en concreto cada uno de los lazos representados.


\begin{figure}[H]
    \centering
    \includegraphics[width=1\textwidth]{figuras/Imagenes_Control/control_cascada.jpg}
    \caption{Control en cascada de Posición + Velocidad}
    \label{fig:Control:control_cascada}
    \immagesource{Autor}
\end{figure}

Este esquema de control aplica principalmente a las articulaciones dos y tres, que son las que incluyen realimentación de posición. La primera articulación se controlará en posiciones incrementales ajustando a lo comandado por parte de la tablet en el extremo.

\section{Control de velocidad para los servos} \label{sec:Control:velocidad_g15}

La relación entre la velocidad articular y la velocidad de los servos no es directa debido a la cadena de transmisión mecánica por la que pasa el movimiento. El objetivo de controlar la velocidad de la articulación es poder acelerar y desacelerar de forma controlada dicha articulación. En este caso bastará con asegurar que, cuando se pida una velocidad superior o inferior la articulación se comporte en consecuencia, sin ser tan relevante la velocidad real de la articulación. El problema, una vez concretado esto, se reduce a controlar la velocidad de los servos.
\\

Los servos elegidos, aunque poseen un control efectivo de velocidad cuando se configuran en modo posicional (se le envían posiciones objetivo en grados a alcanzar), no lo tienen así en un modo de rotación continua. La cadena cinemática construida implica que el servo deberá permitir giro multivuelta en los servos, es por ello que no se podrá utilizar el control de velocidad del modo controlado en posición si no que se utilizará el modo rotación continua, en el cual la consigna a seguir por el mismo es en par a realizar.
\\

El objetivo del lazo de control en velocidad diseñado es el de, introduciendo una consigna de velocidad a alcanzar (en revoluciones por minuto) se obtenga una salida de par a realizar para los servos, en este caso estando en el intervalo de valores de 0 a 1023, que son los valores aceptados por el servo.
\\

El problema que surge es que en función de la posición articular del brazo supone una variación de la carga sobre el servo, lo que hace que la variación del par aplicado no sea lineal a la variación de velocidad obtenida. Esto puede corregirse si para cada situación se compensa el par realizado por el peso del propio brazo para en cada instante.
Como se ha visto en la imagen \ref{fig:Control:control_cascada} la estrategia seguida ha sido la de prealimentar el controlador de velocidad. Esto significa que para cada consigna de velocidad se estimará el par requerido en cadena abierta para luego corregir el posible error y ruido con un controlador tradicional; es decir, se implementará un cálculo de la prealimentación adaptativa que se ajustará según la situación del brazo.
\\

Ajustar dicha prealimentación requiere conocer el comportamiento del servo. En la figura \ref{fig:Control:control_velocidad_par_1}\footnote{Esta gráfica surge de realizar un test de aceleración controlada almacenando los valores de tiempo, par y velocidad. Los valores se recogen de forma discreta a una frecuencia de 100 Hz. Para obtener una gráfica más significativa el test se ha realizado hasta veinte veces para, haciendo la media, obtener una relación estimada de como se comporta el servo de forma \textit{continua}.} se pude ver como se comporta la variación de velocidad del servo en vacío frente a una variación del par requerido. Se pueden reconocer dos etapas en la evolución de dichas variables. En los instantes iniciales la velocidad se mantiene a cero hasta que el par aplicado es capaz de superar el par necesario para mover el servo (superar el rozamiento). A partir de ese punto la variación de la velocidad frente al par aplicado si se comporta de manera lineal.

\begin{figure}
    \centering
    \includegraphics[width=1\textwidth]{figuras/Imagenes_Control/Control_1.jpg}
    \caption{Variación de la velocidad del servo ante variaciones del par aplicado por el mismo}
    \label{fig:Control:control_velocidad_par_1}
    \immagesource{Autor}
\end{figure}

Este comportamiento se repite de forma análoga aplicando diferentes cargas sobre el servo, tal y como puede verse en la figura \ref{fig:Control:control_velocidad_par_2}\footnote{Test de aceleración controlada discretizando los valores a una frecuencia de 100 Hz. El test para cada valor de carga se ha realizado hasta veinte veces obteniéndose una estimación del comportamiento continuo del servo haciendo la media. La carga se ha aplicado en el eje vertical de forma que se mantenga continua a lo largo del test con un montaje del servo similar al que llevará en el brazo robótico.}. Para las diferentes cargas se puede observar una zona en la que, variando el par, la velocidad permanece inalterable (zona en la cual el par aplicado por la carga es superior al aplicado por el servo) y una zona lineal\footnote{Ocurre que el servo, al funcionar en modo multivuelta va almacenando cuerda sobre la polea variando la distancia al eje desde la que se efectúa el par. A cargas elevadas supone una distorsión en el comportamiento esperado. Las repeticiones de los test se han realizado en las mismas condiciones de partida por lo que dichas distorsiones se han mantenido equivalentes en todas ellas}. En la zona de comportamiento lineal, en base a la gráfica mostrada, se puede asumir que la pendiente es igual en todos los casos. Esta deducción concuerda con el hecho de que una vez salvado el peso de la carga, un aumento en el par si debería, físicamente, suponer un aumento proporcional en la velocidad.
\\

\begin{figure}
    \centering
    \includegraphics[width=1\textwidth]{figuras/Imagenes_Control/Control_2.jpg}
    \caption{Variación de la velocidad del servo ante variaciones del par aplicado por el mismo}
    \label{fig:Control:control_velocidad_par_2}
    \immagesource{Autor}
\end{figure}

Matemáticamente significa que, conociendo el punto de funcionamiento [par,velocidad] en el que se está trabajando se pude obtener el par necesario para compensar los efectos de la gravedad sobre la carga. Los servos utilizados permiten leer, entre otros datos, estos dos valores necesarios de forma que en todo momento podemos observar la posición de la situación actual en el plano representado de par frente a velocidad.
\\

Basándose en la representación de la figura \ref{fig:Control:calculo_control} se pude ver, gráficamente como se calcula la prealimentación para el punto (n+1) conocido el punto en el instante (n). Conociendo dicho y la pendiente que relaciona la variación de ambos parámetros una vez vencida la gravedad se obtiene el par necesario para vencer dicho par:

\begin{equation}
    Torque_{Compensacion} =  Torque_n - \frac{Velocidad_{n}}{pendiente}
\end{equation}

Conocido dicho valor se puede calcular, desplazándose de vuelta hacia arriba por la recta el valor de par que proporcione la velocidad requerida:

\begin{equation}
    Torque_{n+1} = Torque_{Compensacion} - \frac{Velocidad_{target}}{pendiente}
\end{equation}

\begin{figure}[H]
    \centering
    \includegraphics[width=1\textwidth]{figuras/Imagenes_Control/calculo_control.jpg}
    \caption{Cálculo matemático de la prealimentación adaptativa.}
    \label{fig:Control:calculo_control}
    \immagesource{Autor}
\end{figure}

Este método de calcular la prealimentación a través de un observador del estado del servo conlleva cierto error en la medida obtenida en base a que la información se obtiene de forma discreta de igual forma que la actualización del par. La frecuencia a la que se actualizan dichos valores es lo suficientemente rápida como para que el efecto sea mínimo. Aún así, posible error así como ruidos o perturbaciones externas son absorbidas por la segunda parte del control en velocidad descrito: el regulador PID implementado.
\\

El regulador implementado es un regulador PID estándar cuyas constantes (constante proporcional, integral y derivativa) se han ajustado de forma experimental.
\\

La frecuencia a la que funciona el lazo de control de velocidad está limitada por el tiempo que se tarda en comunicar toda la información entre el sistema y los servos.

\section{Control de posición articular} \label{sec:Contorl:posicion_articular}

El control de posición articular recibe una consigna de posición a alcanzar y la \textit{transforma} en una consigna de velocidad para el servo. Para que el control en cascada funcione correctamente el lazo de control de velocidad ha deser más rápido que el de posición, concretamente el lazo de control de posición funciona a una frecuencia 10 veces inferior al lazo de control de velocidad.
\\

El regulador PID implementado se encarga de llevar la articulación al punto deseado. Es importante asegurar que conforme se acerca al punto de consigna se desacelere de forma que se eviten oscilaciones provocadas por los muelles que compensan la carga de la segunda y tercera articulación.


 \chapter{Diseño del software} \label{chap:SW}
\hrule
\vspace{3mm}

    En este capítulo se hace una descripción detallada del \ingles{software} involucrado en este proyecto. Primeramente se hace un repaso de la estructura de directorios, que es característica y viene previamente marcada por el software utilizado. Una vez explicado se pasará a describir las diferentes librerías que se han desarrollado (sección \ref{sec:SW:lib}) y del directorio de fuentes principal SRC (sección \ref{sec:SW:src}). Se deja para una sección posterior la interacción e integración de estos objetos así como una explicación más detallada del flujo de la información, funcionamiento del sistema como ejemplos de uso (sección \ref{sec:SW:interacion_informacion_proced}). Finalmente se exponen los test realizados sobre el software (sección \ref{sec:SW:test}) así como la gestión de la complejidad (sección \ref{sec:SW:gestion_complejidad}) del proyecto.
    \\

    Para el desarrollo y test del software se ha utilizado el editor \glosario{Atom} ampliando su funcionalidad con el paquete \glosario{PlatformIO}, que expande las capacidades del editor base para permitir trabajar con diferentes placas \completar, entre ellas las de la gama de \glosario{Arduino}.
    \\

    La elección de esta herramienta conlleva un formato en el árbol de directorios en los que se separa el código debido a la forma que tiene de compilar y lincar los diferentes ficheros. De esta manera y para mantener el orden los ficheros de código se separan en distintos directorios:

    \begin{itemize}
    	\item lib: directorio donde se introducen, en carpetas, las librerías que se van a utilizar.
    	\item src: directorio donde se introduce el fichero o ficheros de código principales.
    	\item test: directorio donde se introducen los ficheros donde se codifican los test. Estos irán metidos dentro de directorios con el nombre de cada test.
    \end{itemize}

    Concretamente, en el caso de este proyecto, queda el siguiente árbol de directorios en el cual se han expandido hasta el nivel de ficheros en algunos casos de modo que sirvan de ejemplo:

    \lstset{language=C, breaklines=true, basicstyle=\footnotesize}
        %Introducir label y caption
        \begin{lstlisting}[frame=single]

     -- Sw
     |-- lib
     |   |-- debug
     |   |   | + debug.cpp
     |   |   | + debug.h
     |   |-- joint_handler
     |   |   | + joint_handler.cpp
     |   |   | + joint_handler.h
     |   |-- joint_rha
     |   |-- rha_types
     |   |-- robot_rha
     |   |-- servo_rha
     |   |-- chuck_handler
     |   |-- readme.txt
     |-- src
     |   |-- main.cpp
     |   |-- utilities.cpp
     |   |-- utilities.h
     |-- test
     |   |-- a_test_servo_rha
     |   |   | + test_servo_rha.cpp
     |   |-- b_ttest_servo_realest_joint_rha
     |   |-- c_test_joint_handler_mock
     |-- platformio.ini
     -------
     |-- code_analysis
     |-- makeAnalysis.sh
        \end{lstlisting}

    En los siguientes apartados se hará un análisis detallado del contenido de estos directorios. Se desarrollan primero los ficheros con información auxiliar y posteriormente en una jerarquía desde más afuera \textcolor{pRojo}{$<$---- ¿?¿W T F?¿?} detallando luego los componentes internos.
\section{Librerías (directorio lib)} \label{sec:SW:lib}

    \subsection{debug} \label{subsec:SW:lib:debug}
        Dentro de este directorio se encuentra el fichero debug.h donde se han definido macros para \glosario{debug} de los diferentes espacios. A continuación se muestra un ejemplo de como se codifican dichas macros para hacer un seguimiento de la clase \ingles{servo\_rha} (Ver apartado \ref{subsec:SW:lib:servo_rha}).

        \lstset{language=C, breaklines=true, basicstyle=\footnotesize}
        %Introducir label y captio
        \begin{lstlisting}[frame=single]

    #ifdef DEBUG_SERVO_RHA
        #define DebugSerialSRHALn(a) {  Serial.print("[DC]  ServoRHA::"); Serial.println(a); }
        #define DebugSerialSRHALn2(a, b) {  Serial.print("[DC]  ServoRHA::"); Serial.print(a); Serial.println(b); }
        #define DebugSerialSRHALn4(a, b, c, d) {  Serial.print("[DC]  ServoRHA::"); Serial.print(a); Serial.print(b); Serial.print(c); Serial.println(d); }
    #else
        #define DebugSerialSRHALn(a)
        #define DebugSerialSRHALn2(a, b)
        #define DebugSerialSRHALn4(a, b, c, d)
    #endif

        \end{lstlisting}

        Como se puede apreciar estas macros utilizan la comunicación serial de \glosario{Arduino}, concretamente la función \ingles{print} y \ingles{println} de la librería \ingles{Serial} para mostrar los mensajes en un formato determinado. Además se definen de tal forma que se pueden activar o desactivar (en este mismo fichero \ingles{debug.h}) de forma que se enviarán o no los mensajes de \glosario{debug}. A través de este tipo de mensajes se puede hacer un seguimiento de la ejecución de los diferentes métodos o funciones de la librería afectada para ubicar fallos en los mismos.

        Para activar la opción de \glosario{debug} bastará con descomentar la línea correspondiente:

        \lstset{language=C, breaklines=true, basicstyle=\footnotesize}
        %Introducir label y captio
        \begin{lstlisting}[frame=single]

    // #define DEBUG_SERVO_RHA
    // #define DEBUG_TEST_SERVO_RHA_MOCK
    #define DEBUG_TEST_SERVO_RHA_REAL
    // #define DEBUG_CYTRON_G15_SERVO
    // #define DEBUG_TEST_CYTRON_G15_SERVO

        \end{lstlisting}

    \subsection{rha\_types} \label{subsec:SW:lib:rha_types}
        Dentro de este directorio se encuentra el fichero \ingles{rha\_types.h} donde se definen algunos tipos de datos y estructuras que se usan en el proyecto.
        \\

        Estas estructuras de datos se listan a continuación, pudiendo ver sus respectivos diagramas de clases en la figura \ref{fig:SW:class_diagram_TRHA}:

        \begin{itemize}
            \item \ingles{SpeedGoal} condensa en un solo objeto toda la información necesaria para codificar un objetivo de velocidad.
            \item \ingles{Regulator} encapsula el funcionamiento de un \glosario{Regulador-PID} estandar. Guarda los valores de las constantes así como de la integral del error para luego, pasándole error, derivada del error e integral del error en un intervalo poder devolver la salida del regulador.
            \item \ingles{Timer} codifica un temporizador (en milisegundos) de forma que se le podrá preguntar al objeto si el tiempo ya ha pasado, bloqueando o no la ejecución del programa hasta el final del tiempo.
            \item \ingles{TimerMicroseconds} hereda las características del objeto \ingles{Timer} funcionando en microsegundos.

        \end{itemize}

        \begin{figure}[H]
            \centering
            \includegraphics[width=1\textwidth]{figuras/Imagenes_SW/class_diagram_TRHA.jpg}
            \caption{Estructuras de datos auxiliares}
            \label{fig:SW:class_diagram_TRHA}
        \end{figure}

        Se puede consultar información de más bajo nivel referente a estos tipos de datos en el Anexo \ref{app:documentacion_software} sección \completar.
    \subsection{joint\_handler} \label{subsec:SW:lib:joint_handler}
        La librería \ingles{joint\_handler} se hace cargo de generar un objeto de la clase \ingles{JointHandler} que será en encargado de gestionar la sincronización de todas las articulaciones. Este objeto es propietario de las articulaciones e implementa un método que cíclicamente sincroniza el funcionamiento de todas las articulaciones. En el objeto \ingles{JointHandler} se implementa a su vez la comunicación con los \glosarioPlural{servo}, es decir, el encapsulamiento de los paquetes de datos con la información genérica de forma que la información que se envía cumpla con el protocolo de comunicación que comparten los \glosarioPlural{servo}. Además implementa las funciones que gestionan el envío y recepción de dichos paquetes de datos.

        El bucle de control implementado se encarga de ir llamando una a una a todas las articulaciones para que hagan las siguientes operaciones (se puede ver representado el funcionamiento de dicho ciclo en la Figura \ref{fig:SW:joint_handler_loop}):
        \begin{enumerate}
            \item Asegurar que cada articulación actualice la información propia, compuesta por la posición recibida de la realimentación así como toda la información proveniente del servo (datos de posición, velocidad, par soportado y dirección en que se aplican, voltaje y temperatura).
            \item Llamar a cada articulación para que se realicen los cálculos correspondientes del \ingles{torque} que se enviará calculado a partir del error entre la velocidad real y deseada. Este valor queda guardado en cada servo para ser posteriormente empaquetado.
            \item Invocar a cada servo para que se adhieran al paquete de información que se va a enviar.
            \item Se enpaqueta la información a enviar a los diferentes \glosarioPlural{servo} en un mismo paquete, añadiendo posteriormente el correspondiente encabezado así como el comprobante de que el paquete se ha recibido correctamente (checksum). Una vez preparado el paquete se envía por el puerto serie.
        \end{enumerate}

        \begin{figure}[H]
        \centering
        \includegraphics[width=0.45\textwidth]{figuras/Imagenes_SW/joint_handler_loop.png}
        \caption{Esquema de ejecución de bucle de control de joint\_handler}
        \label{fig:SW:joint_handler_loop}
        \end{figure}

        Como se puede intuir es en este objeto donde realmente se implementa el lazo de control de velocidad para todos los \glosarioPlural{servo} conectados al bus. Esta serie de operaciones descrita anteriormente constituye, de forma discretizada, el lazo de control representado en la Figura \ref{fig:SW:servo_control_loop} para cada servo y asegura su correcto funcionamiento y sincronismo.

        \begin{figure}[H]
        \centering
        \includegraphics[width=0.85\textwidth]{figuras/Imagenes_SW/servo_control_loop.png}
        \caption{Lazo de control de la velocidad del servo. Cálculos realizados por el objeto \ingles{ServoRHA}.}
        \label{fig:SW:servo_control_loop}
        \end{figure}

        \begin{figure}[H]
            \centering
            \includegraphics[width=1\textwidth]{figuras/Imagenes_SW/class_diagram_JH.jpg}
            \caption{Atributos y métodos más relevantes del objeto \textit{JointHandler}}
            \label{fig:SW:class_diagram_JH}
        \end{figure}

        Se puede consultar información de más bajo nivel referente al objeto \ingles{JointHandler} así como a sus atributos y métodos en el Anexo \ref{app:documentacion_software} sección \completar.

    \subsection{joint\_rha} \label{subsec:SW:lib:joint_rha}
        La librería \ingles{joint\_rha} implementa un objeto de tipo \ingles{JointRHA} que aúna en un mismo objeto las lecturas y capacidades del objeto \ingles{ServoRHA} (explicado en la sección \ref{subsec:SW:lib:servo_rha}) junto con la realimentación de posición de la articulación en cuestión.

        \begin{figure}[H]
            \centering
            \includegraphics[width=0.100\textwidth]{figuras/Imagenes_SW/class_diagram_JRHA.jpg}
            \caption{Atributos y métodos más relevantes del objeto \textit{JointRHA}}
            \label{fig:SW:class_diagram_JRHA}
        \end{figure}

        Se puede consultar información de más bajo nivel referente al objeto \ingles{JointRHA} así como a sus atributos y métodos en el Anexo \ref{app:documentacion_software} sección \completar.

    \subsection{servo\_rha} \label{subsec:SW:lib:servo_rha}
        La librería \ingles{servo\_rha} implementa un objeto del tipo \ingles{ServoRHA} que está encargado de gestionar toda la información referente al servo. Se encarga de encapsular la información específica a un servo en paquetes de datos a petición del objeto \ingles{JointHandler} que luego serán procesados por el mismo previo a ser enviados a través del bus. Estos paquetes se forman a partir de la información contenida en el objeto \ingles{ServoRHA}, que además de formar los paquetes a enviar almacena la información referente al servo que se recibe de los mismos.
        \\

        Gestiona además una parte importante referente al lazo de control de velocidad del servo vista en el apartado \ref{subsec:SW:lib:joint_handler}. El objeto \ingles{ServoRHA} contiene los datos del regulador utilizado así como el offset a aplicar. De esta forma, recibiendo el error realiza las operaciones para calcular y empaquetar el \ingles{torque} deseado. En la Figura \ref{fig:SW:servo_control_loop_servo_part} se representa, recuadrado, la parte correspondiente del lazo de control de velocidad (representado anteriormente en la Figura \ref{fig:SW:servo_control_loop}) que efectúa el objeto en cuestión.
        \begin{figure}[H]
        \centering
        \includegraphics[width=0.85\textwidth]{figuras/Imagenes_SW/servo_control_loop_servo_part.png}
        \caption{Lazo de control de la velocidad del servo. Cálculos realizados por el objeto \textit{ServoRHA}.}
        \label{fig:SW:servo_control_loop_servo_part}
        \end{figure}

        \begin{figure}[H]
            \centering
            \includegraphics[width=0.8\textwidth]{figuras/Imagenes_SW/class_diagram_SRHA.jpg}
            \caption{Atributos y métodos más relevantes del objeto \textit{ServoRHA}}
            \label{fig:SW:class_diagram_SRHA}
        \end{figure}

        Se puede consultar información de más bajo nivel referente al objeto \ingles{ServoRHA} así como a sus atributos y métodos en el Anexo \ref{app:documentacion_software} sección \completar.

    \subsection{robot\_rha} \label{subsec:SW:lib:robot_rha}
        En la librería \ingles{robot\_rha} se implementa el objeto de tipo \ingles{RobotRHA} encargado de coordinar el funcionamiento del robot. Implementa el ciclo de más alto nivel donde se actualizan los objetivos ya sean de posicion o velocidad para llamar a las funciones correspondientes que lo traducen a valores articulares que luego se ejecutan.
        \\

        Tiene diferentes modos de funcionamiento según de donde provengan los comandos a seguir:
        \begin{itemize}
            \item Control a través de un \ingles{Nunchuk} de la consola \ingles{Wii}. Se controla la velocidad del robot en sus diferentes ejes a través del joystic y los botones del mando.

        \end{itemize}

    \subsection{chuck\_handler} \label{subsec:SW:lib:chuck_handler}
        La librería \ingles{chuck\_handler} codifica el objeto de tipo ChuckHandler que se encarga de implementar todo lo necesario para realizar lecturas del mando \ingles{Nunchuck} de la \ingles{Wii} y devolver comandos de velocidad con un periodo establecido. Se llamará al método correspondiente de forma continuada devolviendo este los valores de velocidad cuando se cumpla el tiempo mínimo establecido.

        \begin{figure}[H]
          	\centering
          	\includegraphics[width=0.65\textwidth]{figuras/Imagenes_SW/class_diagram_CHH.jpg}
          	\caption{Atributos y métodos más relevantes del objeto \textit{ChuckHandler}}
          	\label{fig:SW:class_diagram_CHH}
        \end{figure}

\section{SRC. Fichero de código principal} \label{sec:SW:src}

\section{Interacción entre objetos, flujo de la información y de procedimientos} \label{sec:SW:interacion_informacion_proced}

	Como se ha visto en los apartados anteriores (sección \ref{sec:SW:lib}) el control de la información está separado entre los diferentes objetos. Primeramente se va a detallar como se distribuye y gestiona la información para el control de velocidad, revisando como se construyen los paquetes necesarios para satisfacer los requerimientos del protocolo de comunicación de los servos.
	\\

	\begin{figure}[H]
		\centering
		\includegraphics[width=0.45\textwidth]{figuras/Imagenes_SW/class_relation.png}
		\caption{Relaciones entre las diferentes Clases \completarCon{Hacer imagen con el mismo estilo que anteriormente}}
		\label{fig:SW:class_relation}
	\end{figure}

	Como se ha visto es el objeto de tipo \ingles{JointRHA} el encargado de almacenar la información concreta de cada articulación. Aúna la información de la realimentación de posición del potenciómetro con la contenida en el objeto \ingles{ServoRHA} que contiene. Este último almacena la información referente al servo, tanto el estado como la acción de control de velocidad así como la próxima consigna de velocidad (traducida a par) que se va a enviar.
	\\

	La información fluye entre los diferentes objetos en forma de paquetes (vectores o arrays de \ingles{bytes}) que se van completando por el responsable de cada operación.
	\\

	El objeto JoingHandler es el encargado de adecuar la información al protocolo de comunicación concreto de los servos. Para ello pide a los servos la información concreta que se va a enviar a cada en cada caso. Se exponen a continuación dos ejemplos:
	\begin{enumerate}
		\item Un primer caso de lectura de información (actualizar la información de los servos)
		\item Un ejemplo de una operación de escritura (enviar una consigna de velocidad a los servos).
	\end{enumerate}

	En el primer caso será el objeto \ingles{JointHandler} quien pase un paquete vacío al objeto \ingles{JointRHA} (en este caso se pretende actualizar la información del objeto \ingles{ServoRHA} por lo que se accederá al mismo directamente) con una petición de que se almacene en dicho paquete la información referente a la lectura: cuantos bytes se quieren leer y a partir de que dirección de memoria en los registros de los servos (se puede ver una tabla con todas las direcciones de memoria disponibles en los servos en el Anexo \ref{app:registros_g15}).
	\\

	De esta forma el Servo recibe un paquete vacío que rellena con la información correspondiente y lo devuelve al objeto \ingles{JointHandler}, que, a partir de esa información monta un paquete acorde con el protocolo de los servos añadiendo la cabecera correspondiente, la comprobación del error y la orden para que se realice una operación de lectura, entre otros datos.
	\\

	Una vez enviado, y tratándose de una operación de lectura se espera la respuesta por parte del servo correspondiente, en caso de ser satisfactoria (no se haya producido ningún error durante la operación) se filtra nuevamente la cabecera y demás información propia del protocolo de comunicación para volver a enviar, ahora lleno, el paquete de información con los bytes pedidos al objeto ServoRHA.

	\completarCon{Incluir diagrama con el flujo de información y ejemplo}

	El segundo caso será equivalente. El servo rellena en el \ingles{buffer} la información propia como es el ID del servo, la posición sobre la que se quiere escribir, el número de bytes a escribir y la información que se escribirá.
	\\
	El objeto \ingles{JointHandler} se encarga de rellenar los datos propios del protocolo añadiendo en este caso una orden de escritura en el paquete.
	\\
	\completarCon{Incluir diagrama con el flujo de la información y ejemplo}

	\completarCon{Comentar el resto del software}

\section{Test y verificación del software} \label{sec:SW:test}
    Como parte del proyecto se han desarrollado una serie de test para verificar el correcto funcionamiento de las diferentes librerías. El \ingles{testing} del \ingles{software} se ha utilizado como herramienta implémentandose solo en aquellos casos en que facilita la verificación y comprobación del correcto funcionamiento del código implementado. El \ingles{testing} del código no forma parte del núcleo del proyecto por lo que no se han establecido porcentajes mínimos de cobertura (cantidad de código que se está evaluando a través de las pruebas) ni objetivos mínimos. Los test son puramente funcionales desarrollándose los necesario y no forzando el desarrollo de los distintos niveles de \ingles{test}: test unitarios, test de integración y test de sistemas. \completarCon{Alguna cita que hable de los niveles de test}

    Para el desarrollo y ejecución de dichos test se ha utilizado la funcionalidad de test que viene integrada en \glosario{PlatformIO}: \glosario{PlatformIO_Test}. Esta herramienta, permite definir una serie de test que pueden ser ejecutados en la propia placa. De esta forma se puede automatizar el proceso de test.

    Una completa definición de test (desde test unitarios hasta test de integración) permite controlar de forma continuada el correcto funcionamiento del sistema frente a modificaciones en el código. De forma genérica estos test se han agrupado bajos los siguientes nombres:

    \begin{itemize}
        \item test\_rha\_types
        \item test\_pid\_regulator
        \item test\_servo\_rha
        \item test\_joint\_rha
        \item test\_joint\_handler\_mock
    \end{itemize}

    El el directorio SW/test se encuentran definidos los diferentes test que se realizan. Cada fichero de test, destinado a testear de forma parcial o completa una librería, tiene diferentes funciones de test definidas.

    Para realizar un test sobre un método o grupo de métodos se define una función de test en la que se define la ejecución que se va a realizar, con las entradas predefinidas de forma que se puede comprobar como ciertas salidas o parámetros internos satisfacen las necesidades impuestas. Para la definición de estas condiciones así como de los test se sigue el formato propuesto desde \glosario{PlatformIO_Test} y la API que adjuntan.

    Una vez codificados los ficheros para testing es necesario especificar su uso en el fichero de código principal \codigo{src/main.cpp} mediante las directivas al preprocesador que se muestran a continuación. De esta forma mientras se realizan los test no se ejecutará el programa principal.

    \begin{lstlisting}[frame=single]
    #ifndef UNIT_TEST  // disable program main loop while unit testing in progress
    ...
    ...
    ...
    #endif
    \end{lstlisting}

    \begin{figure}[H]
       	\centering
       	\includegraphics[width=1\textwidth]{figuras/Imagenes_SW/test/SWTest_1.jpg}
       	\caption{Salida de Platformio Test para cada caso}
       	\label{fig:SW:test:standard_output}
    \end{figure}

    \begin{figure}[H]
       	\centering
       	\includegraphics[width=1\textwidth]{figuras/Imagenes_SW/test/SWTest_3.jpg}
       	\caption{Salida de Platformio Test con casos fallidos}
       	\label{fig:SW:test:error_output}
    \end{figure}

    \begin{figure}[H]
    	\centering
    	\includegraphics[width=0.75\textwidth]{figuras/Imagenes_SW/test/SWTest_6.jpg}
    	\caption{Salida de Platformio Test con el resumen de los casos de test}
    	\label{fig:SW:test:sum_output}
    \end{figure}

	\subsection{Test librería RHATypes}

		\begin{figure}[H]
			\centering
			\includegraphics[width=0.95\textwidth]{figuras/Imagenes_SW/test/SWTest_2.jpg}
			\caption{Test satisfactorios RHATypes}
			\label{fig:SW:test:rha_types_ok}
		\end{figure}
		\begin{figure}[H]
			\centering
			\includegraphics[width=1\textwidth]{figuras/Imagenes_SW/test/SWTest_9.jpg}
			\caption{Test satisfactorios PIDRegulator (caso dentro de RHATypes)}
			\label{fig:SW:test:pid_regulator_ok}
		\end{figure}
	\subsection{Test librería ServoRHA}
		\begin{figure}[H]
			\centering
			\includegraphics[width=1\textwidth]{figuras/Imagenes_SW/test/SWTest_4.jpg}
			\caption{Test satisfactorios ServoRHA}
			\label{fig:SW:test:servo_rha_ok}
		\end{figure}
	\subsection{Test librería JointRHA}
	\subsection{Test librería JointHandler}
		Test para comprobar el correcto funcionamiento de los métodos encargados de gestionar la construcción en interpretación de los paquetes de datos. No hay comunicación real con los servos en esta serie de test.
		\begin{figure}[H]
			\centering
			\includegraphics[width=0.95\textwidth]{figuras/Imagenes_SW/test/SWTest_5.jpg}
			\caption{Test satisfactorios JointHandler}
			\label{fig:SW:test:joint_handler_ok}
		\end{figure}
	\subsection{Test librería ChuckHandler}
	\subsection{Test librería RobotRHA}

	Resumen de los test:
	\begin{figure}[H]
		\centering
		\includegraphics[width=0.95\textwidth]{figuras/Imagenes_SW/test/SWTest_8.jpg}
		\caption{Resumen Test satisfactorios}
		\label{fig:SW:test:sum_ok}
	\end{figure}

\section{Gestión de la complejidad y mantenibilidad:} \label{sec:SW:gestion_complejidad}
    Para controlar el desarrollo del proyecto de forma paralela en todas sus partes asegurando así un control de la complejidad y mantenibilidad se han ido controlando diferentes métricas referentes al software del proyecto.
    \\

    Además de dichas métricas se han ido haciendo revisiones periódicas del cumplimiento las reglas de codificación en el software (ver Anexo \ref{app:codificacionSW}). Para ello se ha utilizado un \textit{script} \glosario{cpplint} que automatiza la revisión del código.
    \\

    Para obtener la información referente a la complejidad y desarrollo del software se han utilizado dos herramientas (\glosario{lizard} y \glosario{Cloc}) junto con una serie de \ingles{scripts} que se han desarrollado para automatizar la obtención y visualización de la información. Las métricas que se han obtenido y valorado son:

    \begin{enumerate}
        \item Número de líneas de código.
        \item Número de líneas de comentarios.
        \item Número de líneas de mensajes de \ingles{debug}.
        \item Porcentaje de líneas de comentarios (media de todos los ficheros así como máximos y mínimos).
        \item Porcentaje de líneas de mensajes de \ingles{debug} (media de todos los ficheros así como máximos y mínimos).
        \item Número de ficheros.
        \item Número de funciones.
        \item Media de métodos por fichero.
        \item Complejidad ciclomática \completarCon{Comentar que es la complejidad ciclomática...¿Glosario o directamente aquí?} (media entre todos los métodos así como valores extremos).
    \end{enumerate}

    Las métricas de la uno a la cinco de la lista anterior permiten controlar un desarrollo paralelo y equilibrado entre el código así como la documentación del mismo. De igual forma, aunque menos importante, permiten ver el desarrollo paralelo de métodos de \ingles{debugging}. Este se considera menos importante ya que en su mayoría se ha implementado, más que como metodología de desarrollo, cuando las pruebas del software lo requieren.
    \\

    En la Figura \ref{fig:SW:code_analysis} se puede una serie de gráficas donde se puede ver la relación entre el código, la documentación (comentarios) y el \ingles{debuging} (líneas de \ingles{debug}).

    En la imagen se muestra el desarrollo temporal de dichos parámetros de forma que se puede apreciar un desarrollo paralelo tanto del código como de la documentación. Además, se puede apreciar como la relación entre documentación y el \ingles{debuging} con el total de líneas se mantiene relativamente constante gracias a las gráficas porcentuales. Esto concuerda con la metodología de desarrollo adoptada para el software del proyecto, y su control periódico a lo largo del tiempo ha permitido corregir desvíos en cualquiera de las partes.

    \begin{figure}[H]
        \centering
        \includegraphics[width=0.75\textwidth]{figuras/Imagenes_SW/analisis_codigo.png}
        \caption{Análisis del código del proyecto}
        \label{fig:SW:code_analysis}
    \end{figure}

    \subsection{Reglas de codificación del Código}

    \begin{figure}[H]
    	\centering
    	\includegraphics[width=0.65\textwidth]{figuras/Imagenes_SW/test/ReadibilityTest.jpg}
    	\caption{Ejemplo de la Salida de cpplint con errores}
    	\label{fig:SW:test:codif_output}
    \end{figure}

    \begin{figure}[H]
    	\centering
    	\includegraphics[width=1\textwidth]{figuras/Imagenes_SW/test/ReadibilityTest_2.jpg}
    	\caption{Ejemplo de errores con cpplint}
    	\label{fig:SW:test:error_output_cpplint}
    \end{figure}

    \begin{figure}[H]
    	\centering
    	\includegraphics[width=1\textwidth]{figuras/Imagenes_SW/test/ReadibilityTest_3.jpg}
    	\caption{Salida de cpplint una vez eliminados los errores}
    	\label{fig:SW:test:cpplint_ok}
    \end{figure}


 %\include{capitulos/ejemplos}

 \chapter{Resultados y discusión} \label{chap:Resultados}
\hrule
\vspace{3mm}

En este capítulo...


\section{Resultados}


\section{Discusión}

 \chapter{Gestión del proyecto} \label{chap:Gestion}
\chapterimage{figuras/ImagenesPortada/PortadaPlanificacion.jpg}
\hrule
\vspace{3mm}

En este capítulo se describe la gestión del proyecto: ciclo de vida, planificación, presupuesto, etc.

\section{Ciclo de vida}

Explicación de las fases del proyecto: definición, análisis, diseño, construcción, pruebas, implementación, validación, documentación. Ejemplo: diagrama de Pert.

\section{Planificación}

Se puede indicar mediante un diagrama de Gantt.

\subsection{Planificación inicial}

\subsection{Planificación final}


\section{Presupuesto}

    Coste de los materiales en la Tabla \ref{tab:presupuesto}:

    \completar

    \renewcommand{\thempfootnote}{\arabic{mpfootnote}}

    \begin{table}[H]
    \caption{Costes del proyecto}
    \label{tab:presupuesto}
    \begin{center}
    \begin{minipage}{\textwidth}
    \begin{tabular}{ |c|c|c|c| }
    \hline
    Artículo & Coste Unitario\footnote{En los casos en qué ha sido necesario se ha aplicado el cambio a Euros oficial propuesto por el Banco de España en el día en que se han consultado los precios: \url{https://www.bde.es/bde/es/}} & $N^o$ de unidades & Total\\
    \hline
    \hline
    Arduino Uno & 17.00 \euro\footnote{Precios consultados en la página oficial de Arduino a 07 de Septiembre 2017 (\url{https://store.arduino.cc/arduino-uno-rev3}).} & 1 & 17.00 \euro\\
    Cytron G15 Cube Servo & 23.23 \euro\footnote{Precios consultados en la página oficial de Cytron a 07 de Septiembre 2017 (\url{http://www.cytron.com.my/p-g15}).} & 3 & total\\
    Cytron G15 Shield & 6.64 \euro\footnote{Precio consultados en la página oficial de Cytron a 07 de Septiembre 2017 (\url{https://www.cytron.com.my/p-shield-g15}).} & 1 & 6.64 \euro\\
    \hline
    \hline
    Potenciómetro Serie TW & 9.29 \euro\footnote{Precio consultados en la página oficial de RS a 07 de Septiembre 2017 (\url{http://uk.rs-online.com/web/p/potentiometers/5028586/})} & 2 & 18.58\euro \\
    \hline
    \hline
    Barras Aluminio sección cuadrada & 5,626 \euro\footnote{Precio consultados en la página oficial de RS a 07 de Septiembre 2017 (\url{http://es.rs-online.com/web/p/tubos-de-aluminio/3047894/})} & 5 & 28,13 \euro \\
    Rodamiento 13x4 & 1.83\euro\footnote{Precio consultados en la página oficial de RS a 20 de Enero 2018 (\url{https://es.rs-online.com/web/p/rodamientos-de-bola/6189890/})} & & \euro \\
    Rodamiento 10x3 & 2.02\euro\footnote{Precio consultados en la página oficial de RS a 20 de Enero 2018 (\url{https://es.rs-online.com/web/p/rodamientos-de-bola/6189856/})} & & \euro \\
    Hilo Kevlar & 23.23 \euro\footnote{Precio consultados en la página de compra a 07 de Septiembre 2017 (\url{http://www.emmakites.com/index.php?main_page=product_info&cPath=336_365&products_id=1199})} & 1 & 23.23 \euro \\
    GM Series Plastic Wheel & 2.70 \euro\footnote{Precio consultados en la página oficial de Solarbotics a 11 de Septiembre 2017 (\url{https://solarbotics.com/product/gmpw/})} & 1 & 2.70 \euro \\
    \hline
    \end{tabular}
    \end{minipage}
    \end{center}
    \end{table}

\subsection{Personal}

\subsection{Material}

\subsection{Resumen de costes}


 \chapter{Conclusiones} \label{chap:Conclusiones}
\chapterimage{figuras/ImagenesPortada/PortadaConclusion.jpg}
\hrule
\vspace{3mm}

Este capítulo aúna las conclusiones así como posibles líneas de desarrollo a futuro.

\section{Conclusión} \label{sec:Conclusiones:Conclusion}

Una vez se ha alcanzado un punto y aparte en este proyecto se puede afirmar que los objetivos inicialmente planteados han sido cumplidos.
\\

El desarrollo del proyecto ha supuesto un aprendizaje extenso en el área de diseño y fabricación en 3D así como una familiarización con las estructuras diseñadas. Estos conocimientos han permitido obtener una compensación de la carga del brazo bastante notable que ha permitido alcanzar una mayor seguridad al reducir la fuerza de los motores, y por tanto su capacidad de provocar daños.
\\

Se ha implementado de manera más que satisfactoria una librería para la gestión de los servos que gestiona, a bajo nivel, la comunicación serie con los mismos. Además se ha diseñado un protocolo de comunicación de tamaño y tipo de mensaje variable para la comunicación del brazo con el exterior. Aunque podría acoplarse a cualquier otro dispositivo se ha diseñado una interfaz gráfica en python que permite un control intuitivo y claro del brazo robótico. Estos aspectos han supuesto un aprendizaje en el desarrollo de protocolos de comunicación a bajo nivel y diferentes aspectos a controlar (longitud, encabezado, integridad del mensaje). Este desarrollo se ha llevado a cabo en el lenguaje nativo de Arduino, C++, pero también en Python, lenguaje en el que se ha implementado la interfaz gráfica.
\\

En lo referente al control se ha diseñado una estructura de control que permite un movimiento controlado  y estable aun contando con los muelles de la segunda y tercera articulación como con el hecho de estar la articulación suelta en  uno de los sentidos (cabe recordar que el hilo se recoge solo en un sentido). Esta estructura de control aprovecha al máximo la información proporcionada por los servos para adaptarse a la situación de carga para cada punto.
\\

Aunando los aspectos descritos se ha conseguido implementar un brazo robótico capaz de sostener una tablet Surface de \cite{microsoftSurface} y posicionarla dentro en su espacio de trabajo. Las características del brazo robótico así como consideraciones en el software permiten unos niveles de seguridad elevados para los usuarios con los que interactúe.

\section{Desarrollos futuros} \label{sec:Conclusiones:Desarrollos_futuros}

Como se ha visto en el apartado \ref{gestion:planificacion} el desarrollo del prototipo se plantea de manera iterativa en su totalidad. Una vez pasado el primer ciclo de desarrollo se han planteado diferentes líneas a mejorar en una posible segunda iteración. Estas mejoras se detallan a continuación.
\subsection{Primera Articulación}

El giro en el eje Z planeta un reto complicado. Está encargado de mantener el brazo erguido en todo momento sin que esto suponga un perjuicio a la libertad de giro de la articulación. Se plantea sustituir el rodamiento axial por un rodamiento de bolas de tamaño equivalente de forma que ambas partes de la articulación puedan atornillarse al mismo aumentando la robustez de la unión. 
\\

Desde el punto de vista de la transmisión de movimiento se plantea guiar el movimiento desde el motor hasta la rueda de transmisión a través de correas de distribución asegurando la transmisión en esta primera etapa aun manteniendo la flexibilidad de la rueda de transmisión a la base. El giro en Z del brazo robótico seguirá estando condicionado a la fuerza que se oponga impidiendo así que pueda causar daños a los usuarios.

\subsection{Seguridad Software}

El grado de seguridad alcanzada con la estructura es bastante notable, aún así se puede ampliar la seguridad implementada por software extendiéndola a más niveles. La información utilizada para actualizar el controlador de velocidad descrito en la sección \ref{sec:Control:velocidad_g15} puede dar información útil de la situación del brazo robótico que puede ser aprovechada para gestionar la seguridad del mismo.
\\

De igual manera, a todos los niveles se pueden implementar protocolos para gestionar los diferentes tipos de errores de forma que se evite el bloqueo y la necesidad de reiniciar el sistema en caso de diferentes errores.

\subsection{Grados de libertad de Orientación}

En el marco de este proyecto de fin de grado no se han llegado a implementar los grados de libertad de orientación para la tablet. 
\\

Se ha planteado implementar dos grados de libertad que controlen la inclinación de la tablet, fijándose en la figura \ref{fig:Conclusiones:orientacion_tablet} estos dos grados de libertad gestionarán el giro al rededor del eje Z y del eje Y respectivamente. Se plantea utilizar un par de accionamientos paralelos, similar a los vistos en \cite{Chung2009} en el capítulo \ref{chap:estadoarte}, accionados por medio de dos servos ubicados en el extremo del brazo.

\begin{figure}[H]
	\centering
	\includegraphics[width=1\textwidth]{figuras/Imagenes_Conclusion/orientacion_tablet.jpg}
	\caption{Sistema de referencia para los grados de libertad de orientación}
	\label{fig:Conclusiones:orientacion_tablet}
	\immagesource{Imagen del fabricante editada por el autor}
\end{figure}

\subsection{Mejoras generales}

Una segunda iteración podría centrarse también en identificar diferentes mejoras posibles a robustecer que hayan pasado desapercibidas en los test realizados para los diferentes aspectos del diseño.

 \appendix
\addcontentsline{toc}{part}{\appendixname}

%\begin{appendices}

     \chapter{Listado de piezas diseñadas} \label{app:listadoPiezas}
         \hrule
         \vspace{3mm}
     	 \completar
\begin{center}
\begin{longtable}{|c|c|c|c|c|c|}
\caption{Listado de piezas diseñadas de fabricación propia}\\
\hline
\textbf{Num} & \textbf{Esquema Pieza} & \textbf{Referencia} & \textbf{Cantidad} & \textbf{Descripción} & \textbf{Peso Estimado}$^1$ \\
\hline
\endfirsthead
\multicolumn{5}{c}%
{\tablename\ \thetable\ -- \textit{Continuación de la página anterior}} \\
\hline
\textbf{Num} & \textbf{Esquema Pieza} & \textbf{Referencia} & \textbf{Cantidad} & \textbf{Descripción} & \textbf{Peso Estimado}$^1$ \\
\hline
\endhead
\multicolumn{5}{l}{\begin{minipage}{.8\linewidth}
	%do not draw the footnoterule
	\footnotesize{$^1$ El peso estimado se obtiene con el programa Cura \completar aplicando los parámetros de la tabla \ref{tab:listadoPiezas:param_impresion}. Este peso incluye el de los soportes necesarios para su impresión.}
\end{minipage}} \\ 
\hline \multicolumn{5}{r}{\textit{Continua en la página siguiente}} \\
\endfoot
\hline
%\insertTableNotes
\endlastfoot
1 & \iconoImagen{Base} & blah & 1  & blah & 193g \\
\hline
2 & \iconoImagen{RailA} & blah & 1 & blah & 24g \\
\hline
3 & \iconoImagen{RailB} & blah & 1 & blah & 24g \\
\hline
4 & \iconoImagen{PoleaMotor} & blah & 2 & blah & 3g \\
\hline
5 & \iconoImagen{EncajeTuboInterior} & blah & 1 & blah & 54g \\
\hline
6 & \iconoImagen{EncajeTuboExterior} & blah & 1 & blah & 128g \\
\hline
7 & \iconoImagen{FijacionBarraPieB} & blah & 1 & blah & blah \\
\hline
8 & \iconoImagen{FijacionBarraPie} & blah & 1 & blah & blah \\
\hline
10 & \iconoImagen{RuedaMotorGiroZ} & blah & 1 & blah & 18g \\
\hline
19 & \iconoImagen{AdaptadorPoleaNegra} & blah & 2 & blah & 1g \\
\hline
19 & \iconoImagen{AdaptadorPoleaNegra} & \completarCon{SeparadorPoleas} & 2 & blah & 20g \\
\hline
9 & \iconoImagen{SoportePlaca} & blah & 1 & 10g \\
\hline
11 & \iconoImagen{UnionBarrasIntermediasA} & blah & 1 & blah & blah \\
\hline
12 & \iconoImagen{UnionBarrasIntermediasB} & blah & 1 & blah & blah \\
\hline
13 & \iconoImagen{RuedaTransmisionSuperior} & blah & 1 & blah & 8g \\
\hline
14 & \iconoImagen{TapaPotenciometro} & blah & 1 & blah & blah \\
\hline
15 & \iconoImagen{TapaPotenciometro} & \completarCon{Engranaje1} & 1 & blah & blah \\
\hline
16 & \iconoImagen{TapaPotenciometro} & \completarCon{Engranaje2} & 1 & blah & blah \\
\hline
17 & \iconoImagen{UnionBarrasSuperiorA} & blah & 1 & blah & blah \\
\hline
18 & \iconoImagen{PoleaColumpioRedir} & blah & 3 & blah & blah \\
\hline
19 & \iconoImagen{CubrePoleaColumpio} & blah & 2 & blah & 6g \\
\hline
20 & \iconoImagen{CubrePoleaColumpioB} & blah & 2 & blah & blah \\
\hline
21 & \iconoImagen{CubrePoleaRedireccionB} & blah & 1 & blah & blah \\
\hline
22 & \iconoImagen{CubrePoleaRedireccion} & blah & 1 & blah & blah \\
\hline
23 & \iconoImagen{PiezaRodamientosSandwich} & blah & 2 & blah & 39g \\
\hline
24 & \iconoImagen{PiezaRodamientosSandwichB} & blah & 1 & blah & 39g \\
\hline
25 & \iconoImagen{PiezaRodamientosSandwichPotenciometro} & blah & 1 & blah & 44g \\
\hline 
26 & \iconoImagen{PiezaRodamientosSandwichPotenciometro} & \completarCon{Falta tapa potenciometro} & 1 & 1 & 2g \\
\hline
27 & \iconoImagen{PiezaMetacrilato} & blah & 2 & blah & - \\
\hline
28 & \iconoImagen{PiezaUnionSandwich} & blah & 4 & blah & 9g \\
\hline
29 & \iconoImagen{RealimentacionSandwich} & blah & 1 & blah & 24g \\
\hline
30 & \iconoImagen{SandwichAcoplamientoRodamientoBarra} & blah & 1 & blah & blah \\
\hline
\end{longtable}

\end{center}


\begin{center}
\begin{table}[H]
    \caption{Parámetros de las piezas para la estimación de peso}
    \label{tab:listadoPiezas:param_impresion}
    \begin{minipage}{\textwidth}
    \begin{tabular}{ |c|c|c|c| }
    \hline
    a & a & a & a \\ 
    %1 & \iconoImagen{Base}{0.2} & blah & \completar \\
    \hline
    \end{tabular}
    \end{minipage}
\end{table}
\end{center}
    	
     \chapter{Montaje del prototipo} \label{app:montajePiezas}
         \hrule
         \vspace{3mm}
%     	%\input{anexos/documentacionG15}
    	
     \chapter{Reglas de codificación del Software} \label{app:codificacionSW}
        \hrule
        \vspace{3mm}
     	Las reglas de codificación aplicadas al software del proyecto se han obtenido, por utilizar una referencia, de las reglas aplicadas por Google en sus proyectos libres. Esta guía está ampliamente documentada e incluye su propia herramienta para comprobar su correcta aplicación \completar lo que facilita la revisión del código así como corrección de desviaciones de estilo.
\\

En este anexo se traducen y resumen los aspectos más importantes de dicha guía. En algunos casos se han adaptado las reglas al caso concreto de este proyecto.
\\

Establecer unas reglas de codificación, unificando un estilo en la notación y uso de la sintaxis, es interesante de cara a posibilitar una mayor facilidad de lectura en futuros desarrollos aumentando así la mantenibilidad del código.

Estas reglas aplican al código C++ del proyecto, no a los \ingles{scripts} auxiliares.

\section{Aspectos generales} \label{sec:codificacionSW:general}

\section{Ficheros de cabecera} \label{sec:codificacionSW:cabeceras}

    En general todos los ficheros con extensión \codigo{.cpp} correspondientes a las librerías deberán ir acompañados del fichero de cabecera \codigo{.h} correspondiente.
    \\
    
    Quedan exentos de cumplir esta regla los ficheros correspondientes a Test (unitarios, de integración, etc) así como ficheros que contengan únicamente una función \codigo{main()}.

\minititulo{Inclusión Múltiple}

    Para evitar problemas de inclusión múltiple todos los ficheros de cabecera con extensión \codigo{.h} deberán incluir guardas con el siguiente formato y escrito en mayúsculas: \codigo{<nombre\_del\_fichero>\_<extensión>. } 
    \\ 
    
    \lstset{language=C, breaklines=true, basicstyle=\footnotesize}
    %Introducir label y caption
    \begin{lstlisting}[frame=single]
Ejemplo:

    #ifndef SERVO_RHA_H
    #define SERVO_RHA_H
    ...
    ...
    ...
    #endif
    \end{lstlisting}
    
\minititulo{Orden de inclusión de ficheros}

    Para evitar problemas en las dependencias de las distintas librerías se incluirán las mismas dejando para el final las librerías propias del proyecto e incluyendo el resto de la más general a la más particular. 
    \\ 
    
    \lstset{language=C, breaklines=true, basicstyle=\footnotesize}
    %Introducir label y caption
    \begin{lstlisting}[frame=single]
Ejemplo orden al incluir cabeceras:

    #include <stdint.h>    // lib estandar de c++
    #include <Arduino.h>     // lib de Arduino
    #include <SoftwareSerial.h>     // lib para controlar el puerto serie. Basado en Arduino
    
    #include "debug.h"     // def y control de las funciones de debug
    #include "rha_types.h"     // tipos de datos
    #include "joint_rha.h"     // clase a incluir
    \end{lstlisting}
    \completarCon{¿def como abreviatura?}
    
    Se deben incluir todos los ficheros que definan los símbolos utilizados en el fichero sobre el que se incluyen. Las declaraciones anticipadas de objetos no están permitidas salvo excepciones justificadas.
 
\section{Ámbitos}\label{sec:codificacionSW:ambitos}
\minititulo{Espacios de nombres}

    Como norma general las constantes, variables o funciones que no estén contenidas en ningún objeto se incluirán dentro de un espacio de nombres o \ingles{namespace} que haga referencia a la utilidad de las mismas.
    \\ 
    
    No está permitido usar directivas del tipo \codigo{ <using namespace \_\_\_\_;>.}
    \\ 
    
    Los espacios de nombres se escriben con la primera letra de cada palabra, en caso de haber más de una, en mayúscula y sin separación de ningún tipo.
    \\ 
    
    \lstset{language=C, breaklines=true, basicstyle=\footnotesize}
    %Introducir label y caption
    \begin{lstlisting}[frame=single]
Ejemplo: Constantes referentes al test de comportamiento ante una entrada tipo rampa

    namespace SlopeTest {
        #define SAMPLE_SLOPE 110
        #define SAMPLE_TEST_SLOPE 20
        #define SLOPE_SPEED 0.1
    }
    
Prohibido el uso de:
    
    using namespace StepTest;
    \end{lstlisting}
    \completarCon{¿def como abreviatura?}
    
\minititulo{Variables Locales}

    Las variables se definirán preferiblemente en el ámbito más local en que se vayan a utilizar. Preferiblemente la inicialización de las mismas se hará junto a la declaración.
    \\ 
    
    Se pueden dar excepciones, como pueden ser vectores sobre los que se iterará dentro de un bucle u otros casos similares. En estos casos se de declarará el objeto fuera del propio ámbito para evitar recursivas llamadas a constructor y destructor de los mismos.
    \\ 
    
    \lstset{language=C, breaklines=true, basicstyle=\footnotesize}
    %Introducir label y caption
    \begin{lstlisting}[frame=single]
Ejemplo: 
    
    // Siempre que la variable sobre la que se itera no se vaya a utilizar para posteriores operaciones: 
    for(int i = 0, i < 10; i++) {
    }
    
    //mejor que el caso siguiente, que adicionalmente incumple la regla preferente de inicializar la variable cuando se declara:
    int i; 
    for(i = 0, i < 10; i++) {
    }
    
    //queda permitido declarar vectores u otros objetos similares antes si se va a iterar o trabajar sobre los mismos
    int vector[5] = {1,2,3,4,5};
    for(int i = 0, i < 10; i++) {
        Serial.print(vector[i]);
    }

    \end{lstlisting}
    \completarCon{¿def como abreviatura?}
    
\section{Clases}\label{sec:codificacionSW:clases}

\minititulo{Constructores y métodos de Inicialización}

Para todos los objetos debe haber constructores por defecto sin parámetros de entrada. Aunque se pueden añadir constructores que inicialicen los diferentes parámetros será obligatorio generar métodos que los inicialicen una vez construido el objeto así como constructores por defecto para todos los métodos. Arduino, aún estando basado en el lenguaje \codigo{C++} no permite un uso completo de memoria dinámica. Los objetos se declaran como miembros haciendo uso del constructor por defecto para ser inicializados posteriormente.
\\ 

    \lstset{language=C, breaklines=true, basicstyle=\footnotesize}
    %Introducir label y caption
    \begin{lstlisting}[frame=single]
Ejemplo: 
    
	//NO se permite:
    - joint_rha.h -
    ServoRHA servo_*;
    - joint_rha.cpp -
    servo_ = new ServoRHA(1, 10, 5);
    
    //Se llama al constructor del objeto para luego inicializarlo:
    - joint_rha.h -
    ServoRHA servo_;
    - joint_rha.cpp -
    servo.init(... params ...);

    \end{lstlisting}

Para evitar funciones con muchos parámetros que reduzcan la legibilidad del código se permite generar diferentes inicializadores para los distintos parámetros. En la documentación del objeto deberá quedar bien claro que inicializadores deben invocarse para el correcto funcionamiento del mismo.

\minititulo{Estructuras o Clases}

Por norma general las estructuras se utilizarán exclusivamente para objetos pasivos, objetos que contienen información. Todo lo demás se codificará dentro una clase.
\\ 

En el caso de estructuras se permiten únicamente métodos para el manejo de los datos sin añadir ninguno tipo de comportamiento, están permitidos los constructores, destructores, métodos de reset, validación, etc. El acceso a los miembros de la estructura se hará directamente sobre los propios parámetros y no mediante métodos específicos. Los parámetros serán siempre públicos para ser consistente con este punto.
\\ 

Para mayores funcionalidades se generará una clase.

\minititulo{Control de Acceso}

Como norma general se declararan como privados todos los atributos de las clases exceptuando aquellos objetos que a su vez tengan, internamente, control de acceso definido (otras clases). De cara a generar Test con clases propias se permite la declaración de atributos como \codigo{protected}.

\section{Tipos de datos}\label{sec:codificacionSW:datos}

Los tipos de datos usados irán acordes con la librería \codigo{stdint.h}. Estos son del tipo \codigo{int16\_t}, \codigo{uint32\_t}, etc. Este tipo de datos garantiza el control del tamaño del dato declarado.

Se utilizarán los nombres \codigo{float} y \codigo{double} convencionales para declarar datos en coma flotante.

\section{Nombres}\label{sec:codificacionSW:nombres}

\minititulo{Reglas generales}

Los nombres deberán ser descriptivos. Por norma general no se utilizarán abreviaciones que no estén comúnmente aceptadas.

\minititulo{Nombre de los ficheros}

    Los nombres de los ficheros de código C++ se nombran en minúsculas separando, en caso de haber varias palabras, con un guión bajo. Los ficheros correspondientes a los test llevarán, precediendo al nombre la palabra "test".
    \\ 
    
    \lstset{language=C, breaklines=true, basicstyle=\footnotesize}
        %Introducir label y caption
        \begin{lstlisting}[frame=single]
Ejemplos:

    joint_handler.h
    joint_rha.cpp
    test_servo_rha.cpp
    \end{lstlisting}

\minititulo{Nombre de los directorios}

    Los ficheros de código irán contenidos en diferentes directorios para cada librería o conjunto de test. Estos directorios llevarán el nombre de la librería que contienen, en el mismo formato que la misma, en este caso sin extensión. Los test se ejecutan en el orden en que se ordenan los directorios. En este caso se añadirá un caracter para ordenar los mismos de manera adecuada. 
    \\ 
    
    Están exentos de esta regla los ficheros principales (que contienen la función \codigo{main()}, ó \codigo{setup()} y \codigo{loop()} en caso de ser ficheros con extensión \codigo{.ino}). 
    \\
    \lstset{language=C, breaklines=true, basicstyle=\footnotesize}
    %Introducir label y caption
    \begin{lstlisting}[frame=single]
Ejemplos:

/lib/
    joint_handler/
    joint_rha/
/test/
    a_test_servo_rha/
    b_test_joint_rha/
    \end{lstlisting}

\minititulo{Nombres para objetos}

Los nombres llevarán mayúscula al comienzo así como al inicio de cada palabra, sin guion bajo como separación.
\\ 

    \lstset{language=C, breaklines=true, basicstyle=\footnotesize}
    %Introducir label y caption
    \begin{lstlisting}[frame=single]
Ejemplo: 
    
    class ServoRHA{ ... };
    class JointHandler{ ... };
    struct SpeedGoal { ... } ;

    \end{lstlisting}
    
\minititulo{Nombres de variables}   

	Por norma general las variables se nombrarán en minúsculas, separando, cuando fuera necesario, las diferentes palabras mediante un guión bajo.
    
\minititulo{Nombres de atributos de clases}
   
La norma para nombrar atributos de clases será igual que en el caso general acabando, en este caso, con un guión bajo.
\\ 

    \lstset{language=C, breaklines=true, basicstyle=\footnotesize}
    %Introducir label y caption
    \begin{lstlisting}[frame=single]
Ejemplo: 
    
    class Regulator {
        float kp_, ki_, kd_;
        float ierror_[INTEGER_INTERVAL];
        uint8_t index_;    
    ... } ;

    \end{lstlisting}
    
\minititulo{Nombres de miembros de estructuras}
 
Las variables miembro de estructuras serán nombradas de igual forma que en el caso general.
\\ 

    \lstset{language=C, breaklines=true, basicstyle=\footnotesize}
    %Introducir label y caption
    \begin{lstlisting}[frame=single]
Ejemplo: 
    
    struct SpeedGoal {
        uint8_t servo_id;
        int16_t speed;
        int16_t speed_slope;
        uint8_t direction;  
    ... } ;

    \end{lstlisting}
    
\minititulo{Nombres de funciones} 

Las funciones comenzarán en minúscula marcando con mayúscula cada nueva palabra que aparezca. Los acrónimos irán en mayúscula. Esta regla afecta a métodos de clases a de igual manera a excepción de constructores y destructores.
\\ 

    \lstset{language=C, breaklines=true, basicstyle=\footnotesize}
    %Introducir label y caption
    \begin{lstlisting}[frame=single]
Ejemplo: 
    
  class ServoRHA {
 	...
 public:
    ServoRHA() { time_last_error_ = 0; time_last_ = 0; last_error_ = 0;
                error_ = 0; derror_ = 0; ierror_ = 0; }
    ServoRHA(uint8_t servo_id);
    void init(uint8_t servo_id);
    void addUpadteInfoToPacket(uint8_t *buffer);
    bool addTorqueToPacket(uint8_t *buffer);
    void setTorqueOnOfToPacket(uint8_t *buffer, uint8_t onOff);

  ... } ;

    \end{lstlisting}
    
\minititulo{Nombres de parámetros funciones} 
	
Los parámetros de métodos y funciones se nombran siguiendo el caso general para nombrar variables.
    
\minititulo{Espacios de nombres} 

Como se ha visto en la sección \ref{sec:codificacionSW:ambitos} los espacios de nombres se definen de manera equivalente a las clases. 

\minititulo{Nombres de enumeraciones} 

En el caso de enumeraciones se seguirá la misma norma que para las clases y espacios de nombres. En este caso cabe la excepción de poder ser declaradas sin nombre.

\minititulo{Nombres de macros} 

Todo nombre precedido por una instrucción \codigo{\#define} se nombrará en mayúsculas, separando las palabras, si las hubiera, mediante el uso del guión bajo. Esto aplica tanto a macros como constantes.

\section{Comentarios}\label{sec:codificacionSW:comentarios}

Es necesario el uso de comentarios para documentar el código y aumentar la legibilidad del mismo. En este caso se seguirá el estilo utilizado por \ingles{doxygen}, que será la herramienta utilizada para, posteriormente generar la documentación. 

\minititulo{Comentarios de ficheros}

Todos los ficheros deberán llevar comentarios en su cabecera. Estos comentarios tendrán el siguiente aspecto:
\\ 

    \lstset{language=C, breaklines=true, basicstyle=\footnotesize}
    %Introducir label y caption
    \begin{lstlisting}[frame=single]
Ejemplo: 
    
/**
 * @file
 * @brief Implements ServoRHA class. This object inherits from CytronG15Servo object to enhance its capabilities
 *
 * @Author: Enrique Heredia Aguado <enheragu>
 * @Date:   2017_Sep_08
 * @Project: RHA
 * @Filename: servo_rha.h
 * @Last modified by:   quique
 * @Last modified time: 30-Sep-2017
 */

    \end{lstlisting}
    
\minititulo{Comentarios de Clases}

\completarCon{¡TDB!}

\minititulo{Comentarios de funciones}

Todas las funciones y métodos deberán llevar un comentario describiendo su funcionamiento así como los parámetros de entrada y salida. Estos comentarios tendrán el siguiente aspecto y se situarán encima de la definición de la función o método:
\\ 

    \lstset{language=C, breaklines=true, basicstyle=\footnotesize}
    %Introducir label y caption
    \begin{lstlisting}[frame=single]
Ejemplo: 
    
 /** @brief Saves in buffer the package return level of servo (error information for each command sent)
   * @method ServoRHA::addReturnOptionToPacket
   * @param {uint8_t*} buffer array in which add the information
   * @param {uint8_t} option RETURN_PACKET_ALL -> servo returns packet for all commands sent; RETURN_PACKET_NONE -> servo never retunrs state packet; RETURN_PACKET_READ_INSTRUCTIONS -> servo answer packet state when a READ command is sent (to read position, temperature, etc)
   * @see addToPacket()
   */

    \end{lstlisting}
    

\minititulo{Comentarios y aclaraciones}
	
    Cuando sea necesario hacer aclaraciones, a nivel de código se harán utilizando el estilo de comentario con doble barra \codigo{//}. Por lo general los nombres de variables y funciones deberán ser de por si descriptivas por lo que este tipo de comentarios se reservan para partes del código especialmente enrevesadas.
    \\ 
    Los comentarios, cuando vayan en línea con el código, se situarán a dos espacios del mismo, dejando un espacio entre el comentario en sí y la doble barra.
    
\minititulo{TODO y notas}

	En algunos casos se podrán dejar cosas para hacer en futuro (TODO) o notas aclaratorias (NOTE). En ambos casos se pondrá en mayúsculas y seguido de dos puntos. Quedando comentados mediante doble barra.
\\ 

    \lstset{language=C, breaklines=true, basicstyle=\footnotesize}
    %Introducir label y caption
    \begin{lstlisting}[frame=single]
Ejemplo: 

    // TODO: complete CW and CCW selection
    // NOTE: important the use of mascares to obtain direction os movement

    \end{lstlisting}
    
\minititulo{Código en desuso}

En algunas situaciones hay fragmentos de código que ya no se utilizan o están temporalmente deshabilitados. Estos fragmentos serán comentados mediante barra y asterisco :
\\ 

    \lstset{language=C, breaklines=true, basicstyle=\footnotesize}
    %Introducir label y caption
    \begin{lstlisting}[frame=single]
Ejemplo: 

	/* ... 
    ... some code ...
    ... */

    \end{lstlisting}
    
\section{Formato}\label{sec:codificacionSW:formato}

\minititulo{Espacios y tabulaciones}

Por norma general se utilizará cuatro espacios como indentación para distintos ámbitos. 
\\ 

    \lstset{language=C, breaklines=true, basicstyle=\footnotesize}
    %Introducir label y caption
    \begin{lstlisting}[frame=single]
Ejemplo: 

void ServoRHA::setWheelSpeedToPacket( ... ) {
    ...    
    if ( ... ) {
        ...
    }
    ...  
}

    \end{lstlisting}
    
\minititulo{Declaración y definición de funciones}

El valor de retorno así como los parámetros de una función deberán ir en la misma línea. En caso de no caber o para mayor claridad se pondrán a la misma altura que los anteriores.
\\ 

    \lstset{language=C, breaklines=true, basicstyle=\footnotesize}
    %Introducir label y caption
    \begin{lstlisting}[frame=single]
Ejemplo: 

    void ServoRHA::setWheelSpeedToPacket(uint8_t *buffer, uint16_t speed, uint8_t direction) {
	...
    }

    void ServoRHA::setWheelSpeedToPacket(uint8_t *buffer, uint16_t speed,
                                         uint8_t direction) {
	...
    }

    \end{lstlisting}
    
    
\minititulo{Condicionales}

Como norma general no se dejarán espacios entre los paréntesis. Si se dejará un espacio entre la sentencia \codigo{if} y el condicional, así como entre este último y la llave que abre el ámbito condicional.
\\ 

    \lstset{language=C, breaklines=true, basicstyle=\footnotesize}
    %Introducir label y caption
    \begin{lstlisting}[frame=single]
Ejemplo: 
//Forma correcta:
    if (direction == CW) {
        speed = speed | 0x0400;
    }
    
//Ejemplos incorrectos:
    if(direction == CW) {  // Falta un espacio tras la sentencia if
    if (direction == CW){  // Falta un espacio entre el condicional y la llave
    if(direction == CW){  // Combina los casos anteriores 

    \end{lstlisting}
    
    En caso de que el condicional afecte solo a una sentencia esta se pondrá, como norma general, sin llaves y en la misma línea que el condicional. De igual forma se hará tras sentencias de tipo \codigo{else} o combinando \codigo{else if}. En caso de utilizar llaves se seguirá la norma que aplica a dicho caso.
\\ 

    \lstset{language=C, breaklines=true, basicstyle=\footnotesize}
    %Introducir label y caption
    \begin{lstlisting}[frame=single]
Ejemplo: 

    if (speed1 < speed2-speed_margin) return ServoRHAConstants::LESS_THAN;
    else if (speed1 > speed2+speed_margin) return ServoRHAConstants::GREATER_THAN;
    else return ServoRHAConstants::EQUAL;

    \end{lstlisting}
    
    Cuando si afecta a diferentes líneas y hay sentencias de tipo \codigo{else}, estas irán en la misma línea de cierre de llave del condicional (siempre que no afecte a la legibilidad del código ya sea por presencia de comentarios u otras causas similares).
    \\ 

    \lstset{language=C, breaklines=true, basicstyle=\footnotesize}
    %Introducir label y caption
    \begin{lstlisting}[frame=single]
Ejemplo: 

    if (...) {
            ...
        } else {
            ...
        }

    \end{lstlisting}
    
\minititulo{Bucles}

El formato será equivalente al caso de los condicionales:
\\ 

    \lstset{language=C, breaklines=true, basicstyle=\footnotesize}
    %Introducir label y caption
    \begin{lstlisting}[frame=single]
Ejemplo: 

    for (...) {
       ...
    }
    for (...) oneLineStatement;
    while (condition) {
        ...
    }

    \end{lstlisting}  
    
    
\minititulo{Valor de retorno de funciones y métodos}

No es necesario utilizar paréntesis para rodear la expresión a retornar. Solo se utilizarán en los casos en que se utilizarían si se fuera a asignar dicha expresión a una variable.

\minititulo{Formato para clases}

Las directivas \codigo{public}, \codigo{protected} y \codigo{private} irán indentados un espacio respecto a la definición de la clase. Por norma general irán precedidos por una línea en blanco (excepto cuando las preceda la definición de la propia clase).
\\ 

    \lstset{language=C, breaklines=true, basicstyle=\footnotesize}
    %Introducir label y caption
    \begin{lstlisting}[frame=single]
Ejemplo: 


class JointHandler {
 private:  // un espacio
    ...
 public:
    ...

    \end{lstlisting} 
    
\minititulo{Espacios de nombre}

Los espacios de nombres siguen la norma general para indentar diferentes ámbitos.

\section{Espacios en blanco}\label{sec:codificacionSW:espacios_blanco}

Los espacios horizontales dependerán de cada caso. En ningún caso se finalizará una línea con un espacio en blanco.

\minititulo{Caso general}

    \lstset{language=C, breaklines=true, basicstyle=\footnotesize}
    %Introducir label y caption
    \begin{lstlisting}[frame=single]
Ejemplo: 

    void JointHandler::setTimer(uint64_t timer) {  // Un espacio entre el cierre del parentesis y la apertura de llaves
    class TimerMicroseconds : public Timer {  // Espacio entre los dos puntos en casos de herencia o inicializadores dentro de constructores. Se pone un espacio a cada lado.
    void checkWait()  // No se deja espacio entre el nombre y los parentesis. Tampoco entre parentesis vacios.
    float getError() { return error_; }  // Se deja espacio entre llaves e implementacion, a ambos lados.

    \end{lstlisting}

\minititulo{Bucles, condicionales y estructuras de control}

    \lstset{language=C, breaklines=true, basicstyle=\footnotesize}
    %Introducir label y caption
    \begin{lstlisting}[frame=single]
Ejemplo: 

    if (b) {  // Espacio entre la sentencia if y la condicion, asi como esta misma con la apertura de llaves
    } else {  // Espacios al rededor de la sentencia else
    }
    switch (i) {
        case 1:  // No se deja espacio antes de los dos puntos
        ...
        case 2: break;  // Si se deja despues de los mismos
    for (int i = 0 ; i < 5 ; i++) {  // En caso de bucles for, ademas de los espacios al rededor de los parentesis se dejara un espacio tras cada punto y coma.
    \end{lstlisting}
    
\minititulo{Operadores}

    \lstset{language=C, breaklines=true, basicstyle=\footnotesize}
    %Introducir label y caption
    \begin{lstlisting}[frame=single]
Ejemplo: 

    // En general se deja un espacio al rededor de los distintos tipos de operadores
    x = 0;
    v = w * x + y / z;
    v = w*x + y/z;
    v = w * (x + z);
    
    // No se separan operadores unarios de sus argumentos:
    x = -5;
    x++;
    if (x && !y)

    \end{lstlisting}
    
\completarCon{¿Espacios verticales y horizontales deberian ir dentro de la seccion de formato no?}

\section{Espacio vertical}

Por lo general se dejaran espacios verticales para una mayor claridad del código sin abusar de los mismos. Aunque separar diferentes partes puede ayudar demasiados espacios verticales pueden dificultar la lectura de código.

\completarCon{Hablar de como se definen las variables, cuando se pueden poner varias en la misma línea y demás}


\completarCon{A bibliografía -->} https://google.github.io/styleguide/cppguide.html
    
     \chapter{Documentación del software} \label{app:documentacion_software}
         \hrule
         \vspace{3mm}
     	 %\include{doxygen_documentation/refman}
%\input{doxygen_documentation/refman}


     	 
     \chapter{Registros Servos G15} \label{app:registros_g15}
	     \hrule
	     \vspace{3mm}
	         En este anexo se describe como se trabaja con el protocolo de comunicación a bajo nivel para codificar el paso de mensajes entre los servos G15 Cube y el microcontrolador. 
    \\
    
    Antes de describir el formato de la información cabe destacar que en todo momento la información enviada irá codificada en formato hexadecimal, para los paquetes enviados como recibidos desde el microcontrolador. Todos los paquetes, tanto los enviados a los servos como la respuesta por parte de los mismos tendrán en común la siguiente información:
    
    \begin{itemize}
    	\item Encabezado: Los primeros dos bytes del mensaje estarán compuestos por encabezado que será el que marque el inicio del mensaje. Estos bytes serán: 0xFF 0xFF
    	\item Un fin de mensaje: El último byte del mensaje estará marcado por un valor llamado \ingles{CheckSum} que será el encargado de verificar que todo el paquete ha llegado correctamente. El \ingles{CheckSum} es el inverso del valor binario de la suma de todos los bytes enviados a excepción del encabezado y el propio \ingles{CheckSum}. En la figura \completar se puede ver un ejemplo de como se calcula dicho valor.
    \end{itemize}
    
    \completarCon{Ejemplo de como se calcula el checksum}
    
    En los paquetes que se envíen a los servos la información se codificará de la siguiente manera concreta:
    
    \begin{itemize}
    	\item Bytes 0 y 1 para el encabezado.
    	\item Byte 2 codifica el ID del servo al que se quiere enviar la acción. De forma general se puede utilizar la dirección 0xFE (en hexadecimal) para enviar el mensaje a todos los servos conectados.
    	\item Byte 2 codifica la longitud del mensaje. Contando a partir del encabezado y el ID (excluyendo ambos) el número de bytes que se envían. De esta forma, al recibirse el paquete se podrá identificar el inicio del mismo, a que servo va dirigido (el resto ignorarán el paquete) y cuantos bytes tendrá que leer el aludido. Por supuesto el último byte de la cadena será el ya mencionado \ingles{CheckSum} cuyo valor tendrá que coincidir con el esperado al analizar la cadena.
    	\item Byte 3 codifica la instrucción que se desea realizar. Sobre la memoria de los servos se podrán hacer operaciones de lectura y escritura de distinta manera. Se pueden ver las diferentes instrucciones posibles en la tabla \ref{tab:g15_instructions}. Serán explicadas posteriormente más en detalle.
    	\item Bytes del 4 al N: Parámetros que se quieran enviar al servo.
    	\item Byte N+1: \ingles{CheckSum}
    \end{itemize}
    
    En la figura \ref{fig:app:registrosg15:comunicacion_mensaje} se puede ver representado, a modo de resumen gráfico, este esquema de información genérico.	
    
    \begin{figure}[H]
    	\centering
    	\includegraphics[width=0.9\textwidth]{figuras/Imagenes_SW/Packet_G15.png}   
    	\caption{Paquete de información genérico para comunicar con los Servos G15 Cube}
    	\label{fig:app:registrosg15:comunicacion_mensaje}
    \end{figure}
	

	\begin{table}[htbp]
		\centering
		\caption{Resumen de las instrucciones aceptadas por los Cytron G15 Cube servo}
		\label{tab:g15_instructions}
		\begin{center}
			\begin{tabular}{|c|c|c|}
			\hline
			\textbf{Instrucción} & \textbf{Valor Hex.} & \textbf{Comentarios} \\
			\hline
			iPING & 0x01 & Solicita un paquete con el estado del servo \\
			\hline
			iREAD\_DATA & 0x02 & Lee información de la memoria del servo \\
			\hline
			iWRITE\_DATA & 0x03 & Escribe información en la memoria del servo \\
			\hline
			iREG\_WRITE & 0x04 & Escribe sobre la memoria y hasta que llega la acción \ingles{ACTION} para ejecutar dichos cambios \\
			\hline
			iACTION & 0x05 & Activa la acción codificada con la instrucción \ingles{REG\_WRITE} \\
			\hline
			iRESET & 0x06 & Resetea la memoria a los valores por defecto \\
			\hline
			iSYNC\_WRITE & 0x83 & Para escribir simultáneamente información sobre varios servos \\
			\hline
			\end{tabular}
		\end{center}
	\end{table}	
	
	La respuesta por parte de los servos tiene también una estructura general que se detalla a continuación byte a byte:
	\begin{itemize}
		\item Bytes 0 y 1 de encabezado. Igual que en el caso anterior.
		\item Byte 2 codifica el ID del servo que responde.
		\item Byte 3 codifica la longitud a leer.
		\item Byte 4 sirve para informar de posibles errores en el servo. Cada bit del byte codifica un tipo de error, estando todos a 0 cuando la comunicación y el servo se encuentran buen estado. Estos errores están detallados en la tabla \ref{tab:g15_error}, junto a la máscara en binario que se aplicará a dicho byte para comprobar cada error.
		\item Bytes del 4 al N: Parámetros que envía el servo.
		\item Byte N+1: \ingles{CheckSum}
	\end{itemize}
	
	 En la figura \ref{fig:app:registrosg15:comunicacion_mensaje_from_servo} se puede ver representado, a modo de resumen gráfico, este esquema de información genérico que se ha expuesto previamente.	
	 
	 \begin{figure}[H]
	    	\centering
	    	\includegraphics[width=0.9\textwidth]{figuras/Imagenes_SW/Packet_From_G15.png}   
	    	\caption{Paquete de información genérico de retorno de los Servos G15 Cube}
	    	\label{fig:app:registrosg15:comunicacion_mensaje_from_servo}
	 \end{figure}
	
	\begin{table}[htbp]
		\centering
		\caption{Codificación del error de los servos G15 Cube en cada bit del byte de error.}
		\label{tab:g15_error}
		\begin{center}
			\begin{tabular}{|c|c|c|}
				\hline
				\textbf{Bit} & \textbf{Error} & \textbf{Máscara a aplicar} \\
				\hline
				0 & Error en voltaje de entrada & 0X0001 \\
				\hline
				1 & Límite de ángulo & 0X0002 \\
				\hline
				2 & Sobrecalentamiento & 0X0004 \\
				\hline
				3 & Error en el rango pedido & 0X0008 \\
				\hline
				4 & Error en el \ingles{CheckSum} & 0X0010 \\
				\hline
				5 & Sobrecarga & 0X0020 \\
				\hline
				6 & Instrucción incorrecta & 0X0040 \\
				\hline
				7 & - & -  \\
				\hline
			\end{tabular}
		\end{center}
	\end{table}
	
	Aunque como se ha visto anteriormente y se ha detallado en la tabla \ref{tab:g15_error} el servo devuelve un solo byte de error, al leer la información recibida el error, tanto en la librería de Cytron como en la estructura desarrollada para este proyecto este byte se amplia a dos bytes para añadir la posibilidad de nuevos errores en la recepción del paquete de datos. Se pueden ver de forma detallada en la tabla \ref{tab:g15_error_second}, nuevamente junto a la máscara que se aplicará a dicho byte para cada caso. En el byte más bajo queda la información que devuelve el servo y en el más alto la información añadida.
	\\ 
	
	\begin{table}[htbp]
		\centering
		\caption{Codificación del error de comunicación en cada bit del segundo byte de error.}
		\label{tab:g15_error_second}
		\begin{center}
			\begin{tabular}{|c|c|c|}
				\hline
				\textbf{Bit} & \textbf{Error} & \textbf{Máscara a aplicar} \\
				\hline
				8 & Paquete perdido o tiempo de espera superado & 0X0100 \\
				\hline
				9 & Encabezado incorrecto & 0X0200 \\
				\hline
				10 & ID incorrecto & 0X0400 \\
				\hline
				11 & Error en el \ingles{CheckSum} & 0X0800 \\
				\hline
				12 & - & -  \\
				\hline
				13 & - & -  \\
				\hline
				14 & - & -  \\
				\hline
				15 & - & -  \\
				\hline
			\end{tabular}
		\end{center}
	\end{table}
	
	A continuación se detallan las distintas instrucciones que aceptan los Servos, presentadas en la tabla \ref{tab:g15_instructions}.
	\subsection{Petición del estado del servo}
	
	\subsection{Operaciones de lectura}
	
	\subsection{Operaciones de escritura}
	
	\subsection{Operaciones de escritura con activación desacoplada}
	
	\subsection{Operaciones de escritura sobre múltiples servos}


\begin{table}[htbp]
	\caption{Características mas importantes de los Servos G15 de Cytron. Tabla traducida y resumida a los puntos más importantes del Cytron G15 Cube servo User Manual \cite{CytronTechnologies2012} \completarCon{¿esto está bien, hay que poner paginas involucradas?}}
	\label{tab:g15_register}
	\begin{adjustwidth}{-1.9cm}{-1.5cm}
	\begin{tabular}{|c|c|c|c|c|c|c|}
		\hline
		\textbf{Area} & \textbf{\shortstack{Address\\ (Hex)}} & \textbf{Parameter} & \textbf{\shortstack{Read only \\ /Read Write}} & \textbf{\shortstack{Factory\\ default\\ value (Hex)}} & \textbf{\shortstack{Minimum\\ value (Hex)}} & \textbf{\shortstack{Maximum\\ value (Hex)}} \\ \hline
		\multicolumn{ 1}{|c|}{EEPROM} & 0 (0x00) & Model (L) & R & ‘G’ (0x0F) & - & - \\ \cline{ 2- 7}
		\multicolumn{ 1}{|c|}{} & 1 (0x01) & Model(H) & R & 15 (0x47) & - & - \\ \cline{ 2- 7}
		\multicolumn{ 1}{|c|}{} & 2 (0x02) & Firmware Revision & R &  & - & - \\ \cline{ 2- 7}
		\multicolumn{ 1}{|c|}{} & 3 (0x03) & ID & RW & 1 (0x01) & 0 (0x00) & 253 (0xFD) \\ \cline{ 2- 7}
		\multicolumn{ 1}{|c|}{} & 4 (0x04) & Baud Rate & RW & 103 (0x67) & 3 (0x03) & 255 (0xFF) \\ \cline{ 2- 7}
		\multicolumn{ 1}{|c|}{} & 5 (0x05) & Return Delay & RW & 250 (0xFA)  & 1 (0x01)  & 255 (0xFF) \\ \cline{ 2- 7}
		\multicolumn{ 1}{|c|}{} & 6 (0x06) & CW Angle Limit (L) & RW & \multicolumn{ 1}{c|}{ 0 (0x0000) } & \multicolumn{ 1}{c|}{ 0 (0x0000) } & \multicolumn{ 1}{c|}{1087 (0x043F)} \\ \cline{ 2- 4}
		\multicolumn{ 1}{|c|}{} & 7 (0x07) & CW Angle Limit (H) & RW & \multicolumn{ 1}{c|}{} & \multicolumn{ 1}{c|}{} & \multicolumn{ 1}{c|}{} \\ \cline{ 2- 7}
		\multicolumn{ 1}{|c|}{} & 8 (0x08) & CCW Angle Limit (L) & RW & \multicolumn{ 1}{c|}{1087 (0x043F)} & \multicolumn{ 1}{c|}{ 0 (0x0000) } & \multicolumn{ 1}{c|}{1087 (0x043F)} \\ \cline{ 2- 4}
		\multicolumn{ 1}{|c|}{} & 9 (0x09) & CCW Angle Limit (H) & RW & \multicolumn{ 1}{c|}{} & \multicolumn{ 1}{c|}{} & \multicolumn{ 1}{c|}{} \\ \cline{ 2- 7}
		\multicolumn{ 1}{|c|}{} & 10 (0x0A) & Reserved & - & - & - & - \\ \cline{ 2- 7}
		\multicolumn{ 1}{|c|}{} & 11 (0x0B) & Temperature Limit & RW & 70 (0x46) & 0 (0x00) &  \\ \cline{ 2- 7}
		\multicolumn{ 1}{|c|}{} & 12 (0x0C) & Lowest Voltage Limit & RW & 65 (0x41) & 65 (0x41) & 178 (0xB2) \\ \cline{ 2- 7}
		\multicolumn{ 1}{|c|}{} & 13 (0x0D) & Highest Voltage Limit & RW & 150 (0x96)  &  &  \\ \cline{ 2- 7}
		\multicolumn{ 1}{|c|}{} & 14 (0x0E) & Max Torque (L) & RW & \multicolumn{ 1}{c|}{1023 (0x03FF)} & \multicolumn{ 1}{c|}{ 0 (0x0000) } & \multicolumn{ 1}{c|}{1023 (0x03FF)} \\ \cline{ 2- 4}
		\multicolumn{ 1}{|c|}{} & 15 (0x0F) & Max Torque (H) & RW & \multicolumn{ 1}{c|}{} & \multicolumn{ 1}{c|}{} & \multicolumn{ 1}{c|}{} \\ \cline{ 2- 7}
		\multicolumn{ 1}{|c|}{} & 16 (0x10) & Return Packet Enable & RW & 2 (0x02) & 0 (0x00) & 2 (0x02) \\ \cline{ 2- 7}
		\multicolumn{ 1}{|c|}{} & 17 (0x11) & Alarm LED & RW & 36 (0x24) & 0 (0x00) & 127 (0x7F) \\ \cline{ 2- 7}
		\multicolumn{ 1}{|c|}{} & 18 (0x12) & Alarm Shutdown & RW & 36 (0x24) & 0 (0x00) & 127 (0x7F) \\ \cline{ 2- 7}
		\multicolumn{ 1}{|c|}{} & 19 (0x13) & Reserved & - & - & - & - \\ \cline{ 2- 7}
		\multicolumn{ 1}{|c|}{} & 20 (0x14) & Down Calibration (L) & R &  &  &  \\ \cline{ 2- 7}
		\multicolumn{ 1}{|c|}{} & 21 (0x15) & Down Calibration (H) & R &  &  &  \\ \cline{ 2- 7}
		\multicolumn{ 1}{|c|}{} & 22 (0x16) & Up Calibration (L) & R &  &  &  \\ \cline{ 2- 7}
		\multicolumn{ 1}{|c|}{} & 23 (0x17) & Up Calibration (H) & R &  &  &  \\ \hline
		\multicolumn{ 1}{|c|}{RAM} & 24 (0x18) & Torque Enable & RW & 0 (0x00) & 0 (0x00) & 1 (0x01) \\ \cline{ 2- 7}
		\multicolumn{ 1}{|c|}{} & 25 (0x19) & LED & RW & 0 (0x00) & 0 (0x00) & 1 (0x01) \\ \cline{ 2- 7}
		\multicolumn{ 1}{|c|}{} & 26 (0x1A) & CW Compliance Margin & RW & 1 (0x01) & 0 (0x00) & 254(0xFE) \\ \cline{ 2- 7}
		\multicolumn{ 1}{|c|}{} & 27 (0x1B) & CCW Compliance & RW & 1 (0x01) & 0 (0x00) & 254(0xFE) \\ \cline{ 2- 7}
		\multicolumn{ 1}{|c|}{} & 28 (0x1C) & CW Compliance Slope & RW & 32 (0x0020) & 1 (0x01) & 254(0xFE) \\ \cline{ 2- 7}
		\multicolumn{ 1}{|c|}{} & 29 (0x1D) & CCW Compliance Slope & RW & 32 (0x0020) & 1 (0x01) & 254(0xFE) \\ \cline{ 2- 7}
		\multicolumn{ 1}{|c|}{} & 30 (0x1E) & Goal Position (L) & RW & Address 36 & \multicolumn{ 1}{c|}{ 0 (0x0000) } & \multicolumn{ 1}{c|}{1087 (0x043F)} \\ \cline{ 2- 5}
		\multicolumn{ 1}{|c|}{} & 31 (0x1F) & Goal Position (H) & RW & Address 37 & \multicolumn{ 1}{c|}{} & \multicolumn{ 1}{c|}{} \\ \cline{ 2- 7}
		\multicolumn{ 1}{|c|}{} & 32 (0x020) & Moving Speed (L) & RW & \multicolumn{ 1}{c|}{ 0 (0x0000) } & \multicolumn{ 1}{c|}{ 0 (0x0000) } & \multicolumn{ 1}{c|}{1023 (0x03FF)} \\ \cline{ 2- 4}
		\multicolumn{ 1}{|c|}{} & 33 (0x21) & Moving Speed (H) & RW & \multicolumn{ 1}{c|}{} & \multicolumn{ 1}{c|}{} & \multicolumn{ 1}{c|}{} \\ \cline{ 2- 7}
		\multicolumn{ 1}{|c|}{} & 34 (0x22) & Torque Limit (L) & RW & Address 14 & \multicolumn{ 1}{c|}{ 0 (0x0000) } & \multicolumn{ 1}{c|}{1023 (0x03FF)} \\ \cline{ 2- 5}
		\multicolumn{ 1}{|c|}{} & 35 (0x23) & Torque Limit (H) & RW & Address 15 & \multicolumn{ 1}{c|}{} & \multicolumn{ 1}{c|}{} \\ \cline{ 2- 7}
		\multicolumn{ 1}{|c|}{} & 36 (0x24) & Present Position (L) & R & \multicolumn{ 1}{c|}{} & \multicolumn{ 1}{c|}{} & \multicolumn{ 1}{c|}{} \\ \cline{ 2- 4}
		\multicolumn{ 1}{|c|}{} & 37 (0x25) & Present Position (H) & R & \multicolumn{ 1}{c|}{} & \multicolumn{ 1}{c|}{} & \multicolumn{ 1}{c|}{} \\ \cline{ 2- 7}
		\multicolumn{ 1}{|c|}{} & 38 (0x26) & Present Speed (L) & R & \multicolumn{ 1}{c|}{} & \multicolumn{ 1}{c|}{} & \multicolumn{ 1}{c|}{} \\ \cline{ 2- 4}
		\multicolumn{ 1}{|c|}{} & 39 (0x27) & Present Speed (H) & R & \multicolumn{ 1}{c|}{} & \multicolumn{ 1}{c|}{} & \multicolumn{ 1}{c|}{} \\ \cline{ 2- 7}
		\multicolumn{ 1}{|c|}{} & 40 (0x28) & Present Load (L) & R & \multicolumn{ 1}{c|}{} & \multicolumn{ 1}{c|}{} & \multicolumn{ 1}{c|}{} \\ \cline{ 2- 4}
		\multicolumn{ 1}{|c|}{} & 41 (0x29) & Present Load (H) & R & \multicolumn{ 1}{c|}{} & \multicolumn{ 1}{c|}{} & \multicolumn{ 1}{c|}{} \\ \cline{ 2- 7}
		\multicolumn{ 1}{|c|}{} & 42 (0x2A) & Present Voltage & R &  &  &  \\ \cline{ 2- 7}
		\multicolumn{ 1}{|c|}{} & 43 (0x2B) & Present Temperature & R &  &  &  \\ \cline{ 2- 7}
		\multicolumn{ 1}{|c|}{} & 44 (0x2C) & Registered & R & 0 (0x00) & 0 (0x00) & 1 (0x01) \\ \cline{ 2- 7}
		\multicolumn{ 1}{|c|}{} & 45 (0x2D) & Reserved & - & - & - & - \\ \cline{ 2- 7}
		\multicolumn{ 1}{|c|}{} & 46 (0x2E) & Moving & R & 0 (0x00) & 0 (0x00) & 1 (0x01) \\ \cline{ 2- 7}
		\multicolumn{ 1}{|c|}{} & 47 (0x2F) & Lock & RW & 0 (0x00) & 1 (0x01) & 1 (0x01) \\ \cline{ 2- 7}
		\multicolumn{ 1}{|c|}{} & 48 (0x30) & Punch (L) & RW & \multicolumn{ 1}{c|}{32 (0x0020)} & \multicolumn{ 1}{c|}{0 (0x0000)} & \multicolumn{ 1}{c|}{1023 (0x03FF)} \\ \cline{ 2- 4}
		\multicolumn{ 1}{|c|}{} & 49 (0x31) & Punch (H) & RW & \multicolumn{ 1}{c|}{} & \multicolumn{ 1}{c|}{} & \multicolumn{ 1}{c|}{} \\ \hline
	\end{tabular}
\end{adjustwidth}
\end{table}	


        
%\end{appendices}


 \backmatter

% %estilo de bibliografía: plana, alfa...
 \bibliographystyle{ieeetr}

% %genera doble hoja en blanco
 \cleardoublepage

 %apartado de bibliografía
 \addcontentsline{toc}{chapter}{Bibliografía}

 %se incluye la bibliografía. Archivo de tipo .bib (bibtex)
 \bibliography{bibliografia/bibliografia}

%fin del documento
\end{document}
