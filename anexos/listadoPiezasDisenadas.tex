
En este anexo se presenta una tabla listando las piezas que se han diseñado y fabricado para el proyecto. En la tabla \ref{app:listadoPiezas}1 se puede ver una miniatura de la pieza en cuestión, la cantidad necesaria de cada tipo, una estimación del peso (material consumido en su fabricación) así como una breve descripción de la pieza y/o proceso de fabricación de la misma. A su vez llevan asociado una referencia alfanumérica que se corresponde con los ficheros en formato digital entregados, asignada de la siguiente forma:

\begin{center} 
	\textit{RHA + ubicación + número\_de\_pieza}
\end{center}

\begin{itemize}
	\item \textit{ubicación}: A\$ (articulación), B\$ (barra). La letra va acompañada de un número (sustituyendo al carácter \$) que variará dependiendo de donde se encuentre la pieza en el montaje. A1 - articulación uno, B2 - barra 2 y así sucesivamente.
	\item \textit{número\_de\_pieza}: valor numérico que diferencia las piezas en la misma ubicación.
\end{itemize}

Nota: En caso de que la pieza se utilice en varias partes diferenciadas la referencia se tomará para la primera vez que aparece la pieza en orden ascendente (desde la barra 0 en adelante e igual desde la articulación 1).

Nota 2: la miniatura de las piezas no sigue ninguna escala concreta. Es una representación que permite un reconocimiento visual de la pieza, aunque no de su tamaño.


\begin{adjustwidth}{-10cm}{-10cm}
\begin{landscape}
\begin{center}
\begin{longtable}{|c|c|c|c|c|}
\caption{Listado de piezas diseñadas de fabricación propia}\\
\hline
\textbf{Num} & \textbf{Esquema Pieza} & \textbf{Referencia} & \textbf{Cantidad} & \textbf{Descripción}  \\
\hline
\endfirsthead
\multicolumn{5}{c}%
{\raggedright \tablename\ \thetable\ -- \textit{Continuación de la página anterior}} \\
\hline
\textbf{Num} & \textbf{Esquema Pieza} & \textbf{Referencia} & \textbf{Cantidad} & \textbf{Descripción}  \\
\hline
\endhead
\hline \multicolumn{5}{r}{ \raggedright \textit{Continua en la página siguiente} } \\
\endfoot
\hline
%\insertTableNotes
\endlastfoot
1 & \iconoImagen{Base} & RHAB1001 & 1  & \pieceDescription{ Es la base sobre la que se colocan los servos de los tres primeros grados de libertad. También fija la rueda para efectuar el giro en Z}   \\
\hline
2 & \iconoImagen{RailA} & RHAB1002 & 1 & \pieceDescription{ Funciona a modo de fijación de la torre de motores. Además encapsula la polea sobre la que se enrolla el hilo, accionada por el servo. Supone un apoyo para el eje motriz, de forma que absorbe parte de las fuerzas de tracción a las que es sometido.}  \\
\hline
3 & \iconoImagen{RailB} & RHAB1003 & 1 & \pieceDescription{ Su función es equivalente a la de la pieza RHAB1002} \\
\hline
4 & \iconoImagen{PoleaMotor} & RHAB1004 & 2 & \pieceDescription{ Polea motriz sobre la que se enrolla/desenrolla el hilo para accionar el brazo robótico} \\
\hline
5 & \iconoImagen{EncajeTuboInterior} & RHAA1005 & 1 & \pieceDescription{ Es la pieza móvil de la primera articulación.} \\
\hline
6 & \iconoImagen{EncajeTuboExterior} & RHAA1006 & 1 & \pieceDescription{ Pieza fija de la primera articulación y base sobre la que rota el brazo robótico.} \\
\hline
7 & \iconoImagen{FijacionBarraPieB} & RHAB0001 & 1 & \pieceDescription{ Asegura el ajuste entre la barra de acero que soporta el brazo y el soporte elegido.} \\
\hline
8 & \iconoImagen{FijacionBarraPie} & RHAB0002 & 1 & \pieceDescription{ Acompaña a la pieza RHAB0001 en su función.} \\
\hline
10 & \iconoImagen{RuedaMotorGiroZ} & RHAA1001 & 1 & \pieceDescription{ Transmite el giro del motor a la rueda de transmisión para accionar la primera articulación.} \\
\hline
19 & \iconoImagen{AdaptadorPoleaNegra} & RHAB1007 & 2 & \pieceDescription{ Asegura el ajuste entre las poleas de redirección y el eje utilizado} \\
\hline
19 & \iconoImagen{SeparadorPoleaNegraGrande} & RHAB1008 & 2 & \pieceDescription{ Soporta las poleas de redirección encapsulando y protegiendo el cable de salirse de las mismas. Además sirve de anclaje para la vuelta del cable de la segunda articulación.} \\
\hline
9 & \iconoImagen{SoportePlaca} & RHAB1009 & 1 & \pieceDescription{ Soporte para acoplar la electrónica a los servos (Placa Arduino y shield)} \\
\hline
11 & \iconoImagen{UnionBarrasIntermediasA} & RHAA2001 & 1 & \pieceDescription{ Encapsula la polea que redirige el cable hacia la tercera articulación.} \\
\hline
12 & \iconoImagen{UnionBarrasIntermediasB} & RHAA2002 & 1 & \pieceDescription{ Integra el potenciómetro encargado de realimentar la segunda articulación próximo al eje de giro de la misma.}  \\
\hline
13 & \iconoImagen{RuedaTransmisionSuperior} & RHAA2003 & 1 & \pieceDescription{ Polea para redirigir el movimiento hacia la tercera articulación a su paso por el eje de giro de la segunda.}  \\
\hline
14 & \iconoImagen{TapaPotenciometro} & RHAA2004 & 1 & \pieceDescription{ Tapa para el potenciómetro ubicado en la segunda articulación}  \\
\hline
15 & \iconoImagen{EngranajePotenciometro} & RHAA2005 & 1 & \pieceDescription{ Engranaje para transmitir el giro de la articulación al potenciómetro. Gira solidario al potenciómetro.}  \\
\hline
16 & \iconoImagen{EngranajeBarra} & RHAA2006 & 1 & \pieceDescription{ Hace pareja con la pieza RHAA2005 encajándose en la barra y girando solidaria a la misma.}  \\
\hline
17 & \iconoImagen{UnionBarrasSuperiorA} & RHAB2001 & 1 & \pieceDescription{ Cierra las barras verticales en el extremo superior. Guía uno de los ejes de giro de la segunda articulación asegurando que la barra que gira en el mismo se mantenga en la posición adecuada.}  \\
\hline
17 & \iconoImagen{UnionBarrasSuperiorB} & RHAB2002 & 1 & \pieceDescription{ Complementa la pieza RHAB2001. }  \\
\hline
18 & \iconoImagen{PoleaColumpioRedir} & RHAB2003 & 3 & \pieceDescription{ Polea para los \textit{polipastos} correspondientes a la segunda y tercera articulación. También se encuentra redirigiendo el cable hacia la tercera articulación en la etapa final.}  \\
\hline
19 & \iconoImagen{CubrePoleaColumpio} & RHAB2004 & 2 & \pieceDescription{ Asegura la unión y giro de la polea RHAB2003 sobre la barra correspondiente.} \\
\hline
20 & \iconoImagen{CubrePoleaColumpioB} & RHAB2005 & 2 & \pieceDescription{ Acompaña a la pieza RHAB2004 abrazando la barra de sección cuadrada y fijando el montaje a la misma.}  \\
\hline
22 & \iconoImagen{CubrePoleaRedireccion} & RHAB3002 & 1 & \pieceDescription{ Cubre y fija la polea de redirección hacia la tercera articulación justo antes de la misma.}  \\
\hline
21 & \iconoImagen{CubrePoleaRedireccionB} & RHAB3001 & 1 & \pieceDescription{ Encapsula la polea de redirección junto a la pieza RHAB3002. Permite la fijación del montaje a la barra cuadrada. También sirve de punto de encaje para el extremo fijo del hilo.}  \\
\hline
23 & \iconoImagen{PiezaRodamientosSandwich} & RHAA3001 & 2 & \pieceDescription{ Encaja sobre la pieza RHAA3005 quedando fijas a través de los rodamientos sobre los que se apoyarán los ejes de giro. Estas piezas componen la barra de acoplamiento entre los mecanismos paralelos que componen las articulaciones dos y tres.}  \\
\hline
24 & \iconoImagen{PiezaRodamientosSandwichB} & RHAA3002 & 1 & \pieceDescription{ La función es equivalente a la de la pieza RHAA3001 siendo su simétrica. Además esta pieza integra el potenciómetro que realimenta la tercera articulación. }  \\
\hline
25 & \iconoImagen{PiezaRodamientosSandwichPotenciometro} & RHAA3003 & 1 & \pieceDescription{ La función es equivalente a la de la piezas RHAA3001 y RHAA3002. Además esta pieza integra el potenciómetro que realimenta la tercera articulación. }  \\
\hline 
26 & \iconoImagen{TapaPotenciometroA2} & RHAA3004 & 1 & \pieceDescription{Tapa y fija el potenciómetro a la pieza RHAA3002.} \\
\hline
27 & \iconoImagen{PiezaMetacrilato} & RHAA3005 & 2 & \pieceDescription{ Esta pieza y su simétrica componen la base estructural de la barra de acoplamiento entre los mecanismos paralelos, componiendo el eje de giro de la tercera articulación.}  \\
\hline
28 & \iconoImagen{PiezaUnionSandwich} & RHAA3006 & 4 & \pieceDescription{ Encaja sobre la pieza RHAA3005 y su simétrica fijándolas. } \\
\hline
29 & \iconoImagen{RealimentacionSandwich} & RHAA3007 & 1 & \pieceDescription{ Transmite el movimiento de la barra al potenciómetro que realimenta la segunda articulación.} \\
\hline
30 & \iconoImagen{SandwichAcoplamientoRodamientoBarra} & RHAA3008 & 1 & \pieceDescription{ Fija la pieza RHAA3007 al rededor de la barra cuadrada correspondiente. } \\
\hline
30 & \iconoImagen{extremoAuxiliar} & RHAA4001 & 1 & \pieceDescription{ Mantiene las barras del extremo a la distancia adecuada fijando los ejes sobre los que rota. Permite el acoplamiento del extremo añadiendo un grado de libertad (giro en eje Z) } \\
\hline
30 & \iconoImagen{extremoAuxiliarB} & RHAA4002 & 1 & \pieceDescription{ Complementa la pieza RHAA4001 } \\
\hline
30 & \iconoImagen{soporte_tablet_articulado} & RHAA4003 & 1 & \pieceDescription{ Encaja en el montaje de las piezas RHAA4001 y RHAA4002. Además está diseñado como encaje para el soporte de la tablet. }\\
\hline
30 & \iconoImagen{tapaSujetaTablet} & RHAA4004 & 1 & \pieceDescription{ Complementa la pieza RHAA4003 aprisionando el soporte de la tablet por ambos lados. } \\
\hline
\end{longtable}
\end{center}
\end{landscape}
\end{adjustwidth}