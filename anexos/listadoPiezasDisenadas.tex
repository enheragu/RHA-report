\completar

En este anexo se presenta una tabla listando las piezas que se han diseñado y fabricado para el proyecto. En la tabla \ref{app:listadoPiezas}1 se puede ver una miniatura de la pieza en cuestión, la cantidad necesaria de cada tipo, una estimación del peso (material consumido en su fabricación) así como una breve descripción de la pieza y/o proceso de fabricación de la misma. A su vez llevan asociado una referencia alfanumérica que se corresponde con los ficheros en formato digital entregados, asignada de la siguiente forma:

\begin{center} 
	\textit{RHA + ubicación + número\_de\_pieza}
\end{center}

\begin{itemize}
	\item \textit{ubicación}: A\$ (articulación), B\$ (barra). La letra va acompañada de un número (sustituyendo al caracter \$) que variará dependiendo de donde se encuentre la pieza en el montaje. A1 - articulación uno, B2 - barra 2 y así sucesivamente.
	\item \textit{número\_de\_pieza}: valor numérico que diferencia las piezas en la misma ubicación.
\end{itemize}

Nota: En caso de que la pieza se utilice en varias partes diferenciadas la referencia se tomará para la primera vez que aparece la pieza en orden ascendente (desde la barra 0 en adelante e igual desde la articulación 1).

Nota 2: la miniatura de las piezas no sigue ninguna escala concreta. Es una representación que permite un reconocimiento visual de la pieza, aunque no de su tamaño.


\begin{adjustwidth}{-10cm}{-10cm}
\begin{landscape}
\begin{center}
\begin{longtable}{|c|c|c|c|c|c|}
\caption{Listado de piezas diseñadas de fabricación propia}\\
\hline
\textbf{Num} & \textbf{Esquema Pieza} & \textbf{Referencia} & \textbf{Cantidad} & \textbf{Descripción} & \textbf{Peso Estimado}$^1$ \\
\hline
\endfirsthead
\multicolumn{5}{c}%
{\tablename\ \thetable\ -- \textit{Continuación de la página anterior}} \\
\hline
\textbf{Num} & \textbf{Esquema Pieza} & \textbf{Referencia} & \textbf{Cantidad} & \textbf{Descripción} & \textbf{Peso Estimado}$^1$ \\
\hline
\endhead
\multicolumn{5}{l}{\begin{minipage}{.8\linewidth}
	%do not draw the footnoterule
	\footnotesize{$^1$ El peso estimado se obtiene con el programa Cura \completar aplicando los parámetros de la tabla \ref{tab:listadoPiezas:param_impresion}. Este peso incluye el de los soportes necesarios para su impresión. Excepto en los casos donde se especifique lo contrario el juego de Parámetros utilizados es el 1.}
\end{minipage}} \\ 
\hline \multicolumn{5}{r}{\textit{Continua en la página siguiente}} \\
\endfoot
\hline
%\insertTableNotes
\endlastfoot
1 & \iconoImagen{Base} & RHAB1001  & 1  & blah & \completarCon{Corregir con peso al 90\%}193g \\
\hline
2 & \iconoImagen{RailA} & RHAB1002 & 1 & blah & 24g \\
\hline
3 & \iconoImagen{RailB} & RHAB1003 & 1 & blah & 24g \\
\hline
4 & \iconoImagen{PoleaMotor} & RHAB1004 & 2 & blah & 3g \\
\hline
5 & \iconoImagen{EncajeTuboInterior} & RHAA1005 & 1 & \completarCon{peso al 90\%} & 54g \\
\hline
6 & \iconoImagen{EncajeTuboExterior} & RHAA1006 & 1 & blah & 128g \\
\hline
7 & \iconoImagen{FijacionBarraPieB} & RHAB0001 & 1 & blah & blah \\
\hline
8 & \iconoImagen{FijacionBarraPie} & RHAB0002 & 1 & blah & blah \\
\hline
10 & \iconoImagen{RuedaMotorGiroZ} & RHAA1001 & 1 & blah & 18g \\
\hline
19 & \iconoImagen{AdaptadorPoleaNegra} & RHAB1007 & 2 & blah & 1g \\
\hline
19 & \iconoImagen{SeparadorPoleaNegraGrande} & RHAB1008 & 2 & blah & 20g \\
\hline
9 & \iconoImagen{SoportePlaca} & RHAB1009 & 1 & 10g \\
\hline
11 & \iconoImagen{UnionBarrasIntermediasA} & RHAA2001 & 1 & blah & blah \\
\hline
12 & \iconoImagen{UnionBarrasIntermediasB} & RHAA2002 & 1 & blah & blah \\
\hline
13 & \iconoImagen{RuedaTransmisionSuperior} & RHAA2003 & 1 & blah & 8g \\
\hline
14 & \iconoImagen{TapaPotenciometro} & RHAA2004 & 1 & blah & blah \\
\hline
15 & \iconoImagen{EngranajePotenciometro} & RHAA2005 & 1 & blah & blah \\
\hline
16 & \iconoImagen{EngranajeBarra} & RHAA2006 & 1 & blah & blah \\
\hline
17 & \iconoImagen{UnionBarrasSuperiorA} & RHAB2001 & 1 & blah & blah \\
\hline
18 & \iconoImagen{PoleaColumpioRedir} & RHAB2002 & 3 & blah & blah \\
\hline
19 & \iconoImagen{CubrePoleaColumpio} & RHAB2003 & 2 & blah & 6g \\
\hline
20 & \iconoImagen{CubrePoleaColumpioB} & RHAB2004 & 2 & blah & blah \\
\hline
21 & \iconoImagen{CubrePoleaRedireccionB} & RHAB3001 & 1 & blah & blah \\
\hline
22 & \iconoImagen{CubrePoleaRedireccion} & RHAB3002 & 1 & blah & blah \\
\hline
23 & \iconoImagen{PiezaRodamientosSandwich} & RHAA3001 & 2 & blah & 39g \\
\hline
24 & \iconoImagen{PiezaRodamientosSandwichB} & RHAA3002 & 1 & blah & 39g \\
\hline
25 & \iconoImagen{PiezaRodamientosSandwichPotenciometro} & RHAA3003 & 1 & blah & 44g \\
\hline 
26 & \iconoImagen{TapaPotenciometroA2} & RHAA3004 & 1 & 1 & 2g \\
\hline
27 & \iconoImagen{PiezaMetacrilato} & RHAA3005 & 2 & blah & - \\
\hline
28 & \iconoImagen{PiezaUnionSandwich} & RHAA3006 & 4 & blah & 9g \\
\hline
29 & \iconoImagen{RealimentacionSandwich} & RHAA3007 & 1 & blah & 24g \\
\hline
30 & \iconoImagen{SandwichAcoplamientoRodamientoBarra} & RHAA3008 & 1 & blah & blah \\
\hline
31 & \completarCon{Piezas del extremo provisionales} & blah & blah & blah & blah \\
\hline
\end{longtable}
\end{center}
\end{landscape}
\end{adjustwidth}

Se han utilizado diferentes juegos de parámetros según las piezas, se pueden ver los parámetros principales que influyen en el peso final de la pieza en la tabla \ref{tab:listadoPiezas:param_impresion} a continuación:
\begin{center}
	\begin{table}[H]
		\caption{Parámetros de las piezas para la estimación de peso}
		\label{tab:listadoPiezas:param_impresion}
		\begin{minipage}{\textwidth}
			\begin{tabular}{ |c|c|c| }
				\hline
				 & Parámetros 1 & Parámetros 2 \\
				\hline
				\begin{minipage} Grosor capa inferior\\ y superior \end{minipage}& 1.2 mm & 1.2mm \\ 
				\hline
				Grosor de pared & 1.2 mm & 1.2mm \\
				\hline
				Densidad interna & 30\% & 90\% \\
				%1 & \iconoImagen{Base}{0.2} & blah & \completar \\
				\hline
			\end{tabular}
		\end{minipage}
	\end{table}
\end{center}
