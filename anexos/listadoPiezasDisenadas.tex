\completar
\begin{center}
\begin{longtable}{|c|c|c|c|c|c|}
\caption{Listado de piezas diseñadas de fabricación propia}\\
\hline
\textbf{Num} & \textbf{Esquema Pieza} & \textbf{Referencia} & \textbf{Cantidad} & \textbf{Descripción} & \textbf{Peso Estimado}$^1$ \\
\hline
\endfirsthead
\multicolumn{5}{c}%
{\tablename\ \thetable\ -- \textit{Continuación de la página anterior}} \\
\hline
\textbf{Num} & \textbf{Esquema Pieza} & \textbf{Referencia} & \textbf{Cantidad} & \textbf{Descripción} & \textbf{Peso Estimado}$^1$ \\
\hline
\endhead
\multicolumn{5}{l}{\begin{minipage}{.8\linewidth}
	%do not draw the footnoterule
	\footnotesize{$^1$ El peso estimado se obtiene con el programa Cura \completar aplicando los parámetros de la tabla \ref{tab:listadoPiezas:param_impresion}. Este peso incluye el de los soportes necesarios para su impresión.}
\end{minipage}} \\ 
\hline \multicolumn{5}{r}{\textit{Continua en la página siguiente}} \\
\endfoot
\hline
%\insertTableNotes
\endlastfoot
1 & \iconoImagen{Base} & blah & 1  & blah & 193g \\
\hline
2 & \iconoImagen{RailA} & blah & 1 & blah & 24g \\
\hline
3 & \iconoImagen{RailB} & blah & 1 & blah & 24g \\
\hline
4 & \iconoImagen{PoleaMotor} & blah & 2 & blah & 3g \\
\hline
5 & \iconoImagen{EncajeTuboInterior} & blah & 1 & blah & 54g \\
\hline
6 & \iconoImagen{EncajeTuboExterior} & blah & 1 & blah & 128g \\
\hline
7 & \iconoImagen{FijacionBarraPieB} & blah & 1 & blah & blah \\
\hline
8 & \iconoImagen{FijacionBarraPie} & blah & 1 & blah & blah \\
\hline
10 & \iconoImagen{RuedaMotorGiroZ} & blah & 1 & blah & 18g \\
\hline
19 & \iconoImagen{AdaptadorPoleaNegra} & blah & 2 & blah & 1g \\
\hline
19 & \iconoImagen{AdaptadorPoleaNegra} & \completarCon{SeparadorPoleas} & 2 & blah & 20g \\
\hline
9 & \iconoImagen{SoportePlaca} & blah & 1 & 10g \\
\hline
11 & \iconoImagen{UnionBarrasIntermediasA} & blah & 1 & blah & blah \\
\hline
12 & \iconoImagen{UnionBarrasIntermediasB} & blah & 1 & blah & blah \\
\hline
13 & \iconoImagen{RuedaTransmisionSuperior} & blah & 1 & blah & 8g \\
\hline
14 & \iconoImagen{TapaPotenciometro} & blah & 1 & blah & blah \\
\hline
15 & \iconoImagen{EngranajePotenciometro} & blah & 1 & blah & blah \\
\hline
16 & \iconoImagen{EngranajeBarra} & blah & 1 & blah & blah \\
\hline
17 & \iconoImagen{UnionBarrasSuperiorA} & blah & 1 & blah & blah \\
\hline
18 & \iconoImagen{PoleaColumpioRedir} & blah & 3 & blah & blah \\
\hline
19 & \iconoImagen{CubrePoleaColumpio} & blah & 2 & blah & 6g \\
\hline
20 & \iconoImagen{CubrePoleaColumpioB} & blah & 2 & blah & blah \\
\hline
21 & \iconoImagen{CubrePoleaRedireccionB} & blah & 1 & blah & blah \\
\hline
22 & \iconoImagen{CubrePoleaRedireccion} & blah & 1 & blah & blah \\
\hline
23 & \iconoImagen{PiezaRodamientosSandwich} & blah & 2 & blah & 39g \\
\hline
24 & \iconoImagen{PiezaRodamientosSandwichB} & blah & 1 & blah & 39g \\
\hline
25 & \iconoImagen{PiezaRodamientosSandwichPotenciometro} & blah & 1 & blah & 44g \\
\hline 
26 & \iconoImagen{TapaPotenciometroA2} & blah & 1 & 1 & 2g \\
\hline
27 & \iconoImagen{PiezaMetacrilato} & blah & 2 & blah & - \\
\hline
28 & \iconoImagen{PiezaUnionSandwich} & blah & 4 & blah & 9g \\
\hline
29 & \iconoImagen{RealimentacionSandwich} & blah & 1 & blah & 24g \\
\hline
30 & \iconoImagen{SandwichAcoplamientoRodamientoBarra} & blah & 1 & blah & blah \\
\hline
\end{longtable}

\end{center}


\begin{center}
\begin{table}[H]
    \caption{Parámetros de las piezas para la estimación de peso}
    \label{tab:listadoPiezas:param_impresion}
    \begin{minipage}{\textwidth}
    \begin{tabular}{ |c|c|c|c| }
    \hline
    a & a & a & a \\ 
    %1 & \iconoImagen{Base}{0.2} & blah & \completar \\
    \hline
    \end{tabular}
    \end{minipage}
\end{table}
\end{center}
