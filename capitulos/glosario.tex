\newglossaryentry{Arduino}
{
  name={Arduino},
  description={"Arduino es una plataforma de prototipos electrónica de código abierto (open-source) basada en hardware y software flexibles y fáciles de usar.\cite{Arduino:2017}"},
  symbol={\ensuremath{\mathbb{R}}}
}


\newglossaryentry{Atom}
{
  name={Atom},
  description={Se trata de un editor de texto \glosario{open-source} multiplataforma con la opción de instalar mutiles paquetes para ampliar su funcionalidad.},
  symbol={\ensuremath{\mathbb{R}}}
}


\newacronym{API}{API}{Application Programming Interface}

\newglossaryentry{API_description}
{
  name={Application Programming Interface},
  description={en castellano se traduce como \ingles{interfaz de programación de aplicaciones}. No es más que un conjunto de funcionalidades definido por cierta librería o aplicación para ser utilizadas por otro software o directamente por el usuario},
  symbol={\ensuremath{\mathbb{R}}}
}


\newglossaryentry{bus_de_datos}
{
  name={bus de datos},
  description={BUSCAR DEFINICI??N EN LIBRO.},
  symbol={\ensuremath{\mathbb{R}}}
}


\newglossaryentry{cytronG15cube}
{
  name={Cytron G15 Cube servo},
  description={Se trata de un modelo de \glosario{smartservo} de la marca Cytron Technologies. (Descrito en más detalle en la sección \ref(sec:Electronica:Actuadores:G15))},
  symbol={\ensuremath{\mathbb{R}}}
}

\newglossaryentry{cpplint}
{
  name={cpplint},
  description={Es una herramienta generada por Google para revisar de forma automática el cumplimiento de las reglas de codificación del código. (https://github.com/google/styleguide/tree/gh-pages/cpplint)},
  symbol={\ensuremath{\mathbb{R}}}
}

\newglossaryentry{Cloc}
{
  name={Cloc},
  description={Es una herramienta de análisis de código para el conteo de líneas de código, comentarios, lineas en blanco, etc.(http://cloc.sourceforge.net/)},
  symbol={\ensuremath{\mathbb{R}}}
}

\newglossaryentry{depurar}
{
  name={depurar},
  description={Se trata del proceso en el cual, una vez generado el código de un proyecto se buscan, identifican y corrigen errores. A lo largo del texto se utilizar?? indistintamente este término o su versión en inglés \glosario{debug}},
  symbol={\ensuremath{\mathbb{R}}}
}

\newglossaryentry{debug}
{
  name={debug},
  description={Es el término en inglés para el proceso de \glosario{depurar}. A lo largo del texto se utilizarán ambos indistintamente},
  symbol={\ensuremath{\mathbb{R}}}
}

\newglossaryentry{doxygen}
{
  name={doxygen},
  description={\completar.}
}

\newacronym{GDL}{GDL}{Grados de libertad}
\newacronym{DOF}{DOF}{Degrees of Liberty}

\newglossaryentry{GDL_description}
{
  name={Grados de libertad},
  description={\completar En ingl??s se dice Degrees of Liberty o DOF},
  symbol={\ensuremath{\mathbb{R}}}
}


\newglossaryentry{lizard}
{
  name={lizard},
  description={Se trata de una herramienta para el análisis de complejidad en código de diferentes lenguajes (C/C++ entre ellos). Interesante por medir la complejidad ciclomática de cada función. (https://github.com/terryyin/lizard)},
  symbol={\ensuremath{\mathbb{R}}}
}

\newglossaryentry{open-source}
{
  name={open-source},
  description={Se trata de un ecosistema \glosario{open-source} disponible para \glosario{Atom} que configura un entorno gráfico para desarrollo de proyectos embebidos. En este caso se utiliza para la programación y test de proyectos basados en \glosario{Arduino}},
  symbol={\ensuremath{\mathbb{R}}}
}

\newglossaryentry{PlatformIO}
{
  name={PlatformIO},
  description={Se trata de un ecosistema \glosario{open-source} disponible para \glosario{Atom} que configura un entorno gráfico para desarrollo de proyectos embebidos. En este caso se utiliza para la programación y test de proyectos basados en \glosario{Arduino}},
  symbol={\ensuremath{\mathbb{R}}}
}

\newglossaryentry{PlatformIO_Test}
{
  name={PlatformIO Test},
  description={\completar},
  symbol={\ensuremath{\mathbb{R}}}
}

\newglossaryentry{Regulador-PID}
{
  name={Regulador PID},
  description={\completar},
  symbol={\ensuremath{\mathbb{R}}}
}

\newglossaryentry{servo}
{
  name={servo},
  description={Un 'servomotor' es un dispositivo actuador que tiene la capacidad de ubicarse en cualquier posición dentro de su rango de operación, y de mantenerse estable en dicha posición. Está formado por un motor de corriente continua, una caja reductora y un circuito de control. %(https://es.wikipedia.org/wiki/Servomotor_de_modelismo) 
  \completar },
  symbol={\ensuremath{\mathbb{R}}},
  plural={servos}
}

\newglossaryentry{smartservo}
{
  name={smart servo},
  description={\completar},
  symbol={\ensuremath{\mathbb{R}}},
  plural={smart servos}
}





