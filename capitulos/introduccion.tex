\chapter{Introducción} \label{chap:Introduccion}
\hrule
\vspace{3mm}

En este capítulo no deben faltar los siguientes apartados:


\section{Motivación del proyecto}



\section{Objetivos}

El objetivo que este Trabajo de Fin de Grado persigue es el del diseño, construcción y control de brazo robótico. Este proyecto está enmarcado bajo el proyecto Robohealth, proyecto financiado por el Ministerio de Economía y Competitvidad con el objetivo del diseño de sistemas robóticos y domóticos para entornos hospitalarios que mejoren el sistema de salud actual. (\completar http://robohealth.es/)
\\ 

El brazo robótico llevará en su extremo una \ingles{tablet} que, mediante visión artificial será capaz de reconocer gestos del paciente para llevar a cabo diferentes acciones. El brazo tendrá que ubicar la \ingles{tablet} en todo momento a una distancia y posición respecto del paciente que posibilite su correcto funcionamiento.
\\ 

El sistema estará en constante contacto con diferentes usuarios por lo que la seguridad de los mismos será prioritario, incluso desde la fase de diseño del prototipo. 
\\ 

Otra característica importante será el coste económico del prototipo, se busca desarrollar una solución de bajo coste que aumente la viabilidad de su implementación a gran escala.
\\

Para alcanzar estos objetivos se probarán diferentes estructuras y materiales que maximicen los recursos estructurales del prototipo. Una vez se tenga una estructura básica se procederá a añadir los actuadores y sensores para posteriormente implementar el software que lo controle.

\section{Materiales y software utilizados}\label{sec:Introduccion:materiales_software} 

Para la realización de las diferentes etapas del proyecto son necesarios los siguientes materiales. En este apartado se listan sin entrar en mucho detalle ya que se verán en capítulos posteriores.

\textbf{Elementos para el montaje Mecánico}:
    \begin{itemize}
        \item Barras de aluminio de sección cuadrada de 1/2"x1m (lado de la sección x longitud).
        \item Rodamientos 4x13?
        \item Rodamientos 3x13?
        \item Barras de acero cilíndricas de 4mm de diámetro
        \item Barras de acero cilíndricas de 3mm de diámetro
        \item Piezas impresas en impresora 3D. (Listado detallado en el Anexo \ref{app:listadoPiezas}.)
        \item Piezas de metacrilato cortadas con láser. (Listado detallado en el Anexo \ref{app:listadoPiezas}.)
        \item Tornillería: tornillos y tuercas
        \item Hilo de kevlar
        \item 2x - Poleas (LAS NEGRAS)
    \end{itemize}
    
    Se puede ver un análisis más detallado en el Anexo \ref{app:montajePiezas}, donde se detalla el montaje de las diferentes piezas que componen el brazo robótico.

\textbf{Electrónica}:
    \begin{itemize}
        \item 1x - Placa Arduino Uno
        \item 1x - Cytron G15 shield
        \item 3x - Servos G15 cube (Cytron)
        \item 2x - Potenciómetros
    \end{itemize}

\textbf{Software externo utilizado}:
    \begin{itemize}
        \item Autodesk Inventor 2016: Utilizado para el diseño y ensamblaje de la parte física del proyecto. 
        \item Atom con PlatformIO instalado: Utilizado para el desarrollo y verificación del software del proyecto.
        \item Lizard: Se trata de un software de análisis de la complejidad del código. Se compone de una serie de scripts en python que, al ser ejecutados devuelven un fichero con información referente a los ficheros de código sobre los que se invoca.
        \item Cloc: Herramienta para realizar el conteo de líneas en los ficheros de código. Permite diferenciar las lineas en blanco, comentarios, líneas de código, etc.
        \item cpplint: Análisis del cumplimiento de las reglas de codificación en el software. Es una herramienta desarrollada en python por Google para asegurar que los proyectos cumplen con sus reglas de codificación, que también se han seguido en este proyecto. Se pueden ver los aspectos más relevantes de las reglas de codificación en el Anexo \ref{app:codificacionSW}.
        \item doxygen:
    \end{itemize}



\section{Estructura del documento}

    A continuación y para facilitar la lectura del documento, se detalla el contenido de cada capítulo.
    
    \begin{itemize}
    \item En el capítulo \ref{chap:Introduccion} se realiza una introducción. \completar
    \item En el capítulo \ref{chap:estadoarte} se hace un repaso del actual estado del arte recogiendo algunas ideas que han inspirado el proyecto que aquí se describe.
    \item En el capítulo \ref{chap:Mecanica} se hace una descripción detallada del proceso de diseño de la parte mecánica del proyecto. Se resaltan algunos aspectos importantes así como consideraciones para su montaje.
    \item En el capítulo \ref{chap:Electronica} se habla sobre la electrónica involucrada en el proyecto. En ella se describe y justifica la utilización de sensores y actuadores, así como la placa utilizada y la etapa de potencia.
    \item En el capítulo \ref{chap:SW} se expone de forma extensiva el software desarrollado.
    \item El capítulo \ref{chap:Control} expone de forma detallada los distintos aspectos de diseño y desarrollo del control del brazo.
    \item En el capítulo \ref{chap:Resultados} se hace un repaso sobre los resultados generales a los que se ha llegado.
    \item En el capítulo \ref{chap:Gestion} se hace un análisis de la gestión del proyecto.
    \item Finalmente, en el capítulo \ref{chap:Conclusiones} se repasan las conclusiones y desarrollos futuros.
    \end{itemize}
    
    Como contenido adicional, al final del documento también se tienen los siguientes anexos:
    
    \begin{itemize}
    \item En el Anexo \ref{app:listadoPiezas} se listan las piezas impresas que se requieren para montar el prototipo así como algunas características importantes.
    \item En el Anexo \ref{app:codificacionSW} se concretan las reglas de codificación más relevantes que se han aplicado en el desarrollo del código.
    \item En el Anexo \ref{app:documentacion_software} se adjunta la documentación del software generada con la herramienta \glosario{doxygen}
    \end{itemize}
    
    Además es interesante repasar los términos que se incluyen en el Glosario y que aparecerán referenciados a lo largo del documento.
