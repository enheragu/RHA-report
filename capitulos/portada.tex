\begin{titlepage}
\begin{center}

%forma de introducir imágenes. el \\[0.5 cm] de final de línea introduce un salto de ese tamaño.
%width=1\textwidth indica el tamaño de la imágen (valores entre 0-1). 
 \includegraphics[width=1\textwidth]{figuras/cabecera.png}  \\[0.3 cm]

\Large UNIVERSIDAD POLITÉCNICA DE MADRID \\ [0.8 cm]

\Large ESCUELA TÉCNICA SUPERIOR DE INGENIERÍA Y DISEÑO INDUSTRIAL \\ [0.8 cm]

\LARGE Grado en Ingeniería Electrónica y Automática Industrial\\ [0.8 cm]

\LARGE \textbf{TRABAJO FIN DE GRADO}\\[0.8 cm]

\Huge \textsc{Diseño y construcción de un brazo robótico para el apoyo a la interactividad de personas enfermas}\\[0.8 cm]

\LARGE Enrique Heredia Aguado \\[1.7 cm]

%flushleft alinea a la izquierda el texto

\begin{multicols}{2} 
\begin{flushleft} \Large
\emph{Cotutor:} Alberto Brunete González \\
\emph{Departamento:} Ingeniería eléctrica, electrónica automática y física aplicada
\end{flushleft}

\begin{flushleft} \Large
\emph{Tutor:} Miguel Hernando Gutiérrez\\
\emph{Departamento:} Ingeniería eléctrica, electrónica automática y física aplicada
\end{flushleft}

\end{multicols} 

%rellena de blanco el resto de la página para escribir abajo del todo
\vfill

% Bottom of the page
{\large Madrid, Febrero 2018}

\end{center}
\end{titlepage}